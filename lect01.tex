\part*{Введение. Историческая справка}
\underline{Теория вероятностей}: математический анализ случайных явлений.\\
\underline{Теория случайных процессов}: стохастические модели и фактор времени.

\paragraph{Предпосылки к изучению}
\begin{itemize}
\item 1827, Р. Броун~-- броуновское движение частиц в воде~$\Rightarrow$ процесс броуновского движения
\item 1903, Л. Башелье~-- колебания курсов бумаг на бирже~$\Rightarrow$ процесс броуновского движения
\item 1906, А.А. Марков~-- анализ комбинаций гласных и согласных в романе <<Евгений Онегин>>~$\Rightarrow$ марковские цепи
\item 1903, Ф. Лундберг~-- модель деятельности страховой компании~$\Rightarrow$ пуассоновский процесс
\item 1873, Ф. Гальтон, Г. Ватсон~-- анализ вымирания аристократических фамилий в Великобритании~$\Rightarrow$ ветвящиеся процессы
\item Начало XX века, А. Эрланг~-- изучение загрузки телефонных сетей~$\Rightarrow$ теория массового обслуживания
\end{itemize}

\paragraph{Применения}
\begin{itemize}
\item Физика (стохастическое исчисление, теория гиббсовских полей)
\item Экономика (финансовая математика)
\item Биология
\end{itemize}