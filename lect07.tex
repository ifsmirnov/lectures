\section{Гауссовские случайные процессы}

\rightoffset{
\begin{definition}
Случайный вектор $\xi =  \seq{\xi}{1}{n}$ называется гауссовским, если его х.ф. имеет вид

$$\phi_\xi(t) = e^{i \langle a,t\rangle - \frac{1}{2} \langle \Sigma t, t\rangle }$$

где $a \in \myR^n$, а $\Sigma \in Mat(n\times n)$~-- симметрическая и неотрицательно определенная.
В этом случае пишут $\xi \sim N(a, \Sigma)$.
\end{definition}

\begin{theorem}[три эквивалентных определения]\forcenewline
\begin{enumerate}
\item Вектор $(\seq{\xi}{1}{n})$ гауссовский
\item $\xi = A\eta + b$ п.н., где $A \in Mat(n\times m)$, $b \in \myR^n$,
$\eta = (\seq{\eta}{1}{n})$, $\eta_i$~-- нез. $N(0,1)$
\item $\forall \tau \in \myR^n \langle \tau, \xi \rangle $ имеет одномерное нормальное распределение.
\end{enumerate}

\paragraph*{Свойства гаусс. векторов}
\begin{enumerate}
\item Смысл параметров: если $\xi \sim N(a, \Sigma)$, то $a = E\xi$, $\Sigma = D\xi$~--
матрица ковариаций.
\item Если $\xi$ гауссовский, то $A\xi$ гауссовский для всех матриц соответствуещего размера
(т.е. линейное преобразование гауссовского вектора также является гауссовским вектором).
\item Если $\xi = (\seq{\xi}1n) \sim N(a, \Sigma)$, то $\seq{\xi}1n$ нез. в совокупности
$\Leftrightarrow \Sigma$ диагональна $\Leftrightarrow \seq{\xi}1n$ некоррелированы.
\end{enumerate}
\end{theorem}
}

\begin{definition}
Действительный случайный процесс $(X_t, t \in T)$ называется \emph{гауссовским}, если
все его конечномерные распределения гауссовские:

$$\forall n \; \forall \seq{t}1n \in T \text{ вектор } (\seq{X}{t_1}{t_n}) \text{ гауссовский }$$
\end{definition}

\begin{definition}
Процесс $(X_t, t \in T)$ называется \emph{$L^2$-процессом}, если $\forall t \in T E|X_t^2| < +\infty$.

Функция $a(t) = EX_t$ называется \emph{функцией среднего} процесса $X_t$.

Функция $R(s,t) = cov(X_s, X_t)$ называется \emph{ковариационной функцией} процесса $X_t$.

Функция $K(s,t) = EX_sX_t$ называется \emph{корреляционной функцией} процесса $X_t$.
\end{definition}

\begin{remark}
распределение гауссовского вектора однозначно определяется матожиданием и матрицей ковариаций
$\Rightarrow$ конечномерные распределения гауссовского процесса определяются функцией среднего
и ковариационной функцией.
\end{remark}

\begin{definition}
Функция $f(x, y), x, y \in t$ называется \emph{неотрицательно определенной} на $T\times T$, если

$$\forall n \; \forall \seq t 1 n \in T \; \forall \seq x 1 n \in \myR
\sum_{i, j = 1}^{n} f(t_i, t_j) x_i x_j \geq 0$$
\end{definition}

\begin{lem}
Ковариационная и корреляционная функции случайного процесса симметричны и неотрицательно
определены.
\end{lem}
\begin{proof}
Пусть $X_t$~-- $L^2$-процесс, $K(s, t)$~-- его корреляционная функция. Тогда
$\forall n \; \forall \seq t 1 n \in T \; \forall \seq x 1 n \in \myR$

$$\sum_{i, j = 1}^{n} f(t_i, t_j) x_i x_j = \sum_{i,j=1}^n(EX_{t_i}X_{t_j})x_i x_j =
E \pars{\sum_{i,j=1}^n (x_iX_{t_i}) (x_j X_{t_j})} =
E\pars{\sum_{i=1}^n x_i X_{t_i}}^2 \geq 0$$

$\Rightarrow K(s,t)$ неотрицательно определена.

Теперь заметим, что  $R(s,t) = cov(X_s, X_t)$~-- это корреляционная функция для  $Y_t = X_t - EX_t$
$\Rightarrow$ она тоже неотрицательно определена.

Их симметричность очевидна.
\end{proof}

\begin{theorem}[о существовании гауссовских процессов]
Пусть $T$~-- некоторое множество, на нем задана функция $a(t)$, и $R(s, t)$~-- симметричная и
неотрицательно определенная функция на $T \times T$. Тогда существует вероятностное пространство
$(\Omega, \myF, P)$ и гауссовский процесс $(X_t, t \in T)$ т.ч. $a(t) = EX_t$ и
$R(s,t) =  cov(X_s, X_t)$.
\end{theorem}
\begin{proof}
Для $n \in \myN$, $\seq t 1 n  \in T$  рассмотрим вектор $a_{\seq t 1 n} =
(a(t_1), \dots, a(t_n))$, $\Sigma_{\seq t 1 n} = \Vert R(t_i, t_j) \Vert^n_{i,j=1}$.

Тогда $\Sigma_{\seq t 1 n}$ неотрицательно определена. Рассмотрим х.ф.

$$\phi_{\seq t 1 n}(\seq \lambda 1 n) = e^{i \langle a_{\seq t 1 n}, \lambda\rangle -
\frac12\langle\Sigma{\seq t 1 n}\lambda, \lambda\rangle}$$

Легко видеть, что такой набор х.ф. обладает свойствами симметрии и согласованности:

$$\langle a_{\seq t 1 n}, \lambda \rangle
= \sum_{k=1}^n a(t_k)(\lambda_k) = |\forall \sigma| =
\sum_{k=1}^n a(t_{\sigma(k)})(\lambda_{\sigma(k)})$$

$$\langle a_{\seq t 1 n}, (\seq \lambda 1 n)\rangle \vline_{\lambda_n = 0} =
\langle a_{\seq t 1 {n-1}}, (\seq \lambda 1 {n-1}) \rangle$$

Можно проверить, что для ковариационной функции также выполняются подобные равенства.

По теореме Колмогрова это означает, что $\exists (X_t, t \in T)$ т.ч.
$\phi_{\seq t 1 n}$~-- х.ф. $(X_{t_1}, \dots, X_{t_n}) \Rightarrow X_t$~-- гауссовский
процесс и $EX_t = a(t)$, $cov(X_s, X_t) = R(s,t)$.
\end{proof}

\subsection{Процесс броуновского движения (винеровский процесс)}

\begin{definition}
Случайный процесс $(W_t, t \in 0)$ называется \emph{винеровским}, если
\begin{enumerate}
\item $W_0 = 0$ п.н.
\item $W_t$ имеет независимые приращения
\item $W_t - W_t \sim N(0, t-s)$, $t \geq s$
\end{enumerate}
\end{definition}

\begin{statement}
Винеровский процесс существует.
\end{statement}
\begin{proof}
По критерию существования процессов с независимыми приращениями достаточно проверить, что
$\forall s \leq u \leq t$ (опускаем аргумент у х.ф.)

$$\phi_{W_t W_s}  = \phi_{W_t-W_u} \phi_{W_u-W_s}$$

Но т.к. $W_t-W_s \sim N(0, t-s)$

$$\phi_{W_t-W_s}(\tau) = e^{-\frac12 \tau^2(t-s)}$$

Очевидно, свойство выполнено и процесс существует.
\end{proof}

\begin{theorem}[эквивалентное определение винеровского процесса]
Процесс $(W_t, t \geq 0)$  является винеровским $\Leftrightarrow$
\begin{enumerate}
\item $W_t$ гауссовский
\item $\forall t \geq 0 \; EW_t = 0$
\item $cov(W_s, W_t) = min(s, t)$
\end{enumerate}
\end{theorem}
\begin{proof} \forcenewline
$(\Rightarrow)$ $W_t \sim N(0, t) \Rightarrow EW_t = 0$. Посчитаем ковариационную функцию:

\longformula{
cov(W_s, W_t) = |t > s| = cov(W_s, W_t - W_s + W_s) = \\
cov(W_s, W_t-W_s) + cov(W_s, W_s) = 0 + DW_s = s = min(s, t)
}

Пусть $0 \geq t_1 \geq \dots \geq t_n$, $\xi = (\seq{W}{t_1}{t_n})$.
Вектор $\eta = (W_{t_1}, W_{t_2} - W_{t_1}, \dots, W_n - W_{n-1})$ имеет независимые нормальные
компоненты $\Rightarrow \eta$~-- гауссовский вектор.  Очевидно, $\xi = A\eta$ (выписать A!), значит,
$\xi$ также является гауссовским $\Rightarrow W_t$~-- гауссовский процесс.

$(\Leftarrow)$ Почему такой процесс существует? По теореме достаточно проверить, что $min(s, t)$~--
неотрицательно определенная функция.
Для этого можно заметить, что $min(s,t)$~-- это ковариационная функция для пуассоновского процесса
интенсивности $1$. Значит, она неотрицательно определена.

$$EW_t = 0, DW_t = min(t, t) = t \Rightarrow DW_0 = 0, EW_0 = 0 \Rightarrow W_0 = 0 \text{ п.н.}$$

Пусть $0 \leq t_1 < \dots < t_n$ фиксированы. Тогда $(\seq{W}{t_1}{t_n})$~-- гаусс. вектор
$\Rightarrow \xi = (W_{t_n} - W_{t_{n-1}}, \dots, W_{t_2} - W_{t_1}, W_{t_1})$~-- тоже гауссовский
как линейное преобразование гауссовского вектора. Значит, для независимости компонент $\xi$
достаточно проверить, что они некоррелированны.

Пусть $i > j, t_0 = 0$.

\longformula{
cov(W_{t_i} - W_{t_{i-1}}, W_{t_j} - W_{t_{j-1}}) = \\
cov(W_{t_i}, W_{t_j}) - cov(W_{t_{i-1}}, W_{t_j}) -
cov(W_.... = \\
t_j - t_j - t_{j-1} + t_{j-1} = 0
}

$\Rightarrow W_t$ имеет независимые приращения.

$W_t - W_s \sim N(a, \sigma^2)$.

$$a = E(W_t-W_s) = 0$$
\longformula{
\sigma^2 = D(W_t-W_s) = cov(W_t-W_s, W_t-W_s) = \\
cov(W_t, W_t) + cov(W_s, W_s) - 2cov(W_t,W_s) =|t > s| = \\
t - 2s + s = t-s
}
\end{proof}

\subsection{Непрерывность траекторий винеровского процесса}

\begin{definition}
Процесс $(Y_t, t \in T)$ называется \emph{модификацией} процесса $(X_t, t \in T)$, если
$\forall t \in T$

$$P(Y_t = X_t) = 1$$
\end{definition}
\begin{theorem}[Колмогорова о существании непрерывной модификации, б/д]
Пусть процесс $(X_t, t \in [a,b])$ таков, что для некоторых $C, \alpha, \eps > 0$ выполнено:

$$\forall t, s \in [a,b] \; E|X_t-X_s|^\alpha \leq C|t-s|^{1+\eps}$$

Тогда у $X_t$ существует модификация $Y_t$, все траектории которой непрерывны.
\end{theorem}
\begin{corollary}
У $W_t, t \geq 0$ существует непрерывная модификация.
\end{corollary}
\begin{proof}
$$W_t-W_s \sim N(0, |t-s|) \Rightarrow E(W_t-W_s)^4 = 3(|t-s|)^2$$
Значит, у $W_t$ существует непрерывная модификация на любом конечном отрезке.

Пусть $W_t^{(n)}$~-- непрерывная модификация $W_t$ на отрезке $[n, n+1]$, $n \in \myZ_+$.
Рассмотрим процесс

$$X_t(\omega) = \{W_t^{(n)}(\omega), t \in [n, n+1)\}$$

Разрывы траекторий $X_t$ возможны только в целых точках времени, когда $W_{n+1}^{(n)}(\omega) \neq
W_{n+1}^{(n+1)}(\omega)$. Но $W_t^{(n)}$ и $W_t^{(n+1)}$~-- модификации $W_t$, значит,

\longformula{
P(W_{n+1}^{(n)} = W^{}_{n+1}) = 1 = P(W_{n+1}^{(n+1)} = W^{}_{n+1})
\Rightarrow P(\exists n: W_{n+1}^{(n)} \neq W_{n+1}^{(n+1)}) = 0
}

Теперь рассмотрим

$$\tilde X_t(\omega) = \left\lbrace
\begin{array}{l}
X_t(\omega), \text{ если } \forall n \; W_{n+1}^{(n)}(\omega) = W_{n+1}^{(n+1)}(\omega) \\
0, \text{ иначе }
\end{array}
\right.
$$

Это и будет искомая непрерывная модификация.
\end{proof}

\begin{remark}
Условие $\eps > 0$ в теореме Колмогорова существенно.
\end{remark}
\begin{proof}
Пусть $(N_t, t \geq 0)$~-- пуассоновский процесс. Тогда
$$E(N_t-N_s) = \lambda|t-s|$$

Значит, $N_t$ удовлетворяет условию теоремы Колмогорова с $\eps = 0$. Но траектории
$N_t$ разрывны почти наверное на всем $\myR_+$ и разрывны с положительной вероятностью на любом конечном отрезке.
\end{proof}

\begin{remark}
Всюду далее, где это необходимо, считаем, что нам задана непрерывная модификация винеровского
процесса.
\end{remark}

\begin{theorem}[Пэли, Зигмунд, Винер, б/д]
С вероятностью $1$ траектория винеровского процесса не дифференцируема ни в одной точке $\myR_+$.
\end{theorem}
\begin{theorem}[Закон повторного логарифма]
$$P \pars{ \limsup_{t \to +\infty} \frac{W_t}{\sqrt{2t \ln \ln t}} = 1 } = 1$$

$$ \left| \limsup_{t \to +\infty} f(t) := \lim_{t\to +\infty} \sup_{s \geq t} f(s) \right| $$
\end{theorem}
\begin{corollary}
$$P \pars{ \liminf_{t \to +\infty} \frac{W_t}{\sqrt{2t \ln \ln t}} = -1 } = 1$$
\end{corollary}
\begin{proof}
Рассмотрим $Y_t = -W_t$~-- тоже винеровский процесс.
\end{proof}

ЗПЛ означает, что с вероятностью $1$, начиная с некоторого момента $t_0 = t_0(\eps, \omega)$
траектория $W_t$ находится внутри области, ограниченной кривыми $\pm (1+\eps)\sqrt{2t\ln\ln t}$.
В то же время $\forall \eps > 0$ траектория бесконечно много раз в обе стороны выходит из области,
ограниченной кривыми $\pm (1-\eps)\sqrt{2t\ln\ln t}$, после момента времени $T$.

\begin{corollary}[локальный ЗПЛ]
$$\limsup_{t \to +0} f(t) = \lim_{t \to +0} \sup_{s \leq t} f(s)$$
$$P \pars{\limsup_{t \to +0} \frac{W_t}{\sqrt {2t \ln \ln \frac{1}{t}}} = 1} = 1$$
$$P \pars{\liminf_{t \to +0} \frac{W_t}{\sqrt {2t \ln \ln \frac{1}{t}}} = -1} = 1$$
\end{corollary}
\begin{proof}
Рассмотрим процесс $B_t = t \cdot W_{\frac{1}{t}} \ind{t>0}$. Покажем, что $B_t$~-- винеровский.

\begin{enumerate}
\item $\forall \seq t 1 n \geq 0$ имеем $(\seq{B}{t_1}{t_n})$~-- линейное преобразование вектора
$\pars{\seq{W}{\frac{1}{t_1}}{\frac{1}{t_n}}}$, значит, это гауссовский вектор. Тогда $B_t$~--
гауссовский процесс.

\item $EB_t = 0 \forall t \geq 0$.

\item $cov(B_t, B_s) = t s cov(W_\frac{1}{t}, W_\frac{1}{s}) = \frac{ts}{max(t,s)} = min(t, s)$.
\end{enumerate}

Значит, по теореме об эквивалентном определении $B_t$~-- винеровский процесс. Тогда
\longformula{
\limsup_{t \to +0} \frac{W_t}{\sqrt{2t \ln \ln \frac{1}{t}}} = |s = \frac{1}{t}| =\\
\limsup_{s \to +\infty}\frac{W_\frac{1}{s}}{\sqrt{2 \frac{1}{s} \ln \ln s}} =\\
\limsup_{s \to +\infty}\frac{s W_{\frac{1}{s}}}{\sqrt{2s \ln \ln s}} =\\
\limsup_{s \to +\infty}\frac{B_s}{\sqrt{2s \ln \ln s}} = 1 \text{ п.н. по ЗПЛ. }
}
\end{proof}

\begin{theorem}[марковское свойство $W_t$]
Пусть $(W_t, t \geq 0)$~-- винеровский процесс. Тогда $\forall a > 0$ процесс
$X_t = W_{t+a} - W_a$ тоже винеровский.

\underline{Вопрос:} можно ли заметить $a$ на случайное время?
\end{theorem}
