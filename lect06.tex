\section{Процессы с независимыми приращениями}
\begin{definition}
Пусть $(X_t, t \geq 0)$~-- действительный процесс. Он называется \emph{процессом с
независимыми приращениями}, если $\forall n \in \myN \; \forall \; 0 \leq t_1 < \dots < t_n$
случайные величины $X_{t_1}, X_{t_2 - t_1}, \dots, X_{t_n - t_{n-1}}$ независимы в совокупности.
\end{definition}

\begin{theorem}[о существовании процессов с независимыми приращениями]
Пусть $Q_0$~-- вер. мера на $(\myR, \myB(\myR))$ с х.ф. $\phi_0$,
и $\forall s,t$ задана вер. мера $Q_{s,t}$ на $(\myR, \myB(\myR))$ с х.ф. $\phi_{s,t}$.
Тогда процесс с независимыми приращениями
$$(X_t, t \geq 0) \text{ т.ч. } X_0 \overset{d}{=} Q_0, X_t - X_s \overset{d}{=} Q_{s,t}, s < t$$
существует тогда и только тогда, когда
$$\forall \; 0 \leq s < u < t \; \phi_{s,t}(\tau) = \phi_{s,u}(\tau) \phi_{u,t}(\tau)$$
\end{theorem}
\begin{proof} \forcenewline
$(\Rightarrow)$ $\exists X_t \Rightarrow \forall \; s < u < t X_t-X_u$ и
$X_u-X_s$ независимы $\Rightarrow \phi_{s,t}(\tau) = \phi_{X_t-X_s}(\tau)=
\phi_{X_t-X_u+X_u-X_s}(\tau) =
\phi_{X_t-X_u}(\tau) \phi_{X_u-X_s}(\tau) =
\phi_{u,t}(\tau) \phi_{s,u}(\tau)$.

$(\Leftarrow)$ Пусть $X_t$ существует. Тогда $\forall 0 < t_1 < \dots < t_n$
случайные величины. Компоненты вектора $(X_{t_n}-X_{t_{n-1}}, \dots, X_{t_1}-X_0, X_0)$
независимы. Обозначим этот вектор как $\xi$.

$$\phi_\xi(\seq \lambda n 0) =
\phi_{X_{t_n}-X_{t_{n-1}}}(\lambda_n) \cdot \ldots \cdot \phi_{X_0}(\lambda_0) =
\phi_{t_n,t_{n-1}}(...)$$

Рассмотрим $\eta = (X_tn, ..., X_0)^t$. Ясно,  что $\eta = A\xi$, где

$A = $ ---матр.1---

$\Rightarrow \phi_\eta(\lambda vector) = Ee^{i <\eta, \lambda v>} = $
записи на бумажке

Теперь забудем про то, что процесс существует.
$\forall n \forall t_1 < ... t_n$ зададим $\phi_t1..tn$ по формуле $(*)$.
$TODO$
%
% Проверим условия следствия: пусть $l_m = 0$.
% ---опять записи на бумажке---
%
% Смотрим на две подчеркнутые штуки.
%
% По условию $\phi_tm-1 tm(\tau) \phi_tm tm+1(\tau) = \phi_tm-1 tm+1(\tau)$.
%
% --- тут то, что после равно в кружочке --
%
%
% Согласно следствию, $\exists (X_t, t \geq 0)$ т.ч.
% $\phi_tn .. t1$~-- х.ф. $(X_tn .. X_t1)$. По построению $\phi_t1..tn$
% это означает, что $X_t$ имеет независимые приращения, $X_t - X_s \overset{d}{=}
% Q_{s,t}, t > s$, и $X_0 \overset{d}{=} Q_0$.
\end{proof}

%
% \begin{remark}
% \begin{itemize}
% \item Процесс с независимыми приращениями с дискретным временем~-- это случайное блуждание.ы
% \item Если $\phi_{\xi+\eta}(t) = \phi_\xi(t) \phi_\eta(t)$, то это не значит,
% что $\xi$ и $\eta$ независимы.
% \end{itemize}
% \end{remark}
%
% \subsection{Пуассоновский процесс}
% \begin{definition}
% Процесс $N_t, t \geq 0$ называется \emph{пуассоновским процессом интенсовности $\lambda > 0$},
% если
% \begin{enumerate}
% \item $N_0 = 0$ п.н.
% \item $N_t$ имеет независимые приращения
% \item $N_t - N_s \tilde Pois(\lambda(t-s)), t > s \geq 0$
% \end{enumerate}
% \end{definition}
%
% \begin{statement}
% Пуассоновский процесс существует.
% \end{statement}
% \begin{proof}
% Пусть $\phi_s,t$~-- х.ф. $Pois(\lambda(t-s))$. Тогда
% $$\phi_{s,t}(\tau) = ... = e^{\lambda(t-s)(e^{i\tau} - 1)}$$
%
% Отсюда видно, что $\forall s < u < t$
% $$\phi_{s,u}(\tau) \phi_{u,t}(\tau) = \phi_{s,t}(\tau)$$
% Значит, по критерию существования пуассоновский процесс существует.
% \end{proof}
%
% \paragraph{Свойства траекторий}
% \begin{enumerate}
% \item Целочисленные: $N_t \in \myZ_+, t \geq 0$
% \item Неубывающие: $N_t - N_s \geq 0, t > s$
% \end{enumerate}
% Вопросы:
% \begin{enumerate}
% \item размер скачков?
% \item распределение моментов скачков
% \end{enumerate}
%
% \begin{theorem}[явная конструкция пуассоновского процесса]
% Пусть $\{\xi_n, n \in \myN\}$~-- нез. $\tilde Exp(\lambda)$, $S_n =  \xi_1 + \dots + \xi_n$.
% Тогда процесс восстановления
% $$X_t = \sup\{n: S_n \leq t\}$$
% является пуассоновским процессом интенсивности $\lambda$.
% \end{theorem}
% \begin{proof}
% Пусть $X_0 = 0$.
% Рассмотрим вектор $(S_n, \dots, S_1)$. Он имеет плотность:
% $$P_{S_n..S_1}(x_1..x_1) = P_{\xi_n .. \xi_1}(X_n-x_{n-1}, ..., X_2-X_1, X_1) =
% |\xi_i \tilde Exp(\lambda)| = \prod_{k=1}^{n} \lambda e^{-(x_k-x_k-1)\lambda} \ind{X_k > X_k-1} =
% \lambda^n e^{-\lambda X_n} \ind{x_n > \dots > x_1 > 0}$$
%
% Зафиксируем $0 < t_1 < ... < t_n$ и $k_n \geq ... \geq k_1, k_i \in \myZ_+$. Рассмотрим
% $$P(X_tn - X_tn-1 = k_n-k_n-1, ..., X_t1 = k1)$$
%
% Что такое X_tn - X_tn-1? Это количество скачков, которые совершает процесс на отрезке $tn-1, tn$.
%
% $$ = P(S_k1 < t1, \{S_k1+1, ... S_k2\} \in (t1, t2], ..., \{S_{k_{n-1}+1}, .., S_kn\}
% \in (t_n-1, tn], S_{k_n+1} > t_n) =$$
% |подставляем плотность $p_{s1..s_{k_n + 1}} =$|
% интегрируем в области, написанной на бумажке.
%
% дальше много всего на бумажке
%
% Следовательно, $X_t$ имеет независимые приращения, и эти приращения пуассоновские:
%
% $$X_{t_j} - X_{t_{j-1}} \tilde Pois(\lambda(t_j - t_{j-1}))$$
%
% Значит, $X_t$~-- это пуассоновский процесс.
% \end{proof}
% \begin{corollary}
% Пусть $N_t$~-- пуассоновский процесс интенсивности $\lambda > 0$, $Y_n$~--
% момент $n$-того скачка. Тогда
% \begin{enumerate}
% \item П.н. все скачки $N_t$ имеют размер $1$
% \item $Y_n \tilde \Gamma(\lambda, n)$, записать плотность
% \item $Y_n-Y_{n-1}, ..., Y_1$~-- нез. $Exp(\lambda)$.
% \end{enumerate}
% \begin{proof}
% \begin{enumerate}
% \item Пусть $X_t$~-- явная конструкция из теоремы.
%
% $P($У $N_t$ есть скачок размера $\geq 2) = P($ у $X_t$ есть скачок размера $\geq 2) =
% P(\exists n: S_n = S_{n+1}) = P(\exists n; \xi_n = 0) = 0$, т.к. $\xi_n$ абсолютно непрерывны.
%
% \item 2 и 3 выполнены для $X_t$ по построению $\Rightarrow$ верны и для $N_t$.
% \end{enumerate}
% \end{proof}
% \end{corollary}
% \end{document}
