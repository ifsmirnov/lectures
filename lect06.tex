% Title

\ifdefined\HeaderIncluded
\else

\newcommand{\HeaderIncluded}{}

\documentclass[a4paper, 10pt]{article}
\usepackage[utf8]{inputenc}
\usepackage[english, russian]{babel}
\usepackage{amsmath, amsfonts, amssymb, amsthm}
\usepackage{tikz} % some graphics
%\usepackage[indentheadings]{russcorr}
\usepackage{geometry}
\usetikzlibrary{arrows}
\setlength{\parindent}{0pt}


\geometry{left=2cm}
\geometry{right=2cm}
\geometry{top=2cm}
\geometry{bottom=2cm}

\hfuzz=18pt

% Commands

\newcommand*{\hm}[1]{#1\nobreak\discretionary{}%
{\hbox{$\mathsurround=0pt #1$}}{}} % a\hm=b makes "=" carriable to the next line with duplication of the sign

\newcommand{\combus}[2]{\left(\begin{array}{c}#1 \\ #2 \end{array} \right)} % american style for C_n^k
\newcommand{\combru}[2]{C_{#1}^{#2}} % russian C_n^k
\newcommand{\comb}[2]{\combru{#1}{#2}}
\newcommand{\myN}{\mathbb{N}} % nice letters for common number sets
\newcommand{\myZ}{\mathbb{Z}}
\newcommand{\myR}{\mathbb{R}}
\newcommand{\myC}{\mathbb{C}}
\newcommand{\myQ}{\mathbb{Q}}
\newcommand{\myE}{\mathcal{E}} % basis
\newcommand{\myF}{\mathcal{F}} % basis
\newcommand{\myB}{\mathcal{B}} % set system (random processes)
\newcommand{\myM}{\mathcal{M}} % set system (random processes)
\newcommand{\mysetP}{\mathcal{P}}
\newcommand{\mysetR}{\mathcal{R}}
\newcommand{\mysetM}{\mathcal{M}}
\newcommand{\mysetN}{\mathcal{N}}
\newcommand{\mysetX}{\mathcal{X}}
\newcommand{\walls}[1]{\left | #1 \right |} % |smth_vertically_large|
\newcommand{\pars}[1]{\left( #1 \right)} % (smth_vertically_large)
\newcommand{\class}[1]{[ #1 ]} % [smth_vertically_large]
\newcommand{\bra}[1]{\langle #1 \rangle} % brackets for span of vectors, eg. <e_1, ..., e_k>
\newcommand{\myset}[1]{\left\{ #1 \right\}} % because { and } are special symbols in TeX
\newcommand{\mysetso}[2]{\myset{#1 \mid #2}}
\newcommand{\myforall}[1]{\forall #1:}
\newcommand{\myexists}[1]{\exists #1:\ }
\newcommand{\forcenewline}{\ \newline}
\newcommand{\reversein}{\ni}
\renewcommand{\phi}{\varphi}
\newcommand{\eps}{\varepsilon}
\newcommand{\ind}[1]{\mathrm{I}\{ #1 \}}
\newcommand{\myliminf}[1]{\underset{ #1 }{\underline{\lim}}}
\newcommand{\mylimsup}[1]{\underset{ #1 }{\overline{\lim}}}
\newcommand{\seq}[3]{#1_{#2}, \dots, #1_{#3}}

%
% Example text:
% M = {x^2 | x is prime}
%
% Corresponding markup:
% $$ \mysetM = \mysetso{x^2}{x \text{ is prime}} $$
%
\renewcommand{\leq}{\leqslant}
\renewcommand{\geq}{\geqslant}
\newcommand{\myempty}{\varnothing}
\renewcommand{\emptyset}{\varnothing}
\newcommand{\myand}{\;\; \hm\& \;\;}
\newcommand{\myor}{\; \hm\vee \;}
\newcommand{\conj}[1]{\overline{#1}} % complex conjugation
\newcommand{\mycirc}{\circ}
\newcommand{\poly}[2]{#1 [ #2 ]} % ring of polynomials
\newcommand{\Rx}{\poly{\myR}{x}}
\newcommand{\Cx}{\poly{\myC}{x}}
\newcommand{\mydim}[1]{\dim #1}
\newcommand{\mycodim}[1]{\text{codim} #1}
\newcommand{\coords}[2]{\pars{#1}_{#2}}
\newcommand{\myrank}[1]{\text{rank} #1}
\newcommand{\mysegment}[2]{[#1, #2]}
\newcommand{\myinterval}[2]{(#1, #2)}
\newcommand{\mypair}[2]{(#1, #2)}
\newcommand{\myfunc}[3]{#1\!:\,#2 \hm\to #3} % TODO: make spaces between elements of this tag look better
\newcommand{\suchthat}{\!:\,}
\newcommand{\mydef}[1]{\emph{#1}}

% Arrows
\newcommand{\myright}{\;\hm\Rightarrow\;}
\newcommand{\myleft}{\;\hm\Leftarrow\;}
\newcommand{\myleftright}{\;\hm\Leftrightarrow\;}
% \newcommand{\vect}[1]{\overrightarrow{\vphantom{b}#1}}
\newcommand{\vect}[1]{\Vec{\vphantom{b}#1}}

% Sums, prods and other things with \limits
\newcommand{\mysum}{\sum\limits}
\newcommand{\myprod}{\prod\limits}
\newcommand{\mylim}{\lim\limits}
\newcommand{\mysup}{\sup\limits}
\newcommand{\myinf}{\inf\limits}
\newcommand{\mybigcup}{\bigcup\limits}
\newcommand{\mybigcap}{\bigcap\limits}
\newcommand{\mybigor}{\bigvee\limits}
\newcommand{\mybigand}{\bigwedge\limits}
\newcommand{\mybigxor}{\bigoplus\limits}
\newcommand{\xor}{\oplus}
\newcommand{\eq}{\equiv}

\newcommand{\smatrix}[1]{\begin{smallmatrix}#1\end{smallmatrix}}
\newcommand{\psmatrix}[1]{\pars{\begin{smallmatrix}#1\end{smallmatrix}}}
\newcommand{\wsmatrix}[1]{\walls{\begin{smallmatrix}#1\end{smallmatrix}}}


\newcommand{\abs}[1]{|#1|}
% Operators
\DeclareMathOperator{\pr}{pr} % Проекция
\DeclareMathOperator{\Tr}{Tr} % След
\DeclareMathOperator{\mychar}{char} % Характеристика поля
\DeclareMathOperator{\id}{id} % Тождественное преобразование
\DeclareMathOperator{\rk}{rk} % Ранг
\DeclareMathOperator{\Kernel}{Ker}
\DeclareMathOperator{\Image}{Im}
\DeclareMathOperator{\sign}{sign} % Знак
\DeclareMathOperator{\sgn}{sign} % Знак
\DeclareMathOperator{\const}{const}

% Special theorems
\newtheorem*{theorem-menelaus}{Теорема Менелая}

% Theorems
\newtheorem*{theorem-star}{Теорема}
\newtheorem{theorem}{Теорема}[section]
\newtheorem*{theorem-definition-star}{Теорема-определение}
\newtheorem*{corollary-star}{Следствие}
\newtheorem{corollary}{Следствие}[section]
\newtheorem*{property-star}{Свойство}
\newtheorem*{properties}{Свойства}
\newtheorem{property}{Свойство}
\newtheorem*{lem-star}{Лемма}
\newtheorem{lem}{Лемма}[section]
\newtheorem*{proposition-star}{Предложение}
\newtheorem{proposition}{Предложение}
\newtheorem{stage}{Этап}
\newtheorem*{statement}{Утверждение}
\newtheorem*{designation}{Обозначение}
\newtheorem*{usage}{Использование}

\theoremstyle{remark}
\newtheorem*{remark}{Замечание}

\theoremstyle{definition}
\newtheorem{problem}{Задача}
\newtheorem{exercise}{Упражнение}

\theoremstyle{definition}
\newtheorem*{definition-star}{Определение}
\newtheorem{definition}{Определение}[section]

\theoremstyle{definition}
\newtheorem*{example-star}{Пример}
\newtheorem{example}{Пример}
\newtheorem*{examples}{Примеры}
\newtheorem{algorithm}{Алгоритм}
\newtheorem{case}{Случай}
\newtheorem*{case-star}{Случай}

\newcommand{\rightoffset}[1]{
\begin{flushright} \begin{minipage}{0.9\textwidth}
\begin{tabular}{|l} \begin{minipage}{\textwidth}
#1
\end{minipage}
\end{tabular}
\end{minipage}
\end{flushright}
}

\newcommand{\longformula}[1]{
$$\begin{array}{l}
#1
\end{array}$$
}


% Style
\newcommand{\tocstyle}{\setlength{\parindent}{0ex} \setlength{\parskip}{0ex}}
\newcommand{\mainstyle}{\setlength{\parindent}{0ex} \setlength{\parskip}{1ex}}

\mainstyle
\setcounter{secnumdepth}{2}
\renewcommand{\thesubsection}{\arabic{subsection}}


% Titles of lectures
\newcommand{\mylecture}[1]{\setcounter{secnumdepth}{-1} \section{#1} \setcounter{secnumdepth}{2} \setcounter{subsection}{0} \setcounter{corollary}{0} \setcounter{definition}{0} \setcounter{theorem}{0}}

\fi


\ifdefined\Main\else
\begin{document}
\fi

\section{Процессы с независимыми приращениями}
\begin{definition}
Пусть $(X_t, t \geq 0)$~-- действительный процесс. Он называется \emph{процессом с
независимыми приращениями}, если $\forall n \in \myN \; \forall \; 0 \leq t_1 < \dots < t_n$
случайные величины $X_{t_1}, X_{t_2 - t_1}, \dots, X_{t_n - t_{n-1}}$ независимы в совокупности.
\end{definition}

\begin{theorem}[о существовании процессов с независимыми приращениями]
Пусть $Q_0$~-- вер. мера на $(\myR, \myB(\myR))$ с х.ф. $\phi_0$,
и $\forall s,t$ задана вер. мера $Q_{s,t}$ на $(\myR, \myB(\myR))$ с х.ф. $\phi_{s,t}$.
Тогда процесс с независимыми приращениями
$$(X_t, t \geq 0) \text{ т.ч. } X_0 \overset{d}{=} Q_0, X_t - X_s \overset{d}{=} Q_{s,t}, s < t$$
существует тогда и только тогда, когда
$$\forall \; 0 \leq s < u < t \; \phi_{s,t}(\tau) = \phi_{s,u}(\tau) \phi_{u,t}(\tau)$$
\end{theorem}
\begin{proof} \forcenewline
$(\Rightarrow)$ $\exists X_t \Rightarrow \forall \; s < u < t X_t-X_u$ и 
$X_u-X_s$ независимы $\Rightarrow \phi_{s,t}(\tau) = \phi_{X_t-X_s}(\tau)=
\phi_{X_t-X_u+X_u-X_s}(\tau) =
\phi_{X_t-X_u}(\tau) \phi_{X_u-X_s}(\tau) =
\phi_{u,t}(\tau) \phi_{s,u}(\tau)$.

$(\Leftarrow)$ Пусть $X_t$ существует. Тогда $\forall 0 < t_1 < \dots < t_n$
случайные величины. Компоненты вектора $(X_{t_n}-X_{t_{n-1}}, \dots, X_{t_1}-X_0, X_0)$
независимы. Обозначим этот вектор как $\xi$.

$$\phi_\xi(\seq \lambda n 0) =
\phi_{X_{t_n}-X_{t_{n-1}}}(\lambda_n) \cdot \ldots \cdot \phi_{X_0}(\lambda_0) =
\phi_{t_n,t_{n-1}}(...)$$

Рассмотрим $\eta = (X_tn, ..., X_0)^t$. Ясно,  что $\eta = A\xi$, где

$A = $ ---матр.1---

$\Rightarrow \phi_\eta(\lambda vector) = Ee^{i <\eta, \lambda v>} = $
записи на бумажке
 
Теперь забудем про то, что процесс существует. 
$\forall n \forall t_1 < ... t_n$ зададим $\phi_t1..tn$ по формуле $(*)$.
% 
% Проверим условия следствия: пусть $l_m = 0$.
% ---опять записи на бумажке---
% 
% Смотрим на две подчеркнутые штуки.
% 
% По условию $\phi_tm-1 tm(\tau) \phi_tm tm+1(\tau) = \phi_tm-1 tm+1(\tau)$.
% 
% --- тут то, что после равно в кружочке --
% 
% 
% Согласно следствию, $\exists (X_t, t \geq 0)$ т.ч.
% $\phi_tn .. t1$~-- х.ф. $(X_tn .. X_t1)$. По построению $\phi_t1..tn$
% это означает, что $X_t$ имеет независимые приращения, $X_t - X_s \overset{d}{=}
% Q_{s,t}, t > s$, и $X_0 \overset{d}{=} Q_0$.
\end{proof}

% 
% \begin{remark}
% \begin{itemize}
% \item Процесс с независимыми приращениями с дискретным временем~-- это случайное блуждание.ы
% \item Если $\phi_{\xi+\eta}(t) = \phi_\xi(t) \phi_\eta(t)$, то это не значит,
% что $\xi$ и $\eta$ независимы.
% \end{itemize}
% \end{remark}
% 
% \subsection{Пуассоновский процесс}
% \begin{definition}
% Процесс $N_t, t \geq 0$ называется \emph{пуассоновским процессом интенсовности $\lambda > 0$},
% если
% \begin{enumerate}
% \item $N_0 = 0$ п.н.
% \item $N_t$ имеет независимые приращения
% \item $N_t - N_s \tilde Pois(\lambda(t-s)), t > s \geq 0$
% \end{enumerate}
% \end{definition}
% 
% \begin{statement}
% Пуассоновский процесс существует.
% \end{statement}
% \begin{proof}
% Пусть $\phi_s,t$~-- х.ф. $Pois(\lambda(t-s))$. Тогда
% $$\phi_{s,t}(\tau) = ... = e^{\lambda(t-s)(e^{i\tau} - 1)}$$
% 
% Отсюда видно, что $\forall s < u < t$
% $$\phi_{s,u}(\tau) \phi_{u,t}(\tau) = \phi_{s,t}(\tau)$$
% Значит, по критерию существования пуассоновский процесс существует.
% \end{proof}
% 
% \paragraph{Свойства траекторий}
% \begin{enumerate}
% \item Целочисленные: $N_t \in \myZ_+, t \geq 0$
% \item Неубывающие: $N_t - N_s \geq 0, t > s$
% \end{enumerate}
% Вопросы:
% \begin{enumerate}
% \item размер скачков?
% \item распределение моментов скачков
% \end{enumerate}
% 
% \begin{theorem}[явная конструкция пуассоновского процесса]
% Пусть $\{\xi_n, n \in \myN\}$~-- нез. $\tilde Exp(\lambda)$, $S_n =  \xi_1 + \dots + \xi_n$.
% Тогда процесс восстановления
% $$X_t = \sup\{n: S_n \leq t\}$$
% является пуассоновским процессом интенсивности $\lambda$.
% \end{theorem}
% \begin{proof}
% Пусть $X_0 = 0$.
% Рассмотрим вектор $(S_n, \dots, S_1)$. Он имеет плотность:
% $$P_{S_n..S_1}(x_1..x_1) = P_{\xi_n .. \xi_1}(X_n-x_{n-1}, ..., X_2-X_1, X_1) =
% |\xi_i \tilde Exp(\lambda)| = \prod_{k=1}^{n} \lambda e^{-(x_k-x_k-1)\lambda} \ind{X_k > X_k-1} =
% \lambda^n e^{-\lambda X_n} \ind{x_n > \dots > x_1 > 0}$$
% 
% Зафиксируем $0 < t_1 < ... < t_n$ и $k_n \geq ... \geq k_1, k_i \in \myZ_+$. Рассмотрим
% $$P(X_tn - X_tn-1 = k_n-k_n-1, ..., X_t1 = k1)$$
% 
% Что такое X_tn - X_tn-1? Это количество скачков, которые совершает процесс на отрезке $tn-1, tn$.
% 
% $$ = P(S_k1 < t1, \{S_k1+1, ... S_k2\} \in (t1, t2], ..., \{S_{k_{n-1}+1}, .., S_kn\}
% \in (t_n-1, tn], S_{k_n+1} > t_n) =$$
% |подставляем плотность $p_{s1..s_{k_n + 1}} =$|
% интегрируем в области, написанной на бумажке.
% 
% дальше много всего на бумажке
% 
% Следовательно, $X_t$ имеет независимые приращения, и эти приращения пуассоновские:
% 
% $$X_{t_j} - X_{t_{j-1}} \tilde Pois(\lambda(t_j - t_{j-1}))$$
% 
% Значит, $X_t$~-- это пуассоновский процесс.
% \end{proof}
% \begin{corollary}
% Пусть $N_t$~-- пуассоновский процесс интенсивности $\lambda > 0$, $Y_n$~-- 
% момент $n$-того скачка. Тогда
% \begin{enumerate}
% \item П.н. все скачки $N_t$ имеют размер $1$
% \item $Y_n \tilde \Gamma(\lambda, n)$, записать плотность
% \item $Y_n-Y_{n-1}, ..., Y_1$~-- нез. $Exp(\lambda)$.
% \end{enumerate}
% \begin{proof}
% \begin{enumerate}
% \item Пусть $X_t$~-- явная конструкция из теоремы.
% 
% $P($У $N_t$ есть скачок размера $\geq 2) = P($ у $X_t$ есть скачок размера $\geq 2) =
% P(\exists n: S_n = S_{n+1}) = P(\exists n; \xi_n = 0) = 0$, т.к. $\xi_n$ абсолютно непрерывны.
% 
% \item 2 и 3 выполнены для $X_t$ по построению $\Rightarrow$ верны и для $N_t$.
% \end{enumerate}
% \end{proof}
% \end{corollary}
% \end{document}


\ifdefined\Main\else
\end{document}
\fi
