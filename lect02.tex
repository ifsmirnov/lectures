\section*{Общие определения}

\begin{definition}
Пусть $(\Omega, F, P)$~-- вероятностное пространство, а $(E, \mathcal{E})$~-- измеримое пространство.
Отображение $\xi: \Omega \rightarrow E$ называется \emph{случайным элементом}, если оно измеримо,
т.е. $$\forall B \in \mathcal{E}\ \xi^{-1}(B) = \{\omega: \xi(\omega) \in B\} \in F$$
\end{definition}

\begin{definition}
Пусть $T$~-- некоторое множество и на $(\Omega, F,P)$ для $\forall t \in T$ задан случайный элемент $X_t$. Тогда набор $X = {X_t, t \in T}$ называется \emph{случайной функцией} на множестве $T$.
\end{definition}

\begin{remark}
Вообще говоря, не предполагается, что все $X_t$ принимают значения в одном и том же пространстве.
\end{remark}

\begin{definition}
Пусть $X = (X(t, \omega))$. При фиксированном $\omega = \omega_0$ функция
$$\tilde{X}_{\omega_0}(t) = X_t(\omega) \vert _{\omega = \omega_0}$$
на $T$ называется \emph{траекторией} или \emph{реализацией} случайной функции $X = \{X_t : t \in T\}$.
\end{definition}

\subsection{Терминология}

\begin{itemize}
\item Если $T \subset \mathbb{R}$, то случайная функция называется случайным процессом.
\item Если $T = [a, b], (a, b), [a, +\infty)$ и т.д., то процесс $X$ называется процессом с непрерывным временем.
\item Если $T = \mathbb{N}, \mathbb{Z}$ и т. д. то процесс $X$ называется процессом с дискретным временем.
\item Если $T \subset \mathbb{R}^d$, то процесс $X$ называется случайным полем.
\end{itemize}

\begin{remark}
Далее всюду будем использовать термин <<случайный процесс>>.
\end{remark}

\subsection{Примеры}
\begin{enumerate}
\item $X_t(\omega) = \xi(\omega) \cdot f(t)$, где $\xi(\omega)$~-- с.в., $f(t)$~-- детерминированная функция.
\item Пусть $\{\xi_n, n\in\mathbb{N}\}$~-- независимые случайные векторы, $S_n = \xi_1 + \dots + \xi_n$,
$S_0 = 0$. Тогда процесс с дискретным временем $\{S_n, n \in \mathbb{Z}_+\}$
называется случайным блужданием.

Траектория: см. рис. 1
\item Пусть $\{\xi_n : n \in \mathbb{N}\}$~-- норсв, $\xi_n >= 0$, $\xi_n \neq const$ п.н.,
$S_n = \xi_1 + \dots + \xi_n$, $S_0 = 0$. Тогда процесс
$$X_t = \sup\{n: S_n <= t\}, t >= 0$$
называется процессом восстановления.

Траектория: см. рис. 2

\begin{statement}
Процесс восстановления конечен почти наверное.
\end{statement}
\begin{proof}
Пусть сначала $E\xi_i = a > 0$.

Заметим, что $\{S_k \leq t\} \supset \{S_{k+1} \leq t\}$.

$\{X_t=+\infty\} = \{\sup\{n:S_n\leq t\} = +\infty\} =
\{\forall n: S_n \leq t\} = \mybigcap_n\{S_n \leq t\} \Rightarrow$
|по непрерывности вероятностной меры|
$\Rightarrow P(X_t = +\infty) = P(\forall n S_n \leq t) =
\mylim_{n \to \infty} P(S_n \leq t)
= \mylim_{n \to \infty} P(\frac{S_n}{n} \leq \frac{t}{n}) \leq$
|для больших n|
$\leq \mylim_{n \to \infty} P(\frac{S_n}{n} < \frac{a}{2})$.

Но по УЗБЧ $S_n \to a \Rightarrow P(S_n \leq a/2) \to P(a \leq a/2) = 0$.

В силу того, что $X_t\uparrow$ при $t\uparrow$,
$P(\exists t: X_t = +\infty) = P(\exists n: x_n = +\infty) \leq \mysum_{n}{P(x_n = \infty)} = 0$.

Если $E\xi_i = +\infty$, то случай сводится к предыдущему:
$\exists C > 0: \tilde{\xi_n} = \min(\xi_n, C), E\tilde{\xi_n} > 0$.
Тогда $P(X_t = +\infty) = P(\forall n S_n \leq t) \leq P(\tilde{S_n} \leq t) = 0$
(доказали в случае конечного матожидания).
\end{proof}

Откуда может возникнуть процесс восстановления?
Физическая модель~-- <<Модель перегорания лампочки>>. $\xi_n$~-- случайная величина, равная времени работы лампочки, $X_t$~-- сколько раз пришлось заменить лампочку в к моменту времени $t$.

\item Модель страхования Крамера-Лундберга

Пусть есть $\{\xi_n, n \in \myN\}, \{\eta_m, m \in \myN\},
\xi_n \overset{d}{=} \xi_m,
\eta_n \overset{d}{=} \eta_m,
\{\xi_n, \eta_m\}$ независимы,
$\xi_n, \eta_n \geq 0, \xi_n$ невырождены.

Пусть $\{X_t, t \geq 0\}$~-- процесс восстановления, построенный по случайным величинам
$\{\xi_n, n \in \myN\}, y_0, c > 0$. Тогда
$Y_t = y_0 + c \cdot t - \mysum_{k=1}^{X_t} \eta_k$~--
модель страхования Крамера-Лундберга.

\paragraph*{Смысл параметров}
\begin{itemize}
\item $y_0$~-- начальный капитал
\item $c$~-- скорость поступления страховых взносов
\item $S_k = \xi_1 + \dots + \xi_k$~-- время $k$-й выплаты, $\eta_k$~-- размер этой выплаты
\item $X_t$~-- число выплат к моменту времени $t > 0$
\item $\mysum_{k=1}^{X_t} \eta_k$~-- общий размер выплат к этому моменту времени
\item $Y_t$~-- текущий капитал компании
\end{itemize}
\end{enumerate}