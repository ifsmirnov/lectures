% Title

\ifdefined\HeaderIncluded
\else

\newcommand{\HeaderIncluded}{}

\documentclass[a4paper, 10pt]{article}
\usepackage[utf8]{inputenc}
\usepackage[english, russian]{babel}
\usepackage{amsmath, amsfonts, amssymb, amsthm}
\usepackage{tikz} % some graphics
%\usepackage[indentheadings]{russcorr}
\usepackage{geometry}
\usetikzlibrary{arrows}
\setlength{\parindent}{0pt}


\geometry{left=2cm}
\geometry{right=2cm}
\geometry{top=2cm}
\geometry{bottom=2cm}

\hfuzz=18pt

% Commands

\newcommand*{\hm}[1]{#1\nobreak\discretionary{}%
{\hbox{$\mathsurround=0pt #1$}}{}} % a\hm=b makes "=" carriable to the next line with duplication of the sign

\newcommand{\combus}[2]{\left(\begin{array}{c}#1 \\ #2 \end{array} \right)} % american style for C_n^k
\newcommand{\combru}[2]{C_{#1}^{#2}} % russian C_n^k
\newcommand{\comb}[2]{\combru{#1}{#2}}
\newcommand{\myN}{\mathbb{N}} % nice letters for common number sets
\newcommand{\myZ}{\mathbb{Z}}
\newcommand{\myR}{\mathbb{R}}
\newcommand{\myC}{\mathbb{C}}
\newcommand{\myQ}{\mathbb{Q}}
\newcommand{\myE}{\mathcal{E}} % basis
\newcommand{\myF}{\mathcal{F}} % basis
\newcommand{\myB}{\mathcal{B}} % set system (random processes)
\newcommand{\myM}{\mathcal{M}} % set system (random processes)
\newcommand{\mysetP}{\mathcal{P}}
\newcommand{\mysetR}{\mathcal{R}}
\newcommand{\mysetM}{\mathcal{M}}
\newcommand{\mysetN}{\mathcal{N}}
\newcommand{\mysetX}{\mathcal{X}}
\newcommand{\walls}[1]{\left | #1 \right |} % |smth_vertically_large|
\newcommand{\pars}[1]{\left( #1 \right)} % (smth_vertically_large)
\newcommand{\class}[1]{[ #1 ]} % [smth_vertically_large]
\newcommand{\bra}[1]{\langle #1 \rangle} % brackets for span of vectors, eg. <e_1, ..., e_k>
\newcommand{\myset}[1]{\left\{ #1 \right\}} % because { and } are special symbols in TeX
\newcommand{\mysetso}[2]{\myset{#1 \mid #2}}
\newcommand{\myforall}[1]{\forall #1:}
\newcommand{\myexists}[1]{\exists #1:\ }
\newcommand{\forcenewline}{\ \newline}
\newcommand{\reversein}{\ni}
\renewcommand{\phi}{\varphi}
\newcommand{\eps}{\varepsilon}
\newcommand{\ind}[1]{\mathrm{I}\{ #1 \}}
\newcommand{\myliminf}[1]{\underset{ #1 }{\underline{\lim}}}
\newcommand{\mylimsup}[1]{\underset{ #1 }{\overline{\lim}}}
\newcommand{\seq}[3]{#1_{#2}, \dots, #1_{#3}}

%
% Example text:
% M = {x^2 | x is prime}
%
% Corresponding markup:
% $$ \mysetM = \mysetso{x^2}{x \text{ is prime}} $$
%
\renewcommand{\leq}{\leqslant}
\renewcommand{\geq}{\geqslant}
\newcommand{\myempty}{\varnothing}
\renewcommand{\emptyset}{\varnothing}
\newcommand{\myand}{\;\; \hm\& \;\;}
\newcommand{\myor}{\; \hm\vee \;}
\newcommand{\conj}[1]{\overline{#1}} % complex conjugation
\newcommand{\mycirc}{\circ}
\newcommand{\poly}[2]{#1 [ #2 ]} % ring of polynomials
\newcommand{\Rx}{\poly{\myR}{x}}
\newcommand{\Cx}{\poly{\myC}{x}}
\newcommand{\mydim}[1]{\dim #1}
\newcommand{\mycodim}[1]{\text{codim} #1}
\newcommand{\coords}[2]{\pars{#1}_{#2}}
\newcommand{\myrank}[1]{\text{rank} #1}
\newcommand{\mysegment}[2]{[#1, #2]}
\newcommand{\myinterval}[2]{(#1, #2)}
\newcommand{\mypair}[2]{(#1, #2)}
\newcommand{\myfunc}[3]{#1\!:\,#2 \hm\to #3} % TODO: make spaces between elements of this tag look better
\newcommand{\suchthat}{\!:\,}
\newcommand{\mydef}[1]{\emph{#1}}

% Arrows
\newcommand{\myright}{\;\hm\Rightarrow\;}
\newcommand{\myleft}{\;\hm\Leftarrow\;}
\newcommand{\myleftright}{\;\hm\Leftrightarrow\;}
% \newcommand{\vect}[1]{\overrightarrow{\vphantom{b}#1}}
\newcommand{\vect}[1]{\Vec{\vphantom{b}#1}}

% Sums, prods and other things with \limits
\newcommand{\mysum}{\sum\limits}
\newcommand{\myprod}{\prod\limits}
\newcommand{\mylim}{\lim\limits}
\newcommand{\mysup}{\sup\limits}
\newcommand{\myinf}{\inf\limits}
\newcommand{\mybigcup}{\bigcup\limits}
\newcommand{\mybigcap}{\bigcap\limits}
\newcommand{\mybigor}{\bigvee\limits}
\newcommand{\mybigand}{\bigwedge\limits}
\newcommand{\mybigxor}{\bigoplus\limits}
\newcommand{\xor}{\oplus}
\newcommand{\eq}{\equiv}

\newcommand{\smatrix}[1]{\begin{smallmatrix}#1\end{smallmatrix}}
\newcommand{\psmatrix}[1]{\pars{\begin{smallmatrix}#1\end{smallmatrix}}}
\newcommand{\wsmatrix}[1]{\walls{\begin{smallmatrix}#1\end{smallmatrix}}}


\newcommand{\abs}[1]{|#1|}
% Operators
\DeclareMathOperator{\pr}{pr} % Проекция
\DeclareMathOperator{\Tr}{Tr} % След
\DeclareMathOperator{\mychar}{char} % Характеристика поля
\DeclareMathOperator{\id}{id} % Тождественное преобразование
\DeclareMathOperator{\rk}{rk} % Ранг
\DeclareMathOperator{\Kernel}{Ker}
\DeclareMathOperator{\Image}{Im}
\DeclareMathOperator{\sign}{sign} % Знак
\DeclareMathOperator{\sgn}{sign} % Знак
\DeclareMathOperator{\const}{const}

% Special theorems
\newtheorem*{theorem-menelaus}{Теорема Менелая}

% Theorems
\newtheorem*{theorem-star}{Теорема}
\newtheorem{theorem}{Теорема}[section]
\newtheorem*{theorem-definition-star}{Теорема-определение}
\newtheorem*{corollary-star}{Следствие}
\newtheorem{corollary}{Следствие}[section]
\newtheorem*{property-star}{Свойство}
\newtheorem*{properties}{Свойства}
\newtheorem{property}{Свойство}
\newtheorem*{lem-star}{Лемма}
\newtheorem{lem}{Лемма}[section]
\newtheorem*{proposition-star}{Предложение}
\newtheorem{proposition}{Предложение}
\newtheorem{stage}{Этап}
\newtheorem*{statement}{Утверждение}
\newtheorem*{designation}{Обозначение}
\newtheorem*{usage}{Использование}

\theoremstyle{remark}
\newtheorem*{remark}{Замечание}

\theoremstyle{definition}
\newtheorem{problem}{Задача}
\newtheorem{exercise}{Упражнение}

\theoremstyle{definition}
\newtheorem*{definition-star}{Определение}
\newtheorem{definition}{Определение}[section]

\theoremstyle{definition}
\newtheorem*{example-star}{Пример}
\newtheorem{example}{Пример}
\newtheorem*{examples}{Примеры}
\newtheorem{algorithm}{Алгоритм}
\newtheorem{case}{Случай}
\newtheorem*{case-star}{Случай}

\newcommand{\rightoffset}[1]{
\begin{flushright} \begin{minipage}{0.9\textwidth}
\begin{tabular}{|l} \begin{minipage}{\textwidth}
#1
\end{minipage}
\end{tabular}
\end{minipage}
\end{flushright}
}

\newcommand{\longformula}[1]{
$$\begin{array}{l}
#1
\end{array}$$
}


% Style
\newcommand{\tocstyle}{\setlength{\parindent}{0ex} \setlength{\parskip}{0ex}}
\newcommand{\mainstyle}{\setlength{\parindent}{0ex} \setlength{\parskip}{1ex}}

\mainstyle
\setcounter{secnumdepth}{2}
\renewcommand{\thesubsection}{\arabic{subsection}}


% Titles of lectures
\newcommand{\mylecture}[1]{\setcounter{secnumdepth}{-1} \section{#1} \setcounter{secnumdepth}{2} \setcounter{subsection}{0} \setcounter{corollary}{0} \setcounter{definition}{0} \setcounter{theorem}{0}}

\fi


\ifdefined\Main\else
\begin{document}
\fi

\section*{Общие определения}
\begin{definition}
Пусть $(\Omega, F, P)$~-- вероятностное пространство, а $(E, \mathcal{E})$~-- измеримое пространство.
Отображение $\xi: \Omega \rightarrow E$ называется \emph{случайным элементом}, если оно измеримо,
т.е. $$\forall B \in \mathcal{E}\ \xi^{-1}(B) = \{\omega: \xi(\omega) \in B\} \in F$$
\end{definition}
\begin{definition}
Пусть $T$~-- некоторое множество и на $(\Omega, F,P)$ для $\forall t\in T$ задан случайный элемент $X_t$. Тогда набор $X = {X_t, t \in T}$ называется \emph{случайной функцией} на множестве $T$. 
\end{definition}

\begin{remark}
Вообще говоря, не предполагается, что все $X_t$ принимают значения в одном и том же пространстве.
\end{remark}

\begin{definition}
Пусть $X = (X(t, \omega))$. При фиксированном $\omega = \omega_0$ функция
$$\tilde{X}_{\omega_0}(t) = X_t(\omega) \vert _{\omega=\omega_0}$$
на $T$ называется \emph{траекторией} (или реализацией) случайной функции $X = {X_t, t\in T}$. 
\end{definition}

\subsection{Терминология}
\begin{itemize}
\item Если $T \subset \mathbb{R}$, то случайная функция называется случайным процессом.
\item Если $T = [a, b], (a, b), [a, +\infty)$ и т.д., то процесс $X$ называется процессом с непрерывным временем.
\item Если $T = \mathbb{N}, \mathbb{Z}$ и т. д. то процесс $X$ называется процессом с дискретным временем.
\item Если $T \subset \mathbb{R}^d$, то процесс $X$ называется случайным полем.
\end{itemize}

\begin{remark}
Далее всюду будем использовать термин <<случайный процесс>>.
\end{remark}

\subsection{Примеры}
\begin{enumerate}
\item $X_t(\omega) = \xi(\omega) \cdot f(t)$, где $\xi(\omega)$~-- с.в., $f(t)$~-- детерминированная функция.
\item Пусть $\{\xi_n, n\in\mathbb{N}\}$~-- независимые случайные векторы, $S_n = \xi_1 + \dots + \xi_n$,
$S_0 = 0$. Тогда процесс с дискретным временем $\{S_n, n \in \mathbb{Z}_+\}$
называется случайным блужданием.

Траектория: см. рис. 1
\item Пусть $\{\xi_n, n \in \mathbb{N}\}$~-- норсв, $\xi_n >= 0$, $\xi_n \neq const$ п.н.,
$S_n = \xi_1 + \dots + \xi_n$, $S_0 = 0$. Тогда процесс
$$X_t = \sup\{n: S_n <= t\}, t >= 0$$
называется процессом восстановления.

Траектория: см. рис. 2

\begin{statement}
Процесс восстановления конечен почти наверное.
\end{statement}
\begin{proof}
Пусть сначала $E\xi_i = a > 0$.

Заметим, что $\{S_k \leq t\} \supset \{S_{k+1} \leq t\}$.

$\{X_t=+\infty\} = \{\sup\{n:S_n\leq t\} = +\infty\} =
\{\forall n: S_n \leq t\} = \mybigcap_n\{S_n \leq t\} \Rightarrow$
|по непрерывности вероятностной меры|
$\Rightarrow P(X_t = +\infty) = P(\forall n S_n \leq t) =
\mylim_{n \to \infty} P(S_n \leq t)
= \mylim_{n \to \infty} P(\frac{S_n}{n} \leq \frac{t}{n}) \leq$
|для больших n|
$\leq \mylim_{n \to \infty} P(\frac{S_n}{n} < \frac{a}{2})$.

Но по УЗБЧ $S_n \to a \Rightarrow P(S_n \leq a/2) \to P(a \leq a/2) = 0$.

В силу того, что $X_t\uparrow$ при $t\uparrow$,
$P(\exists t: X_t = +\infty) = P(\exists n: x_n = +\infty) \leq \mysum_{n}{P(x_n = \infty)} = 0$. 

Если $E\xi_i = +\infty$, то случай сводится к предыдущему:
$\exists C > 0: \tilde{\xi_n} = \min(\xi_n, C), E\tilde{\xi_n} > 0$.
Тогда $P(X_t = +\infty) = P(\forall n S_n \leq t) \leq P(\tilde{S_n} \leq t) = 0$
(доказали в случае конечного матожидания).
\end{proof}

Откуда может возникнуть процесс восстановления?
Физическая модель~-- <<Модель перегорания лампочки>>. $\xi_n$~-- случайная величина, равная времени работы лампочки, $X_t$~-- сколько раз пришлось заменить лампочку в к моменту времени $t$.

\item Модель страхования Крамера-Лундберга

Пусть есть $\{\xi_n, n \in \myN\}, \{\eta_m, m \in \myN\},
\xi_n \overset{d}{=} \xi_m,
\eta_n \overset{d}{=} \eta_m,
\{\xi_n, \eta_m\}$ независимы,
$\xi_n, \eta_n \geq 0, \xi_n$ невырождены.

Пусть $\{X_t, t \geq 0\}$~-- процесс восстановления, построенный по случайным величинам
$\{\xi_n, n \in \myN\}, y_0, c > 0$. Тогда
$Y_t = y_0 + c \cdot t - \mysum_{k=1}^{X_t} \eta_k$~--
модель страхования Крамера-Лундберга.

\paragraph*{Смысл параметров}
\begin{itemize}
\item $y_0$~-- начальный капитал
\item $c$~-- скорость поступления страховых взносов
\item $S_k = \xi_1 + \dots + \xi_k$~-- время $k$-й выплаты, $\eta_k$~-- размер этой выплаты
\item $X_t$~-- число выплат к моменту времени $t > 0$
\item $\mysum_{k=1}^{X_t} \eta_k$~-- общий размер выплат к этому моменту времени
\item $Y_t$~-- текущий капитал компании
\end{itemize}
\end{enumerate}

\ifdefined\Main\else
\end{document}
\fi
