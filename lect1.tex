% Title

\ifdefined\HeaderIncluded
\else

\newcommand{\HeaderIncluded}{}

\documentclass[a4paper, 10pt]{article}
\usepackage[utf8]{inputenc}
\usepackage[english, russian]{babel}
\usepackage{amsmath, amsfonts, amssymb, amsthm}
\usepackage{tikz} % some graphics
%\usepackage[indentheadings]{russcorr}
\usepackage{geometry}
\usetikzlibrary{arrows}
\setlength{\parindent}{0pt}


\geometry{left=2cm}
\geometry{right=2cm}
\geometry{top=2cm}
\geometry{bottom=2cm}

\hfuzz=18pt

% Commands

\newcommand*{\hm}[1]{#1\nobreak\discretionary{}%
{\hbox{$\mathsurround=0pt #1$}}{}} % a\hm=b makes "=" carriable to the next line with duplication of the sign

\newcommand{\combus}[2]{\left(\begin{array}{c}#1 \\ #2 \end{array} \right)} % american style for C_n^k
\newcommand{\combru}[2]{C_{#1}^{#2}} % russian C_n^k
\newcommand{\comb}[2]{\combru{#1}{#2}}
\newcommand{\myN}{\mathbb{N}} % nice letters for common number sets
\newcommand{\myZ}{\mathbb{Z}}
\newcommand{\myR}{\mathbb{R}}
\newcommand{\myC}{\mathbb{C}}
\newcommand{\myQ}{\mathbb{Q}}
\newcommand{\myE}{\mathcal{E}} % basis
\newcommand{\myF}{\mathcal{F}} % basis
\newcommand{\myB}{\mathcal{B}} % set system (random processes)
\newcommand{\myM}{\mathcal{M}} % set system (random processes)
\newcommand{\mysetP}{\mathcal{P}}
\newcommand{\mysetR}{\mathcal{R}}
\newcommand{\mysetM}{\mathcal{M}}
\newcommand{\mysetN}{\mathcal{N}}
\newcommand{\mysetX}{\mathcal{X}}
\newcommand{\walls}[1]{\left | #1 \right |} % |smth_vertically_large|
\newcommand{\pars}[1]{\left( #1 \right)} % (smth_vertically_large)
\newcommand{\class}[1]{[ #1 ]} % [smth_vertically_large]
\newcommand{\bra}[1]{\langle #1 \rangle} % brackets for span of vectors, eg. <e_1, ..., e_k>
\newcommand{\myset}[1]{\left\{ #1 \right\}} % because { and } are special symbols in TeX
\newcommand{\mysetso}[2]{\myset{#1 \mid #2}}
\newcommand{\myforall}[1]{\forall #1:}
\newcommand{\myexists}[1]{\exists #1:\ }
\newcommand{\forcenewline}{\ \newline}
\newcommand{\reversein}{\ni}
\renewcommand{\phi}{\varphi}
\newcommand{\eps}{\varepsilon}
\newcommand{\ind}[1]{\mathrm{I}\{ #1 \}}
\newcommand{\myliminf}[1]{\underset{ #1 }{\underline{\lim}}}
\newcommand{\mylimsup}[1]{\underset{ #1 }{\overline{\lim}}}
\newcommand{\seq}[3]{#1_{#2}, \dots, #1_{#3}}

%
% Example text:
% M = {x^2 | x is prime}
%
% Corresponding markup:
% $$ \mysetM = \mysetso{x^2}{x \text{ is prime}} $$
%
\renewcommand{\leq}{\leqslant}
\renewcommand{\geq}{\geqslant}
\newcommand{\myempty}{\varnothing}
\renewcommand{\emptyset}{\varnothing}
\newcommand{\myand}{\;\; \hm\& \;\;}
\newcommand{\myor}{\; \hm\vee \;}
\newcommand{\conj}[1]{\overline{#1}} % complex conjugation
\newcommand{\mycirc}{\circ}
\newcommand{\poly}[2]{#1 [ #2 ]} % ring of polynomials
\newcommand{\Rx}{\poly{\myR}{x}}
\newcommand{\Cx}{\poly{\myC}{x}}
\newcommand{\mydim}[1]{\dim #1}
\newcommand{\mycodim}[1]{\text{codim} #1}
\newcommand{\coords}[2]{\pars{#1}_{#2}}
\newcommand{\myrank}[1]{\text{rank} #1}
\newcommand{\mysegment}[2]{[#1, #2]}
\newcommand{\myinterval}[2]{(#1, #2)}
\newcommand{\mypair}[2]{(#1, #2)}
\newcommand{\myfunc}[3]{#1\!:\,#2 \hm\to #3} % TODO: make spaces between elements of this tag look better
\newcommand{\suchthat}{\!:\,}
\newcommand{\mydef}[1]{\emph{#1}}

% Arrows
\newcommand{\myright}{\;\hm\Rightarrow\;}
\newcommand{\myleft}{\;\hm\Leftarrow\;}
\newcommand{\myleftright}{\;\hm\Leftrightarrow\;}
% \newcommand{\vect}[1]{\overrightarrow{\vphantom{b}#1}}
\newcommand{\vect}[1]{\Vec{\vphantom{b}#1}}

% Sums, prods and other things with \limits
\newcommand{\mysum}{\sum\limits}
\newcommand{\myprod}{\prod\limits}
\newcommand{\mylim}{\lim\limits}
\newcommand{\mysup}{\sup\limits}
\newcommand{\myinf}{\inf\limits}
\newcommand{\mybigcup}{\bigcup\limits}
\newcommand{\mybigcap}{\bigcap\limits}
\newcommand{\mybigor}{\bigvee\limits}
\newcommand{\mybigand}{\bigwedge\limits}
\newcommand{\mybigxor}{\bigoplus\limits}
\newcommand{\xor}{\oplus}
\newcommand{\eq}{\equiv}

\newcommand{\smatrix}[1]{\begin{smallmatrix}#1\end{smallmatrix}}
\newcommand{\psmatrix}[1]{\pars{\begin{smallmatrix}#1\end{smallmatrix}}}
\newcommand{\wsmatrix}[1]{\walls{\begin{smallmatrix}#1\end{smallmatrix}}}


\newcommand{\abs}[1]{|#1|}
% Operators
\DeclareMathOperator{\pr}{pr} % Проекция
\DeclareMathOperator{\Tr}{Tr} % След
\DeclareMathOperator{\mychar}{char} % Характеристика поля
\DeclareMathOperator{\id}{id} % Тождественное преобразование
\DeclareMathOperator{\rk}{rk} % Ранг
\DeclareMathOperator{\Kernel}{Ker}
\DeclareMathOperator{\Image}{Im}
\DeclareMathOperator{\sign}{sign} % Знак
\DeclareMathOperator{\sgn}{sign} % Знак
\DeclareMathOperator{\const}{const}

% Special theorems
\newtheorem*{theorem-menelaus}{Теорема Менелая}

% Theorems
\newtheorem*{theorem-star}{Теорема}
\newtheorem{theorem}{Теорема}[section]
\newtheorem*{theorem-definition-star}{Теорема-определение}
\newtheorem*{corollary-star}{Следствие}
\newtheorem{corollary}{Следствие}[section]
\newtheorem*{property-star}{Свойство}
\newtheorem*{properties}{Свойства}
\newtheorem{property}{Свойство}
\newtheorem*{lem-star}{Лемма}
\newtheorem{lem}{Лемма}[section]
\newtheorem*{proposition-star}{Предложение}
\newtheorem{proposition}{Предложение}
\newtheorem{stage}{Этап}
\newtheorem*{statement}{Утверждение}
\newtheorem*{designation}{Обозначение}
\newtheorem*{usage}{Использование}

\theoremstyle{remark}
\newtheorem*{remark}{Замечание}

\theoremstyle{definition}
\newtheorem{problem}{Задача}
\newtheorem{exercise}{Упражнение}

\theoremstyle{definition}
\newtheorem*{definition-star}{Определение}
\newtheorem{definition}{Определение}[section]

\theoremstyle{definition}
\newtheorem*{example-star}{Пример}
\newtheorem{example}{Пример}
\newtheorem*{examples}{Примеры}
\newtheorem{algorithm}{Алгоритм}
\newtheorem{case}{Случай}
\newtheorem*{case-star}{Случай}

\newcommand{\rightoffset}[1]{
\begin{flushright} \begin{minipage}{0.9\textwidth}
\begin{tabular}{|l} \begin{minipage}{\textwidth}
#1
\end{minipage}
\end{tabular}
\end{minipage}
\end{flushright}
}

\newcommand{\longformula}[1]{
$$\begin{array}{l}
#1
\end{array}$$
}


% Style
\newcommand{\tocstyle}{\setlength{\parindent}{0ex} \setlength{\parskip}{0ex}}
\newcommand{\mainstyle}{\setlength{\parindent}{0ex} \setlength{\parskip}{1ex}}

\mainstyle
\setcounter{secnumdepth}{2}
\renewcommand{\thesubsection}{\arabic{subsection}}


% Titles of lectures
\newcommand{\mylecture}[1]{\setcounter{secnumdepth}{-1} \section{#1} \setcounter{secnumdepth}{2} \setcounter{subsection}{0} \setcounter{corollary}{0} \setcounter{definition}{0} \setcounter{theorem}{0}}

\fi


\title{Случайные процессы}

\begin{document}

\maketitle
\tableofcontents

\newpage

\part*{Введение. Историческая справка}
\underline{Теория вероятностей}: математический анализ случайных явлений.\\
\underline{Теория случайных процессов}: стохастические модели и фактор времени.

\paragraph{Предпосылки к изучению}
\begin{itemize}
\item 1827, Р. Броун~-- броуновское движение частиц в воде~$\Rightarrow$ процесс броуновского движения
\item 1903, Л. Башелье~-- колебания курсов бумаг на бирже~$\Rightarrow$ процесс броуновского движения
\item 1906, А.А. Марков~-- анализ комбинаций гласных и согласных в романе <<Евгений Онегин>>~$\Rightarrow$ марковские цепи
\item 1903, Ф. Лундберг~-- модель деятельности страховой компании~$\Rightarrow$ пуассоновский процесс
\item 1873, Ф. Гальтон, Г. Ватсон~-- анализ вымирания аристократических фамилий в Великобритании~$\Rightarrow$ ветвящиеся процессы
\item Начало XX века, А. Эрланг~-- изучение загрузки телефонных сетей~$\Rightarrow$ теория массового обслуживания
\end{itemize}

\paragraph{Применения}
\begin{itemize}
\item Физика (стохастическое исчисление, теория гиббсовских полей)
\item Экономика (финансовая математика)
\item Биология
\end{itemize}

\part{Случайные процессы}
\section*{Общие определения}
\begin{definition}
Пусть $(\Omega, F, P)$~-- вероятностное пространство, а $(E, \mathcal{E})$~-- измеримое пространство.
Отображение $\xi: \Omega \rightarrow E$ называется \emph{случайным элементом}, если оно измеримо,
т.е. $$\forall B \in \mathcal{E}\ \xi^{-1}(B) = \{\omega: \xi(\omega) \in B\} \in F$$
\end{definition}
\begin{definition}
Пусть $T$~-- некоторое множество и на $(\Omega, F,P)$ для $\forall t\in T$ задан случайный элемент $X_t$. Тогда набор $X = {X_t, t \in T}$ называется \emph{случайной функцией} на множестве $T$. 
\end{definition}

\begin{remark}
Вообще говоря, не предполагается, что все $X_t$ принимают значения в одном и том же пространстве.
\end{remark}

\begin{definition}
Пусть $X = (X(t, \omega))$. При фиксированном $\omega = \omega_0$ функция
$$\tilde{X}_{\omega_0}(t) = X_t(\omega) \vert _{\omega=\omega_0}$$
на $T$ называется \emph{траекторией} (или реализацией) случайной функции $X = {X_t, t\in T}$. 
\end{definition}

\subsection{Терминология}
\begin{itemize}
\item Если $T \subset \mathbb{R}$, то случайная функция называется случайным процессом.
\item Если $T = [a, b], (a, b), [a, +\infty)$ и т.д., то процесс $X$ называется процессом с непрерывным временем.
\item Если $T = \mathbb{N}, \mathbb{Z}$ и т. д. то процесс $X$ называется процессом с дискретным временем.
\item Если $T \subset \mathbb{R}^d$, то процесс $X$ называется случайным полем.
\end{itemize}

\begin{remark}
Далее всюду будем использовать термин <<случайный процесс>>.
\end{remark}

\subsection{Примеры}
\begin{enumerate}
\item $X_t(\omega) = \xi(\omega) \cdot f(t)$, где $\xi(\omega)$~-- с.в., $f(t)$~-- детерминированная функция.
\item Пусть $\{\xi_n, n\in\mathbb{N}\}$~-- независимые случайные векторы, $S_n = \xi_1 + \dots + \xi_n$,
$S_0 = 0$. Тогда процесс с дискретным временем $\{S_n, n \in \mathbb{Z}_+\}$
называется случайным блужданием.

Траектория: см. рис. 1
\item Пусть $\{\xi_n, n \in \mathbb{N}\}$~-- норсв, $\xi_n >= 0$, $\xi_n \neq const$ п.н.,
$S_n = \xi_1 + \dots + \xi_n$, $S_0 = 0$. Тогда процесс
$$X_t = \sup\{n: S_n <= t\}, t >= 0$$
называется процессом восстановления.

Траектория: см. рис. 2

\begin{statement}
Процесс восстановления конечен почти наверное.
\end{statement}
\begin{proof}
Пусть сначала $E\xi_i = a > 0$.

Заметим, что $\{S_k \leq t\} \supset \{S_{k+1} \leq t\}$.

$\{X_t=+\infty\} = \{\sup\{n:S_n\leq t\} = +\infty\} =
\{\forall n: S_n \leq t\} = \mybigcap_n\{S_n \leq t\} \Rightarrow$
|по непрерывности вероятностной меры|
$\Rightarrow P(X_t = +\infty) = P(\forall n S_n \leq t) =
\mylim_{n \to \infty} P(S_n \leq t)
= \mylim_{n \to \infty} P(S_n \leq t_n) \leq$ % что такое t_n?
|для больших n|
$\leq \mylim_{n \to \infty} P(\frac{S_n}{n} < \frac{a}{2})$.

Но по УЗБЧ $S_n \to a \Rightarrow P(S_n \leq a/2) \to P(a \leq a/2) = 0$.

В силу того, что $X_t\uparrow$ при $t\uparrow$,
$P(\exists t: X_t = +\infty) = P(\exists b: x_n = +\infty) = 0$. % что это за b и что тут должно быть?

Если $E\xi_i = +\infty$, то случай сводится к предыдущему:
$\exists C > 0: \tilde{\xi_n} = \min(\xi_n, C), E\tilde{\xi_n} > 0$.
Тогда $P(X_t = +\infty) = P(\forall n S_n \leq t) \leq P(\tilde{S_n} \leq t) = 0$
(доказали в случае конечного матожидания).
\end{proof}

Откуда может возникнуть процесс восстановления?
Физическая модель~-- <<Модель перегорания лампочки>>. $\xi_n$~-- случайная величина, равная времени работы лампочки, $X_t$~-- сколько раз пришлось заменить лампочку в к моменту времени $t$.

\item Модель страхования Крамера-Лундберга

Пусть есть $\{\xi_n, n \in \myN\}, \{\eta_m, m \in \myN\},
\xi_n \overset{d}{=} \xi_m,
\eta_n \overset{d}{=} \eta_m,
\{\xi_n, \eta_m\}$ независимы,
$\xi_n, \eta_n \geq 0, \xi_n$ невырождены.

Пусть $\{X_t, t \geq 0\}$~-- процесс восстановления, построенный по случайным величинам
$\{\xi_n, n \in \myN\}, y_0, c > 0$. Тогда
$Y_t = y_0 + c \cdot t - \mysum_{k=1}^{X_t} \eta_k$~--
модель страхования Крамера-Лундберга.

\paragraph*{Смысл параметров}
\begin{itemize}
\item $y_0$~-- начальный капитал
\item $c$~-- скорость поступления страховых взносов
\item $S_k = \xi_1 + \dots + \xi_k$~-- время $k$-й выплаты, $\eta_k$~-- размер этой выплаты
\item $X_t$~-- число выплат к моменту времени $t > 0$
\item $\mysum_{k=1}^{X_t} \eta_k$~-- общий размер выплат к этому моменту времени
\item $Y_t$~-- текущий капитал компании
\end{itemize}
\end{enumerate}

\section{Случайное блуждание на прямой}

\begin{definition}
Пусть ${\xi_n, n\in \myN}$~-- норсв, $P(\xi_n = 1) = p$, $P(\xi_n = -1) = q = 1-p$. Тогда процесс
$(S_n, n \in \myZ_+), S_0 = 0, S_n = \xi_1 + \dots + \xi_n$ называется \emph{простейшим случайным блужанием на прямой}.
Если $p = q = \frac{1}{2}$, то блуждание называется \emph{симметричным}.
\end{definition}

\subsection{Вопросы}
\begin{enumerate}
\item вероятность возвращения в ноль
\item распределение первого момента возвращения в ноль
\item среднее время в нуле
\item геометрия траектории
\end{enumerate}
\begin{remark}
Последние два вопроса~-- только для симметричного случая.
\end{remark}

\subsection{Возвращение в ноль}

$P(\{S_n, n \in \myN\}~\text{вернется в ноль})$~-- ?

$P(s1 \neq 0, s2 \neq 0, \dots, s_{2n-1} \neq 0, s_{2n} = 0)$~-- ?

Вероятность каждой траектории, приводящей в $0$ в момент времени $2n$, одна и та же и равна $(pq)^n$.
Каждую траекторию длины $2n$ можно сопоставить с вектором
$\{\eps_1, \dots, \eps_{2n}\}, \eps_i \in \{-1, 1\}$. 

\begin{definition}
Траектория $(\eps_1, \dots, \eps_{2n})$ длины $2n$ называется
\emph{положительной}, если
$\forall k < 2n~\mysum_{j=1}^k \eps_j > 0$ и $\mysum_{j=1}^{2n} \eps_j = 0$.
Число таких траекторий обозначим через $\tilde{C_n}$.
\end{definition}

\begin{statement}
Наблюдение: $P(S_1 \neq 0, \dots, S_{2n-1} \neq 0, S_{2n} = 0) = 2 \cdot \tilde{C_n} \cdot (pq)^n$. 
\end{statement}

\begin{definition}
Траектория $(\eps_1, \dots, \eps_{2n})$ длины $2n$ называется
\emph{неотрицательной}, если
$\forall k < 2n~\mysum_{j=1}^k \eps_j \geq 0$ и $\mysum_{j=1}^{2n} \eps_j = 0$.
Число таких траекторий обозначим через $C_n$.
\end{definition}

\begin{statement}
$\tilde{C_n} = C_{n-1}$.
\end{statement}
\begin{proof}
Смотри рисунок 3.

Чтобы получить из положительной траектории неотрицательную, покажем,
что в начале стоит $1$, в конце~-- $-1$. Чтобы получить из неотрицательной положительную,
добавим в начало $1$, а в конец $-1$. Получили биекцию между положительными траекториями длины $2n$
и неотрицательными длины $2n-2$.
\end{proof}

\begin{statement}
Пусть $C_0 = 1$. Тогда $C_n = \mysum_{k=0}^{n-1} C_k \cdot C_{n-1-k}$.
\end{statement}
\begin{proof}
Смотри рисунок 4.

Пусть $2k$~-- первый момент возвращения траектории в ноль.
Ясно, что таких траекторий $\tilde{C_k} \cdot C_{n-k}$. Суммируя по $k=1\dots n$:

$$C_n = \sum_{k=1}^n \tilde{C_k} \cdot C_{n-k} = \sum_{k=0}^{n-1} C_k \cdot C_{n-k-1}$$
\end{proof}

Вывод: $C_n$~-- это числа Каталана. $C_n =\frac{1}{n+1} \cdot \comb{2n}{n}$.

Производящая функция: $f(t) = \mysum_{n=0}^{\infty} C_n t^n = \frac{1}{2t}(1-\sqrt{1-4t}), |t| \leq \frac14$.

\begin{theorem}[распределение момента возвращения в ноль]
$$P(S_1 \neq 0, \dots, S_{2n-1} \neq 0, S_{2n} = 0) = \frac{1}{2n-1}\comb{2n}{n}(pq)^n$$
\end{theorem}
\begin{proof}
\begin{eqnarray*}
&&P(S_1 \neq 0, \dots, S_{2n-1} \neq 0 , s_{2n} = 0) = \\
&&2\tilde{C_n} (pq)^n = \text{| утв.1 |} = \\
&&2 C_{n-1} (pq)^n = \\
&&\frac2n C_{2n-2}{n-1} (pq)^n = \\
&&\frac{2n}{n\cdot n} C_{2n-2}^{n-1}(pq)^n = \\
&&\frac{1}{2n-1}C_{2n}^{n}(pq)^n.
\end{eqnarray*}
\end{proof}

\begin{theorem}[вероятность возвращения в ноль]

$$P(\{S_n, n \geq 1\} \text{вернется в ноль}) = 1 - |p-q|$$
\end{theorem}
\begin{proof}
\begin{eqnarray*}
&& P(\{S_n, n >= 1\} \text{вернется в 0}) = \\
&& \mysum_{n=1}^{\infty} P(S_1 \neq 0, \dots, S_{2n} = 0) = \\
&& \mysum_{n=1}^{\infty} 2 \tilde{C_N} (pq)^n = \\
&& \mysum_{n=1}^{\infty} 2 C_{n-1} (pq)^n = \\ 
&& \mysum_{n=0}^{\infty} 2 C_n(pq)^{n-1} = \\
&& 2 pq \cdot f(pq) = \\
&& 2 pq * \frac{1}{2pq}(1 - \sqrt{1-4pq}) = |\text{ т.к. } (p+q)^2 = 1| = \\
&& 1 - \sqrt{(p-q)^2} = 1 - |p-q|
\end{eqnarray*}
\end{proof}

\begin{corollary}
Симметричное случайное блуждание на прямой возвратно с вероятностью 1.
\end{corollary}

\subsection{Среднее время нахождения в нуле}
Пусть $(S_n, n \in \myN)$~-- простейшее симметричное случайное блуждание на прямой.
Обозначим через $L_n(0)$ число нулей в последовательности $S_k, k = 0 \dots n$.
\underline{Вопрос}: $EL_n(0) \sim ?$

\begin{lem}
$$EL_n(0) = E|S_{n+1}|$$
\end{lem}
\begin{proof}
Рассмотрим $|S_{n+1}|$.
$$
|S_{n+1}| = |S_n + \xi_{n+1}| = \left\lbrace
\begin{array}{l r}
	S_n + \xi_{n+1}, & S_n > 0 \\
	1, & S_n = 0 \\
	-(S_n + \xi_{n+1}), & S_n < 0
\end{array}
\right.
$$

$|S_{n+1}| = (S_n + \xi_{n+1})\ind{S_n > 0} + \ind{S_n = 0}
- (S_n + \xi_{n+1}) \ind{S_n < 0} =
\ind{S_n = 0} + (S_n + \xi_{n+1}) \sign(S_n)$

Отсюда $|S_{n+1}| = \ind{S_n = 0} + |S_n| + \xi_{n+1} \sign(S_n) = $ |индукция|
$ = \mysum_{k=0}^n (\ind{S_k = 0} + \xi_{k+1} \sign(S_k)) = L_n(0) +
\mysum_{k=0}^n \xi_{k+1}\sign(S_k)$.

Берем матожидание у обеих частей равенства:
$E|S_{n+1}| = EL_n(0) + \mysum_{k=0}^nE(\xi_{k+1}\sign(S_k)) =
EL_n(0) + \mysum_{k=0}^n E\xi_{k+1} E\sign(S_k) = EL_n(0)$.
\end{proof}

Согласно ЦПТ,
$\frac{S_n}{\sqrt{n}} \overset{d}{\to} \eta \sim \mathcal{N}(0, 1)$.

По теореме о наследовании сходимости
$\frac{|S_n|}{\sqrt{n}} \overset{d}{\to} |\eta| \sim |\mathcal{N}(0, 1)|$.

\underline{Вопрос}: верно ли данное?

$$E \frac{|S_n|}{\sqrt{n}} \to E|\eta| =
\int_\myR \frac{1}{\sqrt{\frac{2}{\pi}}} |x| e^{-\frac{x^2}{2}} dx = \sqrt{\frac{2}{\pi}}$$

\begin{definition}
Множество случайных величин $\{\xi_\alpha, \alpha \in \mathfrak{A}\}$ называется
\emph{равномерно интегрируемым}, если
$$\mylim_{c \to +\infty}\sup_{\alpha \in \mathfrak{A}}
E(|\xi_\alpha| \: \ind{|\xi_\alpha| \geq c}) = 0$$
\end{definition}

Смысл: <<хвосты>> распределения равномерно малы.

\begin{theorem}[б/д]
Пусть $\{\xi_n, n \in \myN\}$~-- с.в., $\xi_n \overset{d}{\to} \xi$. Тогда 
$$E\xi_n \to E\xi \Leftrightarrow \{\xi_n, n \in \myN\} \text{ равномерно интегрируемо}$$
\end{theorem}
\begin{remark}
Если сходимость $\overset{p}{\to}$ или $\overset{\text{п.н.}}{\to}$, то равномерная интергрируемость $\Leftrightarrow \xi_n \overset{L_1}{\to} \xi$.
\end{remark}

\begin{theorem}[достаточное условие равномерной интегрируемости]
Пусть $\{\xi_n, n \in \myN\}$~-- с.в., $G(t) \geq 0: \frac{G(t)}{t} \to +\infty
\text{ при } t \to +\infty$.
Если $\mysup_n EG(|\xi_n|) < +\infty$, то последовательность равномерно интегрируема.
\end{theorem}
\begin{proof}
Положим $M = \mysup_n EG(|\xi_n|)$. $\forall \eps > 0$ положим $a = \frac{M}{\eps}$.
Возьмем $c > 0: \frac{G(t)}{t} > a \; \forall t > c$. Тогда
$\forall t > c, \forall n \in \myN$

$$E(|\xi_n|\ind{|\xi_n| \geq t}) \leq E\pars{\frac{G(|\xi_n|)}{a} \ind{|\xi_n| >= t}} \leq
E\frac{G(|\xi_n|)}{a} \leq \frac{M}{a} = \eps$$
$\Leftrightarrow \{\xi_n, n \in \myN\}$ равномерно интегрируема.
\end{proof}

\begin{theorem}[среднее время в нуле]
$$EL_n(0) \sim \sqrt{\frac{2n}{\pi}}$$
\end{theorem}
\begin{proof}
Согласно лемме, $EL_n(0) = E|S_{n+1}|$. Покажем, что
$\{\xi_n = \frac{S_n}{\sqrt{n}}\}$ равномерно интегрируема. Подберем соответствующую функцию $G$.
Попробуем $G(t) = t^2$.

$$EG(|\xi_n|) = E\xi_n^2 = \frac{(E\xi_n^2)}{n} = \frac{DS_n}{n}
= \frac{\mysum_{k=1}^n D\xi_k}{n} = \frac{n}{n} = 1$$

Согласно достаточному условию, получили, что последовательность $\frac{|S_n|}{\sqrt{n}}$ равномерно
интегрируема. Тогда по теореме
$$\frac{E|S_n|}{\sqrt{n}} \to E\eta = \sqrt{\frac{2}{\pi}} (\eta \sim \mathcal{N}(0, 1))$$

Отсюда
$$EL_n(0) = E|S_{n+1}| \sim \sqrt{\frac{2(n+1)}{\pi}} \sim \sqrt{\frac{2n}{\pi}}$$
\end{proof}

\subsection{Свойства траекторий}
\begin{theorem}[закон повторного логарифма, б/д]
$$P \pars {\mylimsup{n\to\infty} \frac{S_n}{\sqrt{2n \ln \ln n}} = 1 } = 1$$
\end{theorem}
\begin{corollary}
$$P \pars {\myliminf{n\to\infty} \frac{S_n}{\sqrt{2n \ln \ln n}} = -1 } = -1$$
\end{corollary}
\begin{proof}
Рассмотрим $X_n = -S_n$~-- симметричное случайное блуждание $\Leftrightarrow$ по ЗПЛ
получаем, что 
$$ \text{п.н. } 1 = \mylimsup{n}\frac{X_n}{\sqrt{2n \ln \ln n}} =
-\myliminf{n}\frac{S_n}{\sqrt{2n \ln \ln n}}$$
\end{proof}

Смысл: 
--- рис.5 ---


ЗПЛ означает, что с вероятностью $1$ траектория случайного блуждания начиная с некоторого
момента лежит внутри между кривыми $\pm(1+\eps)\sqrt{2n \ln \ln n}$ и в то же время
бесконечно много раз выходит в обе стороны из области, ограниченной кривыми
$\pm(1-\eps)\sqrt{2n \ln \ln n}$.


\section{Ветвящиеся процессы Гальтона-Ватсона}
Физическая модель: --- рис.6 ---
В каждый следующий момент времени каждая частица распадается на некоторое случайное число
таких же частиц.

Мат. модель:
Пусть $\xi$~-- случайная величина со значениями в $\myZ_+$.
$\{\xi_k^{(n)}, k, n \in \myN\}$~-- независимые случайные величины с тем
же распределением, что и $\xi$. Положим

$$X_0 = 1, X_1 = \xi_1^{(1)}, X_n = \mysum_{k=1}^{X_{n-1}} \xi_k^{(n)}$$

\begin{definition}
$\{X_n, n \in \myZ_+\}$~-- \emph{ветвящийся процесс Гальтона-Ватсона}, построенный по с.в. $\xi$.
\end{definition}

\begin{itemize}
\item $X_n$~-- число частиц в $n$-м поколении
\item $\xi_k^{(n)}$~-- число потомков $k$-й частицы в $n-1$-м поколении
\end{itemize}

\underline{Вопрос}: какова вероятность вырождения процесса?

\subsection{Производящие функции}
\begin{definition}
Пусть $\xi$~-- случайная величина. Тогда ее \emph{производящей функцией} называется
$$\phi_\xi(z) = Ez^\xi, z \in \myR$$
\end{definition}

\paragraph{Свойства производящих функций}

\begin{enumerate}
\item $\phi_\xi(1) = 1$
\item $\phi'_\xi(1) = E\xi$
\item Если $\xi$ и $\eta$ независимы, то $\phi_{\xi + \eta}(z) = \phi_\xi(z) \phi_\eta(z)$
\end{enumerate}

Если $\xi$ принимает значения в $\myZ_+$, то введем $p_k = P(\xi = k), k \in \myZ_+$.

\begin{enumerate}
\setcounter{enumi}{3}
\item $\phi_\xi(z) = \mysum_{k=0}^{\infty}z^k p_k$
\item $\phi_\xi(0) = p_0$
\item $p_k = \frac{f_\xi^{(k)}(z)}{k!}$
\item Ряд для $\phi_\xi(z)$ сходится абсолютно и равномерно в области $\{|z| \leq 1\}$
\item $\phi_\xi(z)$ непрерывно дифференцируема бесконечное число раз в области $\{|z| < 1\}$
\end{enumerate}

Пусть далее $\{X_n, n \in \myZ_+\}$~-- ветвящийся процесс Г.-В., построенный по $\xi$.

\begin{lem} \label{lem:phi_xi_n}
$$\phi_{X_{n_1}}(z) = \phi_{X_n}(\phi_\xi(z))$$
\end{lem}
\begin{proof}
$\phi_{X_{n+1}}(z) = Ez^{X_{n+1}}$

$$
 E \pars {z^{X_{n+1}} | X_n = m} =
 E \pars {z^{\mysum_{k=1}^{X_n} \xi_k^{(n+1)}} \vline X_n = m} =
 E \pars {z^{\mysum_{k=1}^m \xi_k^{(n+1}} \vline X_n = m} =
 E z^{\mysum_{k=1}^m \xi_k^{(n+1)}} =
 (\phi_\xi(z))^m
$$

Значит,
$$\phi_{X_{n+1}} = Ez^{X_{n+1}} = E(E(z^{X_{n+1}}|X_n)) =
E(\phi_\xi(z))^m \vline_{m = X_n} = \phi_{X_n}(\phi_\xi(z))$$
\end{proof}

\begin{corollary} \forcenewline
\begin{enumerate}
\item $\phi_{X_n}(z) =
	\underset{n \text{ раз}}{\underbrace{\phi_\xi(\phi_\xi(\dots\phi_\xi(}} z)\dots))$
\item $\phi_{X_{n+1}}(z) = \phi_\xi(\phi_{X_n}(z))$
\begin{proof}
Применяем индуктивно лемму \ref{lem:phi_xi_n}:
$$\phi_{X_{n+1}} = 
	\underset{n+1 \text{ раз}}{\underbrace{\phi_\xi(\phi_\xi(\dots\phi_\xi(}} z)\dots)) =
	\phi_\xi(\phi_{X_n}(z))$$
\end{proof}
\end{enumerate}
\end{corollary}

\subsection{Вероятность вырождения процесса} \forcenewline
Положим $q_n = P(X_n = 0)$, $q = P(\text{процесс выродился}) = P(\exists n: X_n = 0)$
\begin{lem}
$$q_n \leq q_{n+1} \text{  и  } q = \mylim_n q_n$$
\end{lem}
\begin{proof}
$\{X_n = 0\} \subset \{X_{n+1} = 0\} \Rightarrow q_n \leq q_{n+1}$

Но $P(\exists n: X_n = 0) = P\pars{\mybigcup_n \{X_n = 0\}} =$
|по непрерывности вероятностной меры| $ = \mylim_n P(X_n = 0) = \mylim_n q_n$.
\end{proof}

\end{document}
