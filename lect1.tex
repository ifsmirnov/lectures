% Title

\ifdefined\HeaderIncluded
\else

\newcommand{\HeaderIncluded}{}

\documentclass[a4paper, 10pt]{article}
\usepackage[utf8]{inputenc}
\usepackage[english, russian]{babel}
\usepackage{amsmath, amsfonts, amssymb, amsthm}
\usepackage{tikz} % some graphics
%\usepackage[indentheadings]{russcorr}
\usepackage{geometry}
\usetikzlibrary{arrows}
\setlength{\parindent}{0pt}


\geometry{left=2cm}
\geometry{right=2cm}
\geometry{top=2cm}
\geometry{bottom=2cm}

\hfuzz=18pt

% Commands

\newcommand*{\hm}[1]{#1\nobreak\discretionary{}%
{\hbox{$\mathsurround=0pt #1$}}{}} % a\hm=b makes "=" carriable to the next line with duplication of the sign

\newcommand{\combus}[2]{\left(\begin{array}{c}#1 \\ #2 \end{array} \right)} % american style for C_n^k
\newcommand{\combru}[2]{C_{#1}^{#2}} % russian C_n^k
\newcommand{\comb}[2]{\combru{#1}{#2}}
\newcommand{\myN}{\mathbb{N}} % nice letters for common number sets
\newcommand{\myZ}{\mathbb{Z}}
\newcommand{\myR}{\mathbb{R}}
\newcommand{\myC}{\mathbb{C}}
\newcommand{\myQ}{\mathbb{Q}}
\newcommand{\myE}{\mathcal{E}} % basis
\newcommand{\myF}{\mathcal{F}} % basis
\newcommand{\myB}{\mathcal{B}} % set system (random processes)
\newcommand{\myM}{\mathcal{M}} % set system (random processes)
\newcommand{\mysetP}{\mathcal{P}}
\newcommand{\mysetR}{\mathcal{R}}
\newcommand{\mysetM}{\mathcal{M}}
\newcommand{\mysetN}{\mathcal{N}}
\newcommand{\mysetX}{\mathcal{X}}
\newcommand{\walls}[1]{\left | #1 \right |} % |smth_vertically_large|
\newcommand{\pars}[1]{\left( #1 \right)} % (smth_vertically_large)
\newcommand{\class}[1]{[ #1 ]} % [smth_vertically_large]
\newcommand{\bra}[1]{\langle #1 \rangle} % brackets for span of vectors, eg. <e_1, ..., e_k>
\newcommand{\myset}[1]{\left\{ #1 \right\}} % because { and } are special symbols in TeX
\newcommand{\mysetso}[2]{\myset{#1 \mid #2}}
\newcommand{\myforall}[1]{\forall #1:}
\newcommand{\myexists}[1]{\exists #1:\ }
\newcommand{\forcenewline}{\ \newline}
\newcommand{\reversein}{\ni}
\renewcommand{\phi}{\varphi}
\newcommand{\eps}{\varepsilon}
\newcommand{\ind}[1]{\mathrm{I}\{ #1 \}}
\newcommand{\myliminf}[1]{\underset{ #1 }{\underline{\lim}}}
\newcommand{\mylimsup}[1]{\underset{ #1 }{\overline{\lim}}}
\newcommand{\seq}[3]{#1_{#2}, \dots, #1_{#3}}

%
% Example text:
% M = {x^2 | x is prime}
%
% Corresponding markup:
% $$ \mysetM = \mysetso{x^2}{x \text{ is prime}} $$
%
\renewcommand{\leq}{\leqslant}
\renewcommand{\geq}{\geqslant}
\newcommand{\myempty}{\varnothing}
\renewcommand{\emptyset}{\varnothing}
\newcommand{\myand}{\;\; \hm\& \;\;}
\newcommand{\myor}{\; \hm\vee \;}
\newcommand{\conj}[1]{\overline{#1}} % complex conjugation
\newcommand{\mycirc}{\circ}
\newcommand{\poly}[2]{#1 [ #2 ]} % ring of polynomials
\newcommand{\Rx}{\poly{\myR}{x}}
\newcommand{\Cx}{\poly{\myC}{x}}
\newcommand{\mydim}[1]{\dim #1}
\newcommand{\mycodim}[1]{\text{codim} #1}
\newcommand{\coords}[2]{\pars{#1}_{#2}}
\newcommand{\myrank}[1]{\text{rank} #1}
\newcommand{\mysegment}[2]{[#1, #2]}
\newcommand{\myinterval}[2]{(#1, #2)}
\newcommand{\mypair}[2]{(#1, #2)}
\newcommand{\myfunc}[3]{#1\!:\,#2 \hm\to #3} % TODO: make spaces between elements of this tag look better
\newcommand{\suchthat}{\!:\,}
\newcommand{\mydef}[1]{\emph{#1}}

% Arrows
\newcommand{\myright}{\;\hm\Rightarrow\;}
\newcommand{\myleft}{\;\hm\Leftarrow\;}
\newcommand{\myleftright}{\;\hm\Leftrightarrow\;}
% \newcommand{\vect}[1]{\overrightarrow{\vphantom{b}#1}}
\newcommand{\vect}[1]{\Vec{\vphantom{b}#1}}

% Sums, prods and other things with \limits
\newcommand{\mysum}{\sum\limits}
\newcommand{\myprod}{\prod\limits}
\newcommand{\mylim}{\lim\limits}
\newcommand{\mysup}{\sup\limits}
\newcommand{\myinf}{\inf\limits}
\newcommand{\mybigcup}{\bigcup\limits}
\newcommand{\mybigcap}{\bigcap\limits}
\newcommand{\mybigor}{\bigvee\limits}
\newcommand{\mybigand}{\bigwedge\limits}
\newcommand{\mybigxor}{\bigoplus\limits}
\newcommand{\xor}{\oplus}
\newcommand{\eq}{\equiv}

\newcommand{\smatrix}[1]{\begin{smallmatrix}#1\end{smallmatrix}}
\newcommand{\psmatrix}[1]{\pars{\begin{smallmatrix}#1\end{smallmatrix}}}
\newcommand{\wsmatrix}[1]{\walls{\begin{smallmatrix}#1\end{smallmatrix}}}


\newcommand{\abs}[1]{|#1|}
% Operators
\DeclareMathOperator{\pr}{pr} % Проекция
\DeclareMathOperator{\Tr}{Tr} % След
\DeclareMathOperator{\mychar}{char} % Характеристика поля
\DeclareMathOperator{\id}{id} % Тождественное преобразование
\DeclareMathOperator{\rk}{rk} % Ранг
\DeclareMathOperator{\Kernel}{Ker}
\DeclareMathOperator{\Image}{Im}
\DeclareMathOperator{\sign}{sign} % Знак
\DeclareMathOperator{\sgn}{sign} % Знак
\DeclareMathOperator{\const}{const}

% Special theorems
\newtheorem*{theorem-menelaus}{Теорема Менелая}

% Theorems
\newtheorem*{theorem-star}{Теорема}
\newtheorem{theorem}{Теорема}[section]
\newtheorem*{theorem-definition-star}{Теорема-определение}
\newtheorem*{corollary-star}{Следствие}
\newtheorem{corollary}{Следствие}[section]
\newtheorem*{property-star}{Свойство}
\newtheorem*{properties}{Свойства}
\newtheorem{property}{Свойство}
\newtheorem*{lem-star}{Лемма}
\newtheorem{lem}{Лемма}[section]
\newtheorem*{proposition-star}{Предложение}
\newtheorem{proposition}{Предложение}
\newtheorem{stage}{Этап}
\newtheorem*{statement}{Утверждение}
\newtheorem*{designation}{Обозначение}
\newtheorem*{usage}{Использование}

\theoremstyle{remark}
\newtheorem*{remark}{Замечание}

\theoremstyle{definition}
\newtheorem{problem}{Задача}
\newtheorem{exercise}{Упражнение}

\theoremstyle{definition}
\newtheorem*{definition-star}{Определение}
\newtheorem{definition}{Определение}[section]

\theoremstyle{definition}
\newtheorem*{example-star}{Пример}
\newtheorem{example}{Пример}
\newtheorem*{examples}{Примеры}
\newtheorem{algorithm}{Алгоритм}
\newtheorem{case}{Случай}
\newtheorem*{case-star}{Случай}

\newcommand{\rightoffset}[1]{
\begin{flushright} \begin{minipage}{0.9\textwidth}
\begin{tabular}{|l} \begin{minipage}{\textwidth}
#1
\end{minipage}
\end{tabular}
\end{minipage}
\end{flushright}
}

\newcommand{\longformula}[1]{
$$\begin{array}{l}
#1
\end{array}$$
}


% Style
\newcommand{\tocstyle}{\setlength{\parindent}{0ex} \setlength{\parskip}{0ex}}
\newcommand{\mainstyle}{\setlength{\parindent}{0ex} \setlength{\parskip}{1ex}}

\mainstyle
\setcounter{secnumdepth}{2}
\renewcommand{\thesubsection}{\arabic{subsection}}


% Titles of lectures
\newcommand{\mylecture}[1]{\setcounter{secnumdepth}{-1} \section{#1} \setcounter{secnumdepth}{2} \setcounter{subsection}{0} \setcounter{corollary}{0} \setcounter{definition}{0} \setcounter{theorem}{0}}

\fi


\title{Случайные процессы}

\begin{document}

\maketitle
\tableofcontents

\newpage

\part*{Введение. Историческая справка}
\underline{Теория вероятностей}: математический анализ случайных явлений.\\
\underline{Теория случайных процессов}: стохастические модели и фактор времени.

\paragraph{Предпосылки к изучению}
\begin{itemize}
\item 1827, Р. Броун~-- броуновское движение частиц в воде~$\Rightarrow$ процесс броуновского движения
\item 1903, Л. Башелье~-- колебания курсов бумаг на бирже~$\Rightarrow$ процесс броуновского движения
\item 1906, А.А. Марков~-- анализ комбинаций гласных и согласных в романе <<Евгений Онегин>>~$\Rightarrow$ марковские цепи
\item 1903, Ф. Лундберг~-- модель деятельности страховой компании~$\Rightarrow$ пуассоновский процесс
\item 1873, Ф. Гальтон, Г. Ватсон~-- анализ вымирания аристократических фамилий в Великобритании~$\Rightarrow$ ветвящиеся процессы
\item Начало XX века, А. Эрланг~-- изучение загрузки телефонных сетей~$\Rightarrow$ теория массового обслуживания
\end{itemize}

\paragraph{Применения}
\begin{itemize}
\item Физика (стохастическое исчисление, теория гиббсовских полей)
\item Экономика (финансовая математика)
\item Биология
\end{itemize}

\part{Случайные процессы}
\section*{Общие определения}
\begin{definition}
Пусть $(\Omega, F, P)$~-- вероятностное пространство, а $(E, \mathcal{E})$~-- измеримое пространство.
Отображение $\xi: \Omega \rightarrow E$ называется \emph{случайным элементом}, если оно измеримо,
т.е. $$\forall B \in \mathcal{E}\ \xi^{-1}(B) = \{\omega: \xi(\omega) \in B\} \in F$$
\end{definition}
\begin{definition}
Пусть $T$~-- некоторое множество и на $(\Omega, F,P)$ для $\forall t\in T$ задан случайный элемент $X_t$. Тогда набор $X = {X_t, t \in T}$ называется \emph{случайной функцией} на множестве $T$. 
\end{definition}

\begin{remark}
Вообще говоря, не предполагается, что все $X_t$ принимают значения в одном и том же пространстве.
\end{remark}

\begin{definition}
Пусть $X = (X(t, \omega))$. При фиксированном $\omega = \omega_0$ функция
$$\tilde{X}_{\omega_0}(t) = X_t(\omega) \vert _{\omega=\omega_0}$$
на $T$ называется \emph{траекторией} (или реализацией) случайной функции $X = {X_t, t\in T}$. 
\end{definition}

\subsection{Терминология}
\begin{itemize}
\item Если $T \subset \mathbb{R}$, то случайная функция называется случайным процессом.
\item Если $T = [a, b], (a, b), [a, +\infty)$ и т.д., то процесс $X$ называется процессом с непрерывным временем.
\item Если $T = \mathbb{N}, \mathbb{Z}$ и т. д. то процесс $X$ называется процессом с дискретным временем.
\item Если $T \subset \mathbb{R}^d$, то процесс $X$ называется случайным полем.
\end{itemize}

\begin{remark}
Далее всюду будем использовать термин <<случайный процесс>>.
\end{remark}

\subsection{Примеры}
\begin{enumerate}
\item $X_t(\omega) = \xi(\omega) \cdot f(t)$, где $\xi(\omega)$~-- с.в., $f(t)$~-- детерминированная функция.
\item Пусть $\{\xi_n, n\in\mathbb{N}\}$~-- независимые случайные векторы, $S_n = \xi_1 + \dots + \xi_n$,
$S_0 = 0$. Тогда процесс с дискретным временем $\{S_n, n \in \mathbb{Z}_+\}$
называется случайным блужданием.

Траектория: см. рис. 1
\item Пусть $\{\xi_n, n \in \mathbb{N}\}$~-- норсв, $\xi_n >= 0$, $\xi_n \neq const$ п.н.,
$S_n = \xi_1 + \dots + \xi_n$, $S_0 = 0$. Тогда процесс
$$X_t = \sup\{n: S_n <= t\}, t >= 0$$
называется процессом восстановления.

Траектория: см. рис. 2

\begin{statement}
Процесс восстановления конечен почти наверное.
\end{statement}
\begin{proof}
Пусть сначала $E\xi_i = a > 0$.

Заметим, что $\{S_k \leq t\} \supset \{S_{k+1} \leq t\}$.

$\{X_t=+\infty\} = \{\sup\{n:S_n\leq t\} = +\infty\} =
\{\forall n: S_n \leq t\} = \mybigcap_n\{S_n \leq t\} \Rightarrow$
|по непрерывности вероятностной меры|
$\Rightarrow P(X_t = +\infty) = P(\forall n S_n \leq t) =
\mylim_{n \to \infty} P(S_n \leq t)
= \mylim_{n \to \infty} P(S_n \leq t_n) \leq$ % что такое t_n?
|для больших n|
$\leq \mylim_{n \to \infty} P(\frac{S_n}{n} < \frac{a}{2})$.

Но по УЗБЧ $S_n \to a \Rightarrow P(S_n \leq a/2) \to P(a \leq a/2) = 0$.

В силу того, что $X_t\uparrow$ при $t\uparrow$,
$P(\exists t: X_t = +\infty) = P(\exists b: x_n = +\infty) = 0$. % что это за b и что тут должно быть?

Если $E\xi_i = +\infty$, то случай сводится к предыдущему:
$\exists C > 0: \tilde{\xi_n} = \min(\xi_n, C), E\tilde{\xi_n} > 0$.
Тогда $P(X_t = +\infty) = P(\forall n S_n \leq t) \leq P(\tilde{S_n} \leq t) = 0$
(доказали в случае конечного матожидания).
\end{proof}

Откуда может возникнуть процесс восстановления?
Физическая модель~-- <<Модель перегорания лампочки>>. $\xi_n$~-- случайная величина, равная времени работы лампочки, $X_t$~-- сколько раз пришлось заменить лампочку в к моменту времени $t$.

\item Модель страхования Крамера-Лундберга

Пусть есть $\{\xi_n, n \in \myN\}, \{\eta_m, m \in \myN\},
\xi_n \overset{d}{=} \xi_m,
\eta_n \overset{d}{=} \eta_m,
\{\xi_n, \eta_m\}$ независимы,
$\xi_n, \eta_n \geq 0, \xi_n$ невырождены.

Пусть $\{X_t, t \geq 0\}$~-- процесс восстановления, построенный по случайным величинам
$\{\xi_n, n \in \myN\}, y_0, c > 0$. Тогда
$Y_t = y_0 + c \cdot t - \mysum_{k=1}^{X_t} \eta_k$~--
модель страхования Крамера-Лундберга.

\paragraph*{Смысл параметров}
\begin{itemize}
\item $y_0$~-- начальный капитал
\item $c$~-- скорость поступления страховых взносов
\item $S_k = \xi_1 + \dots + \xi_k$~-- время $k$-й выплаты, $\eta_k$~-- размер этой выплаты
\item $X_t$~-- число выплат к моменту времени $t > 0$
\item $\mysum_{k=1}^{X_t} \eta_k$~-- общий размер выплат к этому моменту времени
\item $Y_t$~-- текущий капитал компании
\end{itemize}
\end{enumerate}

\section{Случайное блуждание на прямой}

\begin{definition}
Пусть ${\xi_n, n\in \myN}$~-- норсв, $P(\xi_n = 1) = p$, $P(\xi_n = -1) = q = 1-p$. Тогда процесс
$(S_n, n \in \myZ_+), S_0 = 0, S_n = \xi_1 + \dots + \xi_n$ называется \emph{простейшим случайным блужанием на прямой}.
Если $p = q = \frac{1}{2}$, то блуждание называется \emph{симметричным}.
\end{definition}

\subsection{Вопросы}
\begin{enumerate}
\item вероятность возвращения в ноль
\item распределение первого момента возвращения в ноль
\item среднее время в нуле
\item геометрия траектории
\end{enumerate}
\begin{remark}
Последние два вопроса~-- только для симметричного случая.
\end{remark}

\subsection{Возвращение в ноль}

$P(\{S_n, n \in \myN\}~\text{вернется в ноль})$~-- ?

$P(s1 \neq 0, s2 \neq 0, \dots, s_{2n-1} \neq 0, s_{2n} = 0)$~-- ?

Вероятность каждой траектории, приводящей в $0$ в момент времени $2n$, одна и та же и равна $(pq)^n$.
Каждую траекторию длины $2n$ можно сопоставить с вектором
$\{\eps_1, \dots, \eps_{2n}\}, \eps_i \in \{-1, 1\}$. 

\begin{definition}
Траектория $(\eps_1, \dots, \eps_{2n})$ длины $2n$ называется
\emph{положительной}, если
$\forall k < 2n~\mysum_{j=1}^k \eps_j > 0$ и $\mysum_{j=1}^{2n} \eps_j = 0$.
Число таких траекторий обозначим через $\tilde{C_n}$.
\end{definition}

\begin{statement}
Наблюдение: $P(S_1 \neq 0, \dots, S_{2n-1} \neq 0, S_{2n} = 0) = 2 \cdot \tilde{C_n} \cdot (pq)^n$. 
\end{statement}

\begin{definition}
Траектория $(\eps_1, \dots, \eps_{2n})$ длины $2n$ называется
\emph{неотрицательной}, если
$\forall k < 2n~\mysum_{j=1}^k \eps_j \geq 0$ и $\mysum_{j=1}^{2n} \eps_j = 0$.
Число таких траекторий обозначим через $C_n$.
\end{definition}

\begin{statement}
$\tilde{C_n} = C_{n-1}$.
\end{statement}
\begin{proof}
Смотри рисунок 3.

Чтобы получить из положительной траектории неотрицательную, покажем,
что в начале стоит $1$, в конце~-- $-1$. Чтобы получить из неотрицательной положительную,
добавим в начало $1$, а в конец $-1$. Получили биекцию между положительными траекториями длины $2n$
и неотрицательными длины $2n-2$.
\end{proof}

\begin{statement}
Пусть $C_0 = 1$. Тогда $C_n = \mysum_{k=0}^{n-1} C_k \cdot C_{n-1-k}$.
\end{statement}
\begin{proof}
Смотри рисунок 4.

Пусть $2k$~-- первый момент возвращения траектории в ноль.
Ясно, что таких траекторий $\tilde{C_k} \cdot C_{n-k}$. Суммируя по $k=1\dots n$:

$$C_n = \sum_{k=1}^n \tilde{C_k} \cdot C_{n-k} = \sum_{k=0}^{n-1} C_k \cdot C_{n-k-1}$$
\end{proof}

Вывод: $C_n$~-- это числа Каталана. $C_n =\frac{1}{n+1} \cdot \comb{2n}{n}$.

Производящая функция: $f(t) = \mysum_{n=0}^{\infty} C_n t^n = \frac{1}{2t}(1-\sqrt{1-4t}), |t| \leq \frac14$.

\begin{theorem}[распределение момента возвращения в ноль]
$$P(S_1 \neq 0, \dots, S_{2n-1} \neq 0, S_{2n} = 0) = \frac{1}{2n-1}\comb{2n}{n}(pq)^n$$
\end{theorem}
\begin{proof}
\begin{eqnarray*}
&&P(S_1 \neq 0, \dots, S_{2n-1} \neq 0 , s_{2n} = 0) = \\
&&2\tilde{C_n} (pq)^n = \text{| утв.1 |} = \\
&&2 C_{n-1} (pq)^n = \\
&&\frac2n C_{2n-2}{n-1} (pq)^n = \\
&&\frac{2n}{n\cdot n} C_{2n-2}^{n-1}(pq)^n = \\
&&\frac{1}{2n-1}C_{2n}^{n}(pq)^n.
\end{eqnarray*}
\end{proof}

\begin{theorem}[вероятность возвращения в ноль]

$$P(\{S_n, n \geq 1\} \text{вернется в ноль}) = 1 - |p-q|$$
\end{theorem}
\begin{proof}
\begin{eqnarray*}
&& P(\{S_n, n >= 1\} \text{вернется в 0}) = \\
&& \mysum_{n=1}^{\infty} P(S_1 \neq 0, \dots, S_{2n} = 0) = \\
&& \mysum_{n=1}^{\infty} 2 \tilde{C_N} (pq)^n = \\
&& \mysum_{n=1}^{\infty} 2 C_{n-1} (pq)^n = \\ 
&& \mysum_{n=0}^{\infty} 2 C_n(pq)^{n-1} = \\
&& 2 pq \cdot f(pq) = \\
&& 2 pq * \frac{1}{2pq}(1 - \sqrt{1-4pq}) = |\text{ т.к. } (p+q)^2 = 1| = \\
&& 1 - \sqrt{(p-q)^2} = 1 - |p-q|
\end{eqnarray*}
\end{proof}

\begin{corollary}
Симметричное случайное блуждание на прямой возвратно с вероятностью 1.
\end{corollary}

\subsection{Среднее время нахождения в нуле}
Пусть $(S_n, n \in \myN)$~-- простейшее симметричное случайное блуждание на прямой.
Обозначим через $L_n(0)$ число нулей в последовательности $S_k, k = 0 \dots n$.
\underline{Вопрос}: $EL_n(0) \sim ?$

\begin{lem}
$$EL_n(0) = E|S_{n+1}|$$
\end{lem}
\begin{proof}
Рассмотрим $|S_{n+1}|$.
$$
|S_{n+1}| = |S_n + \xi_{n+1}| = \left\lbrace
\begin{array}{l r}
	S_n + \xi_{n+1}, & S_n > 0 \\
	1, & S_n = 0 \\
	-(S_n + \xi_{n+1}), & S_n < 0
\end{array}
\right.
$$

$|S_{n+1}| = (S_n + \xi_{n+1})\ind{S_n > 0} + \ind{S_n = 0}
- (S_n + \xi_{n+1}) \ind{S_n < 0} =
\ind{S_n = 0} + (S_n + \xi_{n+1}) \sign(S_n)$

Отсюда $|S_{n+1}| = \ind{S_n = 0} + |S_n| + \xi_{n+1} \sign(S_n) = $ |индукция|
$ = \mysum_{k=0}^n (\ind{S_k = 0} + \xi_{k+1} \sign(S_k)) = L_n(0) +
\mysum_{k=0}^n \xi_{k+1}\sign(S_k)$.

Берем матожидание у обеих частей равенства:
$E|S_{n+1}| = EL_n(0) + \mysum_{k=0}^nE(\xi_{k+1}\sign(S_k)) =
EL_n(0) + \mysum_{k=0}^n E\xi_{k+1} E\sign(S_k) = EL_n(0)$.
\end{proof}

Согласно ЦПТ,
$\frac{S_n}{\sqrt{n}} \overset{d}{\to} \eta \sim \mathcal{N}(0, 1)$.

По теореме о наследовании сходимости
$\frac{|S_n|}{\sqrt{n}} \overset{d}{\to} |\eta| \sim |\mathcal{N}(0, 1)|$.

\underline{Вопрос}: верно ли данное?

$$E \frac{|S_n|}{\sqrt{n}} \to E|\eta| =
\int_\myR \frac{1}{\sqrt{\frac{2}{\pi}}} |x| e^{-\frac{x^2}{2}} dx = \sqrt{\frac{2}{\pi}}$$

\begin{definition}
Множество случайных величин $\{\xi_\alpha, \alpha \in \mathfrak{A}\}$ называется
\emph{равномерно интегрируемым}, если
$$\mylim_{c \to +\infty}\sup_{\alpha \in \mathfrak{A}}
E(|\xi_\alpha| \: \ind{|\xi_\alpha| \geq c}) = 0$$
\end{definition}

Смысл: <<хвосты>> распределения равномерно малы.

\begin{theorem}[б/д]
Пусть $\{\xi_n, n \in \myN\}$~-- с.в., $\xi_n \overset{d}{\to} \xi$. Тогда 
$$E\xi_n \to E\xi \Leftrightarrow \{\xi_n, n \in \myN\} \text{ равномерно интегрируемо}$$
\end{theorem}
\begin{remark}
Если сходимость $\overset{p}{\to}$ или $\overset{\text{п.н.}}{\to}$, то равномерная интергрируемость $\Leftrightarrow \xi_n \overset{L_1}{\to} \xi$.
\end{remark}

\begin{theorem}[достаточное условие равномерной интегрируемости]
Пусть $\{\xi_n, n \in \myN\}$~-- с.в., $G(t) \geq 0: \frac{G(t)}{t} \to +\infty
\text{ при } t \to +\infty$.
Если $\mysup_n EG(|\xi_n|) < +\infty$, то последовательность равномерно интегрируема.
\end{theorem}
\begin{proof}
Положим $M = \mysup_n EG(|\xi_n|)$. $\forall \eps > 0$ положим $a = \frac{M}{\eps}$.
Возьмем $c > 0: \frac{G(t)}{t} > a \; \forall t > c$. Тогда
$\forall t > c, \forall n \in \myN$

$$E(|\xi_n|\ind{|\xi_n| \geq t}) \leq E\pars{\frac{G(|\xi_n|)}{a} \ind{|\xi_n| >= t}} \leq
E\frac{G(|\xi_n|)}{a} \leq \frac{M}{a} = \eps$$
$\Leftrightarrow \{\xi_n, n \in \myN\}$ равномерно интегрируема.
\end{proof}

\begin{theorem}[среднее время в нуле]
$$EL_n(0) \sim \sqrt{\frac{2n}{\pi}}$$
\end{theorem}
\begin{proof}
Согласно лемме, $EL_n(0) = E|S_{n+1}|$. Покажем, что
$\{\xi_n = \frac{S_n}{\sqrt{n}}\}$ равномерно интегрируема. Подберем соответствующую функцию $G$.
Попробуем $G(t) = t^2$.

$$EG(|\xi_n|) = E\xi_n^2 = \frac{(E\xi_n^2)}{n} = \frac{DS_n}{n}
= \frac{\mysum_{k=1}^n D\xi_k}{n} = \frac{n}{n} = 1$$

Согласно достаточному условию, получили, что последовательность $\frac{|S_n|}{\sqrt{n}}$ равномерно
интегрируема. Тогда по теореме
$$\frac{E|S_n|}{\sqrt{n}} \to E\eta = \sqrt{\frac{2}{\pi}} (\eta \sim \mathcal{N}(0, 1))$$

Отсюда
$$EL_n(0) = E|S_{n+1}| \sim \sqrt{\frac{2(n+1)}{\pi}} \sim \sqrt{\frac{2n}{\pi}}$$
\end{proof}

\subsection{Свойства траекторий}
\begin{theorem}[закон повторного логарифма, б/д]
$$P \pars {\mylimsup{n\to\infty} \frac{S_n}{\sqrt{2n \ln \ln n}} = 1 } = 1$$
\end{theorem}
\begin{corollary}
$$P \pars {\myliminf{n\to\infty} \frac{S_n}{\sqrt{2n \ln \ln n}} = -1 } = -1$$
\end{corollary}
\begin{proof}
Рассмотрим $X_n = -S_n$~-- симметричное случайное блуждание $\Leftrightarrow$ по ЗПЛ
получаем, что 
$$ \text{п.н. } 1 = \mylimsup{n}\frac{X_n}{\sqrt{2n \ln \ln n}} =
-\myliminf{n}\frac{S_n}{\sqrt{2n \ln \ln n}}$$
\end{proof}

Смысл: 
--- рис.5 ---


ЗПЛ означает, что с вероятностью $1$ траектория случайного блуждания начиная с некоторого
момента лежит внутри между кривыми $\pm(1+\eps)\sqrt{2n \ln \ln n}$ и в то же время
бесконечно много раз выходит в обе стороны из области, ограниченной кривыми
$\pm(1-\eps)\sqrt{2n \ln \ln n}$.


\section{Ветвящиеся процессы Гальтона-Ватсона}
Физическая модель: --- рис.6 ---
В каждый следующий момент времени каждая частица распадается на некоторое случайное число
таких же частиц.

Мат. модель:
Пусть $\xi$~-- случайная величина со значениями в $\myZ_+$.
$\{\xi_k^{(n)}, k, n \in \myN\}$~-- независимые случайные величины с тем
же распределением, что и $\xi$. Положим

$$X_0 = 1, X_1 = \xi_1^{(1)}, X_n = \mysum_{k=1}^{X_{n-1}} \xi_k^{(n)}$$

\begin{definition}
$\{X_n, n \in \myZ_+\}$~-- \emph{ветвящийся процесс Гальтона-Ватсона}, построенный по с.в. $\xi$.
\end{definition}

\begin{itemize}
\item $X_n$~-- число частиц в $n$-м поколении
\item $\xi_k^{(n)}$~-- число потомков $k$-й частицы в $n-1$-м поколении
\end{itemize}

\underline{Вопрос}: какова вероятность вырождения процесса?

\subsection{Производящие функции}
\begin{definition}
Пусть $\xi$~-- случайная величина. Тогда ее \emph{производящей функцией} называется
$$\phi_\xi(z) = Ez^\xi, z \in \myR$$
\end{definition}

\paragraph{Свойства производящих функций}

\begin{enumerate}
\item $\phi_\xi(1) = 1$
\item $\phi'_\xi(1) = E\xi$
\item Если $\xi$ и $\eta$ независимы, то $\phi_{\xi + \eta}(z) = \phi_\xi(z) \phi_\eta(z)$
\end{enumerate}

Если $\xi$ принимает значения в $\myZ_+$, то введем $p_k = P(\xi = k), k \in \myZ_+$.

\begin{enumerate}
\setcounter{enumi}{3}
\item $\phi_\xi(z) = \mysum_{k=0}^{\infty}z^k p_k$
\item $\phi_\xi(0) = p_0$
\item $p_k = \frac{f_\xi^{(k)}(z)}{k!}$
\item Ряд для $\phi_\xi(z)$ сходится абсолютно и равномерно в области $\{|z| \leq 1\}$
\item $\phi_\xi(z)$ непрерывно дифференцируема бесконечное число раз в области $\{|z| < 1\}$
\end{enumerate}

Пусть далее $\{X_n, n \in \myZ_+\}$~-- ветвящийся процесс Г.-В., построенный по $\xi$.

\begin{lem} \label{lem:phi_xi_n}
$$\phi_{X_{n_1}}(z) = \phi_{X_n}(\phi_\xi(z))$$
\end{lem}
\begin{proof}
$\phi_{X_{n+1}}(z) = Ez^{X_{n+1}}$

$$
 E \pars {z^{X_{n+1}} | X_n = m} =
 E \pars {z^{\mysum_{k=1}^{X_n} \xi_k^{(n+1)}} \vline X_n = m} =
 E \pars {z^{\mysum_{k=1}^m \xi_k^{(n+1}} \vline X_n = m} =
 E z^{\mysum_{k=1}^m \xi_k^{(n+1)}} =
 (\phi_\xi(z))^m
$$

Значит,
$$\phi_{X_{n+1}} = Ez^{X_{n+1}} = E(E(z^{X_{n+1}}|X_n)) =
E(\phi_\xi(z))^m \vline_{m = X_n} = \phi_{X_n}(\phi_\xi(z))$$
\end{proof}

\begin{corollary} \forcenewline
\begin{enumerate}
\item $\phi_{X_n}(z) =
	\underset{n \text{ раз}}{\underbrace{\phi_\xi(\phi_\xi(\dots\phi_\xi(}} z)\dots))$
\item $\phi_{X_{n+1}}(z) = \phi_\xi(\phi_{X_n}(z))$
\begin{proof}
Применяем индуктивно лемму \ref{lem:phi_xi_n}:
$$\phi_{X_{n+1}} = 
	\underset{n+1 \text{ раз}}{\underbrace{\phi_\xi(\phi_\xi(\dots\phi_\xi(}} z)\dots)) =
	\phi_\xi(\phi_{X_n}(z))$$
\end{proof}
\end{enumerate}
\end{corollary}

\subsection{Вероятность вырождения процесса} \forcenewline
Положим $q_n = P(X_n = 0)$, $q = P(\text{процесс выродился}) = P(\exists n: X_n = 0)$
\begin{lem}
$$q_n \leq q_{n+1} \text{  и  } q = \mylim_n q_n$$
\end{lem}
\begin{proof}
$\{X_n = 0\} \subset \{X_{n+1} = 0\} \Rightarrow q_n \leq q_{n+1}$

Но $P(\exists n: X_n = 0) = P\pars{\mybigcup_n \{X_n = 0\}} =$
|по непрерывности вероятностной меры| $ = \mylim_n P(X_n = 0) = \mylim_n q_n$.
\end{proof}

\begin{lem}
Вероятность вырождения $q$ является решением уравнения
$$s = \phi_\xi(s)$$
\end{lem}
\begin{proof}
$$
q \leftarrow q_n = P(X_n = 0) = \phi_{X_n}(0) = \phi_\xi(\phi_{X_{n-1}}(0)) = \phi_\xi(q_{n-1})
\to \phi_\xi(q)
$$
\end{proof}

Вопрос: что делать, если на $[0,1]$ решений несколько?

Всегда есть решение $s = 1$.

\begin{theorem}[о вероятности вырождения]

Пусть $\xi \neq 1$ п.н. Пусть $\mu = E\xi$ (м.б. $\mu = +\infty$). Тогда
\begin{enumerate}
\item Если $\mu \leq 1$, то уравнение
$$s = \phi_\xi(s)$$
имеет только одно решение $s = 1$ на $[0, 1]$. Тогда $q = 1$.
\item Если $\mu > 1$, то уравнение
$$s = \phi_\xi(s)$$
имеет единственное решение $s_0 \in [0, 1)$. В этом случае $q = s_0$.
\end{enumerate}
\end{theorem}
\begin{proof} \forcenewline
\begin{enumerate}
\item Рассмотрим производную $\phi'_\xi(s) = \mysum_{k=1}^{\infty}k s^{k-1}P(\xi = k)$
для $s \in [0, 1]$. Заметим, что эта функция строго возрастает (поскольку каждое слагаемое
строго возрастает) и положительна (поскольку есть хоть одна ненулевая вероятность).
Действительно, если $\phi_\xi(s) = 0$ для $s > 0 \Rightarrow P(\xi = k) = 0 \; \forall k \geq 1$.
Но тогда $P(\xi = 0) = 1$ и $q = 1$.

Далее считаем производную положительной.

Для $s \in (0, 1)$:
$$
1 - \phi_\xi(s) = \phi'_\xi(\theta)(1-s)
$$
где $\theta = \theta(s) \in (s, 1)$
Но $\phi'_\xi(\theta) < \phi'_\xi(1) = \mu = 1$, т.к. производная строго возрастает.
$$\Rightarrow 1 - \phi_\xi(s) < 1-s \text{ при } s \in [0, 1) \Rightarrow s < \phi_\xi(s)$$
Решений, отличных от $1$, нет.

График в этом случае выглядит так:
--- рис.7 ---

\item Рассмотрим 
$$\phi''_\xi(s) = \mysum_{k=2}^{\infty} k(k-1)s^{k-2}P(\xi = k) \text{ для } s \in [0, 1)$$
Функция строго возрастает и положительна на $(0, 1)$.

Действительно, если вдруг
$\phi''_\xi(s) = 0 \Rightarrow \forall k \geq 2 \; P(\xi=k) = 0 \Rightarrow
\xi < 1 \text{ п.н. } \Rightarrow E\xi = \mu \leq 1$, что противоречит условию.

Теперь считаем, что $\phi'_\xi(s)$ строго возрастает на $[0, 1)$. 
$$\Rightarrow 1 - \phi'_\xi(s) \text{ меняет знак на } [0, 1) \text { не более одного раза }$$

$$1 - \phi'_\xi(0) = 1 - P(\xi = 1) >  0$$
$$1 - \phi'_\xi(1) = 1 - \mu < 0$$
$$\Rightarrow 1 - \phi'_\xi(s) \text{ меняет знак ровно один раз}$$

График выглядит так:
--- рис.8 ---

Пусть $\phi'_\xi(s_1) = 1$. Что можно сказать про $s - \phi_\xi(s)$?
При $s < s_1$ возрастает, при $s > s_1$ возрастает.

Если $\phi_\xi(0) = P(\xi = 0) = 0$, то $s - \phi_\xi(s) \vline_{s=0} = 0$ --- рис.9 ---,
то есть ровно один корень $s = 0 \in [0, 1)$. Ясно, что в этом случае $\boxed{q = 0}$.

Если $\phi_\xi(0) > 0$, то $s - \phi_\xi(s) \vline_{s = 0} < 0$
$\Rightarrow \exists s_0$~-- единственное решение уравнения на $[0, 1)$

Заметим, что при $s < s_0 \; s < \phi_\xi(s)$, а при $s > s_0 \; s > \phi_\xi(s)$.

Но $q_n = \phi_\xi(q_{n-1}) \leq$ |т.к. $q_n \geq q_{n-1}$| $ \leq \phi_\xi(q_n)
\Rightarrow q_n \notin(s_0, 1)$.

Если $q_n = 1$, то $q_{n-1} = 1$, т.к. $\phi_\xi(s) = 1 \Leftrightarrow s = 1$.
По индукции получаем, что $q_0 = 1$. Но $q_0 = 0$, т.к. в нулевой момент времени
всегда есть одна частица.
Значит, $q \in [0, s_0]$ как предел $q_n \Rightarrow q = s_0$.

График: --- рис 11 --- 
\end{enumerate}
\end{proof}

\underline{Вывод:} вероятность вырождения~-- это наименьший корень уравнения $s = \phi_\xi(s)$
из отрезка $[0, 1]$.

\underline{Интерпретация:} если среднее число потомков меньше $1$, то процесс обречен на вымирание.
Иначе есть ненулевая вероятность того, что мы будем живы до бесконечности.

\section{Конечномерные распределения случайных процесов}
Пусть $(X_t, t \in T)$~-- случайный процесс на $(\Omega, F, P)$, и $X_t$  принимает
значения в $(S_t, \myB_t)$.

\begin{definition}
Множество $S = \myprod_{t\in T} S_t $ называется \emph{пространством траекторий случайного процесса}.
\end{definition}
%% поставить нормальный крестик

$$S = \{y = (y(t), t \in t): \forall t \in T \; y(t) \in S_t\}$$

\begin{definition}
Для $\forall t \in T$ и $B_t \in \myB_t$ введем \emph{элементарный цилиндр}
с основанием $B_t$:
$$C(t, B_t) = \{y \in S: y(t) \in B_t\}$$
\underline{Смысл:} это все траектории, проходящие через $B_t$  в момент времени $t$.
--- рис.12 --- \forcenewline
\end{definition}

\begin{definition}
Минимальная $\sigma$-алгебра $\myB_T$, содержащая все эти элементарные цилиндры,
называется \emph{цилиндрической $\sigma$-алгеброй} на $S$.

$$\myB_T = \sigma\{C(t, B_t): t \in T, B_t \in \myB_T\}$$
\end{definition}

$(S, \myB_T)$~-- измеримое пространство.

\begin{lem}
$X = (X_t, t \in T)$ является случайным процессом $\Leftrightarrow X: \Omega \to S$
является измеримым относительно цилиндрической $\sigma$-алгебры $\myB_T$.
\end{lem}

\rightoffset{
\underline{Напоминание}
Критерий измеримости отображения

$X: \Omega \to E, ~ \mathcal{M} \subset \mathcal{E}
\text{ т.ч. }\sigma(\mathcal{M}) = \mathcal{E}$.
Тогда $X$~-- с.в. $\Leftrightarrow \forall B \in \mathcal{M} \; X^{-1}(B) \in \mathcal{F}$
}

\begin{proof}

$(\Rightarrow)$ Для $\forall t X_t$~-- случайный элемент со значениями в $(S_t, \myB_t)$. Рассмотрим 
$\myM$~-- система элементарных цилиндров в $\myB_T$. По определению $\sigma(\myM) = \myB_T$.

$\forall C(t, B_t) \in \myM$ получаем 

$$X^{-1}(C(t, B_t)) = \{\omega:  X(\omega) \in C(t, B_t)\}
= \{\omega: X_t(\omega) \in B_t \} = \{X_t^{-1}(B_t) \in \myF\}$$
т.к. $X_t$~-- случайный элемент.

$(\Leftarrow)$ По критерию измеримости отображения
$$\forall B \in \myB_T \; X^{-1}(B) \in \myF$$
\end{proof}

\begin{remark}
Лемма устанавливает эквивалентное определение случайного процесса как единого случайного
элемента со значениями в пространстве траекторий.
\end{remark}

\begin{definition}
\emph{Распределением} $P_X$ случайного процесса $X =  (X_t, t \in T)$
называется вероятностная мера на $(S, \myB_T)$
т.ч. $\forall \myB \in \myB_T \; P_X(B) =  P(X \in B)$.
\end{definition}

Это определение удобно только в том случае, когда <<время>> конечно.
Для счетного (или тем более континуального) <<времени>> это определение
очень трудно для понимания.

\begin{definition}
Пусть $X = (X_t, t \in T)$~--  случайный процесс, $\forall n \in \myN \;
\forall t_1, \dots, t_n \in T$
пусть $P_{t_1, \dots, t_n}$ обозначает распределение вектора $(X_{t_1}, \dots, X_{t_n})$.
Тогда набор вероятностных мер $\{P_{t_1, \dots, t_n}: n \in \myN, t_i \in T\}$
называется \emph{набором конечномерных распределений} случайного процесса $X$, а сами
$P_{t_1, \dots, t_n}$ называются конечномерными распределениями $X$.
\end{definition}

\rightoffset{
\underline{Напоминание:}
$P_{..}$~-- вер.мера на
$(S_{t_1} \times \dots \times S_{t_n},
\myB_{t_1} \otimes \dots \otimes \myB_{t_n})$, 
определенная по правилу ---2---.
}

\begin{lem}
Пусть $X$ и $Y$~-- случайные процессы c одинаковым временем, имеющие одно и то же пространство
траекторий. Тогда $P_X = P_Y \Leftrightarrow$ совпадают их конечномерные распределения.
\end{lem}

\rightoffset{
\underline{Напоминание:} \forcenewline
Единственность продолжения меры

Пусть $(E, \myE)$~-- измеримое пространство, $P, Q$~-- две вероятностные меры на нем.
Пусть $\myM \subset \myE$~-- $\pi$-система и $\sigma(\myM) =  \myE$. Тогда 
$$P\vline_\myM = Q\vline_\myM \Leftrightarrow P\vline_\myE = Q\vline_\myE$$
}

\begin{proof}
Рассмотрим цилиндры (не элементарные!) в $S$:
$$ \forall b \; \forall t_1 \dots t_n \in T \;\forall \seq{B}{t_1}{t_n}, B_{t_i} \in \myB_{t_i}
\text{ определим }C(t_1, \dots ,t_n, B_{t_1}, \dots, B_{t_n}) = \{y \in S: y(t_i) \in B_{t_i} \forall i = 1..n\}$$
Это пересечения каких-то элементарных цилиндров. Фактически, мы фиксируем значение
случайного процесса в нескольких моментах времени, а не только в одном, как это делалось
в случае элементарного цилиндра.

Пусть $\myM$~-- множество цилиндров.
Заметим, что $\myM$~--  $\pi$-система, причем $\sigma(\myM) = \myB_T$.

$(\Rightarrow)$ Пусть $t_1, \dots, t_n \in T, B_{t_1}, \dots, B_{t_n}, B_{t_i} \in \myB_{t_i}$.
Тогда
$$
\begin{array}{rllll}
P^X_{t_1, \dots, t_n}(B_{t_1} \times \dots \times B_{t_n}) &=&
P( (X_{t_1}, \dots, X_{t_n}) \in B_{t_1} \times \dots \times B_{t_n}) = \\
&&P(X \in C(t_1, \dots, t_n, B_{t_1}, \dots, B_{t_n})) = \\
&&P_X(C(t_1, \dots, t_n, B_{t_1}, \dots, B_{t_n})) = \\
&&P_Y(C(t_1, \dots, t_n, B_{t_1}, \dots, B_{t_n})) = \\
&&P( (Y_{t_1}, \dots, Y_{t_n}) \in B_{t_1} \times \dots \times B_{t_n}) =
P^Y_{t_1, \dots, t_n}(B_{t_1} \times \dots \times B_{t_n})
\end{array}
$$
Доказали, что распределения $X$ и $Y$ совпадают на прямоугольниках.
Но $B_{t_1} \times \dots \times B_{t_n}$~-- порождающая $\pi$-система
для $\myB_{t_1} \otimes \dots \otimes \myB_{t_n} \Rightarrow$
по единственности продолжения меры $P^x_{t_1 \dots t_n} = P^y_{t_1 \dots t_n}$.

$(\Leftarrow)$
$$
\begin{array}{l}
P_X(C(t_1, \dots, t_n, B_{t_1}, \dots, B_{t_n})) = \\
P^X_{t_1 \dots t_n}(B_{t_1} \times \dots \times B_{t_n}) = \\ % ???
P^Y_{t_1 \dots t_n}(B_{t_1} \times \dots \times B_{t_n}) = \\ % ???
P_Y(C(t_1, \dots, t_n, B_{t_1}, \dots, B_{t_n}))
\end{array}
$$

Доказали, что $P_x$ и $P_Y$ совпадают на $\myM$. Но $\myM$~-- порождающая
$\pi$-система для $\myB_T \Rightarrow$ по единствености продолжения меры
$P_X = P_Y$.
\end{proof}

Пусть $P_{t_1 \dots t_n}, n \in \myN, t_1 \dots t_n \in T$~-- конечномерное распределение $(X_t, t \in T)$.

\begin{lem}
Для $\{P_{t_1 \dots t_n}\}$ выполнены условия симметрии $(1)$ и согласованности $(2)$:
\begin{enumerate}
\item $\forall n \; \forall t_1 \dots t_n \in T \; \forall \sigma$~-- перестановки $\{1 .. n\}$ выполнено
$$P_{t_1 \dots t_n}(B_{t_1} \times \dots \times B_{t_n}) =
P_{t_{\sigma_1} \dots t_{\sigma_n}} (B_{t_{\sigma_1}} \times \dots \times B_{t_{\sigma_n}})
$$

\item $\forall n \; \forall t_1 \dots t_n \in T$
$$P_{t_1 \dots t_n}(B_{t_1} \times \dots \times B_{t_{n-1}} \times S_{t_n}) =
P_{t_1 \dots t_{n-1}}(B_{t_1} \times \dots \times B_{t_{n-1}})$$
\end{enumerate}

\begin{proof} \forcenewline
\begin{enumerate}
\item $P_{t_1 \dots t_n}(B_{t_1} \times \dots \times B_{t_n})
= P(X_{t_1} \in B_{t_1}, \dots, X_{t_n} \in B_{t_n})$~-- не зависит от перестановки.
\item $\{X_{t_n} \in S_{t_n}\} =  \Omega$, поэтому это событие ничего не добавляет в пересечение.
\end{enumerate}

Пусть теперь $X$~--  вещественный процесс, т.е. $(S_t, \myB_t) = (\myR, \myB(\myR)) \; \forall t \in T$.
\begin{theorem}[Колмогорова о существовании случайного процесса, б/д]
Пусть $T$~-- некоторое множество, $\forall n \in \myN \; \forall t_1 \dots t_n \in T$ задана
вероятностная мера
$P_{t_1 \dots t_n}$ на $(\myR^n, \myB{\myR^n})$,
причем для системы $\{P_{t_1, \dots t_n}\}$  выполнены условия симметрии и согласованности.
Тогда $\exists$ вероятностное пространство $(\Omega, \myF, P)$
и вещественный случайный процесс $(X_t, t \in T)$
т.ч. $\{P_{t_1 \dots t_n}\}$~-- это его конечномерные распределения.
\end{theorem}
\end{proof}
\end{lem}

% 26 sep 2013
\begin{theorem}[условия симметрии и согласованности для характеристических функций]
Пусть $T$~-- некоторое множество, $\forall t_1, \dots, t_n \in T$ задана вер. мера
$P_{t_1, \dots, t_n}$ на $(\myR^n, \myB(\myR^n)$ с х.ф. $\phi_{t_1, \dots, t_n}$.
Тогда $\{P_{t_1, \dots, t_n}, n \in \myN, t_i \in T\}$ обладают условиями симм. и согл. $\Leftrightarrow$
выполнены условия симметрии и согласованности для х.ф.:
\begin{enumerate}
\item $\phi_{t_1, \dots, t_n}(\lambda_1, \dots, \lambda_n) =
\phi_{t_{\sigma_1}, \dots, t_{\sigma_n}}(\lambda_{\sigma_1}, \dots, \lambda_{\sigma_n})
$
\item $\phi_{t_1, \dots, t_n}(\lambda_1, \dots, \lambda_{n-1}, 0) =
\phi_{t_1, \dots, t_{n-1}}(\lambda_1, \dots, \lambda_n)$
\end{enumerate}
\end{theorem}

\begin{corollary}
Пусть $T \subset \myR$, $\forall n \in \myN \; \forall t_1 < \dots < t_n \in T$
задана х.ф. $\phi_{t_1, \dots, t_n}$ в $\myR^n$. Тогда

$$\exists (\Omega, \myF, P) \text{ и случайный процесс } (X_t, t \in T) \text{ т.ч. }
\phi_{t_1, \dots, t_n} \text{ -- х.ф. } (X_{t_1}, \dots, X_{t_n})$$
$$\Leftrightarrow$$
$$\forall m \; \phi_{t_1, \dots, t_n}(\lambda_1, \dots, \lambda_n) \vline_{\lambda_m = 0} = %\big
\phi_{t_1, \dots, t_{m-1}, t_{m+1}, \dots, t_n}(\lambda_1, \dots, \lambda_{m-1}, \lambda_{m+1}, \dots, \lambda_n)$$
\end{corollary}

\begin{proof} \forcenewline
$(\Rightarrow)$ Очевидно из теоремы Колмогорова и теоремы об условиях симметрии и согласованности
для хар. функций

$(\Leftarrow)$ Пусть $s_1, \dots, s_n \in T, s_i \neq s_j$, рассмотрим $t_1 < \dots < t_n$ т.ч.
$t_i = s_{\sigma_i}$ для некоторой перестановки $\sigma$.

Зададим
$$\phi_{s_1, \dots, s_n}(\lambda_1, \dots, \lambda_n) :=
\phi_{t_1, \dots, t_n}(\lambda_{\sigma_1}, \dots, \lambda_{\sigma_n})$$

Проверим условия симметрии и согласованности для таких хар. функций.

Условие симметрии дано по построению. Проверим согласованность.

$$
\begin{array}{l}
\phi_{s_1, \dots, s_n}(\lambda_1, \dots, \lambda_n) \vline_{\lambda_n=0} = \\
 |s_n = t_m \Rightarrow \lambda_n = \lambda_{\sigma_m}| = \\
\phi_{t_1, \dots, t_n}(\lambda_{\sigma_1}, \dots, \lambda_{\sigma_n}) \vline_{\lambda_{\sigma_m} = 0} =\\
|\text{условие следствия}| =\\
\phi_{t_1, \dots, t_{m-1}, t_{m+1}, \dots, t_n}
(\lambda_{\sigma_1}, \dots, \lambda_{\sigma_{m-1}}, \lambda_{\sigma_{m+1}}, \dots, \lambda_{\sigma_n}) = \\
|\text{по построению}| = \\
\phi_{s_1, \dots, s_{n-1}}(\lambda_1, \dots, \lambda_{n-1})
\end{array}
$$

Условия симметрии и согласованности проверяются аналогично,
если есть совпадающие $s_i = s_j$.  По теореме Колмогорова искомый процесс существует.
\end{proof}

\section{Процессы с независимыми приращениями}
\begin{definition}
Пусть $(X_t, t \geq 0)$~-- действительный процесс. Он называется \emph{процессом с
независимыми приращениями}, если $\forall n \in \myN \; \forall \; 0 \leq t_1 < \dots < t_n$
случайные величины $X_{t_1}, X_{t_2 - t_1}, \dots, X_{t_n - t_{n-1}}$ независимы в совокупности.
\end{definition}

\begin{theorem}[о существовании процессов с независимыми приращениями]
Пусть $Q_0$~-- вер. мера на $(\myR, \myB(\myR))$ с х.ф. $\phi_0$,
и $\forall s,t$ задана вер. мера $Q_{s,t}$ на $(\myR, \myB(\myR))$ с х.ф. $\phi_{s,t}$.
Тогда процесс с независимыми приращениями
$$(X_t, t \geq 0) \text{ т.ч. } X_0 \overset{d}{=} Q_0, X_t - X_s \overset{d}{=} Q_{s,t}, s < t$$
существует тогда и только тогда, когда
$$\forall \; 0 \leq s < u < t \; \phi_{s,t}(\tau) = \phi_{s,u}(\tau) \phi_{u,t}(\tau)$$
\end{theorem}
\begin{proof} \forcenewline
$(\Rightarrow)$ $\exists X_t \Rightarrow \forall \; s < u < t X_t-X_u$ и 
$X_u-X_s$ независимы $\Rightarrow \phi_{s,t}(\tau) = \phi_{X_t-X_s}(\tau)=
\phi_{X_t-X_u+X_u-X_s}(\tau) =
\phi_{X_t-X_u}(\tau) \phi_{X_u-X_s}(\tau) =
\phi_{u,t}(\tau) \phi_{s,u}(\tau)$.

$(\Leftarrow)$ Пусть $X_t$ существует. Тогда $\forall 0 < t_1 < \dots < t_n$
случайные величины. $(X_tn-X_tn-1, ..., X_t1-X_0, X_0)$ независимы. Обозначим этот вектор как $\xi$.
% 
% $\phi_\xi(\ln .. l0) = \phi_{X_tn-X_t{n-1}}(l1) * ... * (\phi_{X_0})(l0)} =
% \phi_{t_n,t_{n-1}}(...)$
% 
% Рассмотрим $\eta = (X_tn, ..., X_0)^t$. Ясно,  что $\eta = A\xi$, где
% 
% $A = $ ---матр.1---
% 
% $\Rightarrow \phi_\eta(\lambda vector) = Ee^{i <\eta, \lambda v>} = $
% записи на бумажке
% 
% Теперь забудем про то, что процесс существует. 
% $\forall n \forall t_1 < ... t_n$ зададим $\phi_t1..tn$ по формуле $(*)$.
% 
% Проверим условия следствия: пусть $l_m = 0$.
% ---опять записи на бумажке---
% 
% Смотрим на две подчеркнутые штуки.
% 
% По условию $\phi_tm-1 tm(\tau) \phi_tm tm+1(\tau) = \phi_tm-1 tm+1(\tau)$.
% 
% --- тут то, что после равно в кружочке --
% 
% 
% Согласно следствию, $\exists (X_t, t \geq 0)$ т.ч.
% $\phi_tn .. t1$~-- х.ф. $(X_tn .. X_t1)$. По построению $\phi_t1..tn$
% это означает, что $X_t$ имеет независимые приращения, $X_t - X_s \overset{d}{=}
% Q_{s,t}, t > s$, и $X_0 \overset{d}{=} Q_0$.
\end{proof}

% 
% \begin{remark}
% \begin{itemize}
% \item Процесс с независимыми приращениями с дискретным временем~-- это случайное блуждание.ы
% \item Если $\phi_{\xi+\eta}(t) = \phi_\xi(t) \phi_\eta(t)$, то это не значит,
% что $\xi$ и $\eta$ независимы.
% \end{itemize}
% \end{remark}
% 
% \subsection{Пуассоновский процесс}
% \begin{definition}
% Процесс $N_t, t \geq 0$ называется \emph{пуассоновским процессом интенсовности $\lambda > 0$},
% если
% \begin{enumerate}
% \item $N_0 = 0$ п.н.
% \item $N_t$ имеет независимые приращения
% \item $N_t - N_s \tilde Pois(\lambda(t-s)), t > s \geq 0$
% \end{enumerate}
% \end{definition}
% 
% \begin{statement}
% Пуассоновский процесс существует.
% \end{statement}
% \begin{proof}
% Пусть $\phi_s,t$~-- х.ф. $Pois(\lambda(t-s))$. Тогда
% $$\phi_{s,t}(\tau) = ... = e^{\lambda(t-s)(e^{i\tau} - 1)}$$
% 
% Отсюда видно, что $\forall s < u < t$
% $$\phi_{s,u}(\tau) \phi_{u,t}(\tau) = \phi_{s,t}(\tau)$$
% Значит, по критерию существования пуассоновский процесс существует.
% \end{proof}
% 
% \paragraph{Свойства траекторий}
% \begin{enumerate}
% \item Целочисленные: $N_t \in \myZ_+, t \geq 0$
% \item Неубывающие: $N_t - N_s \geq 0, t > s$
% \end{enumerate}
% Вопросы:
% \begin{enumerate}
% \item размер скачков?
% \item распределение моментов скачков
% \end{enumerate}
% 
% \begin{theorem}[явная конструкция пуассоновского процесса]
% Пусть $\{\xi_n, n \in \myN\}$~-- нез. $\tilde Exp(\lambda)$, $S_n =  \xi_1 + \dots + \xi_n$.
% Тогда процесс восстановления
% $$X_t = \sup\{n: S_n \leq t\}$$
% является пуассоновским процессом интенсивности $\lambda$.
% \end{theorem}
% \begin{proof}
% Пусть $X_0 = 0$.
% Рассмотрим вектор $(S_n, \dots, S_1)$. Он имеет плотность:
% $$P_{S_n..S_1}(x_1..x_1) = P_{\xi_n .. \xi_1}(X_n-x_{n-1}, ..., X_2-X_1, X_1) =
% |\xi_i \tilde Exp(\lambda)| = \prod_{k=1}^{n} \lambda e^{-(x_k-x_k-1)\lambda} \ind{X_k > X_k-1} =
% \lambda^n e^{-\lambda X_n} \ind{x_n > \dots > x_1 > 0}$$
% 
% Зафиксируем $0 < t_1 < ... < t_n$ и $k_n \geq ... \geq k_1, k_i \in \myZ_+$. Рассмотрим
% $$P(X_tn - X_tn-1 = k_n-k_n-1, ..., X_t1 = k1)$$
% 
% Что такое X_tn - X_tn-1? Это количество скачков, которые совершает процесс на отрезке $tn-1, tn$.
% 
% $$ = P(S_k1 < t1, \{S_k1+1, ... S_k2\} \in (t1, t2], ..., \{S_{k_{n-1}+1}, .., S_kn\}
% \in (t_n-1, tn], S_{k_n+1} > t_n) =$$
% |подставляем плотность $p_{s1..s_{k_n + 1}} =$|
% интегрируем в области, написанной на бумажке.
% 
% дальше много всего на бумажке
% 
% Следовательно, $X_t$ имеет независимые приращения, и эти приращения пуассоновские:
% 
% $$X_{t_j} - X_{t_{j-1}} \tilde Pois(\lambda(t_j - t_{j-1}))$$
% 
% Значит, $X_t$~-- это пуассоновский процесс.
% \end{proof}
% \begin{corollary}
% Пусть $N_t$~-- пуассоновский процесс интенсивности $\lambda > 0$, $Y_n$~-- 
% момент $n$-того скачка. Тогда
% \begin{enumerate}
% \item П.н. все скачки $N_t$ имеют размер $1$
% \item $Y_n \tilde \Gamma(\lambda, n)$, записать плотность
% \item $Y_n-Y_{n-1}, ..., Y_1$~-- нез. $Exp(\lambda)$.
% \end{enumerate}
% \begin{proof}
% \begin{enumerate}
% \item Пусть $X_t$~-- явная конструкция из теоремы.
% 
% $P($У $N_t$ есть скачок размера $\geq 2) = P($ у $X_t$ есть скачок размера $\geq 2) =
% P(\exists n: S_n = S_{n+1}) = P(\exists n; \xi_n = 0) = 0$, т.к. $\xi_n$ абсолютно непрерывны.
% 
% \item 2 и 3 выполнены для $X_t$ по построению $\Rightarrow$ верны и для $N_t$.
% \end{enumerate}
% \end{proof}
% \end{corollary}
% \end{document}

\section{Гауссовские случайные процессы}

\rightoffset{
\begin{definition}
Случайный вектор $\xi =  \seq{\xi}{1}{n}$ называется гауссовским, если его х.ф. имеет вид

$$\phi_\xi(t) = e^{i<a,t> - \frac{1}{2}<\Sigma t, t>}$$

где $a \in \myR^n$, а $\Sigma \in Mat(n\times n)$~-- симметрическая и неотрицательно определенная.
В этом случае пишут $\xi \tilde N(a, \Sigma)$.
\end{definition}

\begin{theorem}[три эквивалентных определения]\forcenewline
\begin{enumerate}
\item Вектор $(\seq{\xi}{1}{n})$ гауссовский
\item $\xi = A\eta + b$ п.н., где $A \in Mat(n\times m)$, $b \in \myR^n$,
$\eta = (\seq{\eta}{1}{n}$, $\eta_i$~-- нез. $N(0,1)$
\item $\forall \tau \in \myR^n <\tau, \xi>$ имеет одномерное нормальное распределение.
\end{enumerate}

\paragraph*{Свойства гаусс. векторов}
\begin{enumerate}
\item Смысл параметров: если $\xi \tilde N(a, \Sigma)$, то $a = E\xi$, $\Sigma = D\xi$~--
матрица ковариаций.
\item Если $\xi$ гауссовский, то $A\xi$ гауссовский для всех матриц соответствуещего размера
(т.е. линейное преобразование гауссовского вектора также является гауссовским вектором).
\item Если $\xi = (\seq{\xi}1n) \tilde N(a, \Sigma)$, то $\seq{\xi}1n$ нез. в совокупности
$\Leftrightarrow \Sigma$ диагональна $\Leftrightarrow \seq{\xi}1n$ некоррелированы.
\end{enumerate}
\end{theorem}
}

\begin{definition}
Действительный случайный процесс $(X_t, t \in T)$ называется \emph{гауссовским}, если
все его конечномерные распределения гауссовские:

$$\forall n \; \forall \seq{t}1n \in T \text{ вектор } (\seq{X}{t_1}{t_n}) \text{ гауссовский }$$
\end{definition}

\begin{definition}
Процесс $(X_t, t \in T)$ называется \emph{$L^2$-процессом}, если $\forall t \in T E|X_t^2| < +\infty$.

Функция $a(t) = EX_t$ называется \emph{функцией среднего} процесса $X_t$.

Функция $R(s,t) = cov(X_s, X_t)$ называется \emph{ковариационной функцией} процесса $X_t$.

Функция $K(s,t) = EX_sX_t$ называется \emph{корреляционной функцией} процесса $X_t$.
\end{definition}

\begin{remark}
распределение гауссовского вектора однозначно определяется матожиданием и матрицей ковариаций
$\Rightarrow$ конечномерные распределения гауссовского процесса определяются функцией среднего
и ковариационной функцией.
\end{remark}

\begin{definition}
Функция $f(x, y), x, y \in t$ называется \emph{неотрицательно определенной} на $T\times T$, если

$$\forall n \; \forall \seq t 1 n \in T \; \forall \seq x 1 n \in \myR
\sum_{i, j = 1}^{n} f(t_i, t_j) x_i x_j \geq 0$$
\end{definition}

\begin{lem}
Ковариационная и корреляционная функции случайного процесса симметричны и неотрицательно
определены.
\end{lem}
\begin{proof}
Пусть $X_t$~-- $L^2$-процесс, $K(s, t)$~-- его корреляционная функция. Тогда
$\forall n \; \forall \seq t 1 n \in T \; \forall \seq x 1 n \in \myR$

$$\sum_{i, j = 1}^{n} f(t_i, t_j) x_i x_j \geq 0 = \sum_{i,j=1}^n(EX_{t_i}X_{t_j})x_i x_j =
E(\sum_{i,j=1}^n x_iX_{t_i} x_j X_{t_j}) = E(\sum_{i=1}^n x_i X_{t_i})^2 \geq 0$$

$\Rightarrow K(s,t)$ неотрицательно определена.

Теперь заметим, что  $R(s,t) = cov(X_s, X_t)$~-- это корреляционная функция для  $Y_t = X_t - EX_t$
$\Rightarrow$ она тоже неотрицательно определена.

Их симметричность очевидна.
\end{proof}

\begin{theorem}[о существовании гауссовских процессов]
Пусть $T$~-- некоторое множество, на нем задана функция $a(t)$, и $R(s, t)$~-- симметричная и 
неотрицательно определенная функция на $T \times T$. Тогда существует вероятностное пространство
$(\Omega, \myF, P)$ и гауссовский процесс $(X_t, t \in t)$ т.ч. $a(t) = EX_t$ и
$R(s,t) =  cov(X_s, X_t)$.
\end{theorem}
\begin{proof}
Для $n \in \myN$, $\seq t 1 n  \in T$  рассмотрим вектор $\seq{a}{t_1}{t_n} =
(a(t_1), \dots, a(t_n))$, $\Sigma_{\seq t 1 n} = ||R(t_i, t_j)||^n_{i,j=1}$.

Тогда $\Sigma_{\seq t 1 n}$ неотрицательно определена. Рассмотрим х.ф.

$$\phi_{\seq t 1 n}(\seq \lambda 1 n) = e^{i<a_{\seq t 1 n}, \lambda} -
\frac12<\Sigma{\seq t 1 n}\lambda, \lambda>$$

Легко видеть, что такой набор х.ф. обладает свойствами симметрии и согласованности:

$$<a_{\seq t 1 n}, \lambda> = \sum_{k=1}^n a(t_k)(\lambda_k) = |\forall \sigma| =
\sum_{k=1}^n a(t_{\sigma(k)})(\lambda_{\sigma(k)})$$

$$<a_{\seq t 1 n}, (\seq \lambda 1 n)> \vline_{\lambda_n = 0} =
<a_{\seq t 1 {n-1}}, (\seq \lambda 1 {n-1})$$

Для $R$ аналогично (написать!).

По теореме Колмогрова это означает, что $\exists (X_t, t \in T)$ т.ч.
$\phi_{\seq t 1 n}$~-- х.ф. $(X_{t_1}, \dots, X_{t_n}) \Rightarrow X_t$~-- гауссовский 
процесс и $EX_t = a(t)$, $cov(X_s, X_t) = R(s,t)$.
\end{proof}

\subsection{Процесс броуновского движения (винеровский процесс)}

\begin{definition}
Случайный процесс $(W_t, t \in 0)$ называется \emph{винеровским}, если
\begin{enumerate}
\item $W_0 = 0$ п.н.
\item $W_t$ имеет независимые приращения
\item $W_t - W_t \tilde N(0, t-s)$, $t \geq s$
\end{enumerate}
\end{definition}

\begin{statement}
Винеровский процесс существует.
\end{statement}
\begin{proof}
По критерию существования процессов с независимыми приращениями достаточно проверить, что
$\forall s \leq u \leq t$ (опускаем аргумент у х.ф.)

$$\phi_{W_t W_s}  = \phi_{W_t-W_u} \phi_{W_u-W_s}$$

Но т.к. $W_t-W_s \sim N(0, t-s)$

$$\phi_{W_t-W_s}(\tau) = e^{-\frac12 \tau^2(t-s)}$$

Очевидно, свойство выполнено и процесс существует.
\end{proof}

\begin{theorem}[эквивалентное определение винеровского процесса]
Процесс $(W_t, t \geq 0)$  является винеровским $\Leftrightarrow$
\begin{enumerate}
\item $W_t$ гауссовский
\item $\forall t \geq 0 \; EW_t = 0$
\item $cov(W_s, W_t) = min(s, t)$
\end{enumerate}
\end{theorem}
\begin{proof}
$(\Rightarrow)$ $W_t \sim N(0, t) \Rightarrow EW_t = 0$. Посчитаем ковариационную функцию:

$$cov(W_s, W_t) = |t > s| = cov(W_s, W_t - W_s + W_s) = cov(W_s, W_t-W_s) + cov(W_s, W_s)
= 0 + DW_s = s = min(s, t)$$.

Пусть $0 \geq t_1 \geq \dots \geq t_n$, $\xi = (\seq{W}{t_1}{t_n})$.
Вектор $\eta = (W_{t_1}, W_{t_2} - W_{t_1}, \dots, W_n - W_{n-1})$ имеет независимые нормальные
компоненты $\Rightarrow \eta$~-- гауссовский вектор.  Очевидно, $\xi = A\eta$ (выписать A!), значит,
$\xi$ также является гауссовским $\Rightarrow W_t$~-- гауссовский процесс.

$(\Leftarrow)$ Почему такой процесс существует? По теореме достаточно проверить, что $min(s, t)$~--
неотрицательно определенная функция.

Для этого можно заметить, что $min(s,t)$~-- это ковариационная функция для пуассоновского процесса
интенсивности $1$. Значит, она неотрицательно определена.

$$EW_t = 0, DW_t = min(t, t) = t \Rightarrow DW_0 = 0, EW_0 = 0 \Rightarrow W_0 = 0 \text{ п.н.}$$

Пусть $0 \leq t_1 < \dots < t_n$ фикс. Тогда $(\seq{W}{t_1}{t_n})$~-- гаусс. вектор
$\Rightarrow \xi = (W_{t_n} - W_{t_{n-1}}, \dots, W_{t_2} - W_{t_1}, W_{t_1})$~-- тоже гауссовский
как линейное преобразование гауссовского вектора. Значит, для независимости компонент $\xi$
достаточно проверить, что они некоррелированны.

Пусть $i > j, t_0 = 0$.

$$cov(W_{t_i} - W_{t_{i-1}}, W_{t_j} - W_{t_{j-1}}) =$$
$$cov(W_{t_i}, W_{t_j}) - cov(W_{t_{i-1}}, W_{t_j}) -
cov(W_.... =$$
$$t_j - t_j - t_{j-1} + t_{j-1} = 0$$

$\Rightarrow W_t$ имеет независимые приращения.

$W_t - W_s \sim N(a, \sigma^2)$.

$$a = E(W_t-W_s) = 0$$
$$\sigma^2 = D(W_t-W_s) = cov(W_t-W_s, W_t-W_s) =
cov(W_t, W_t) + cov(W_s, W_s) - 2cov(W_t,W_s) = |t > s| = t - 2s + s = t-s$$
\end{proof}

\subsection{Непрерывность траекторий $W_t$}

\begin{definition}
Процесс $(Y_t, t \in T)$ называется \emph{модификацией} процесса $(X_t, t \in T)$, если
$\forall t \in T$

$$P(Y_t = X_t) = 1$$
\end{definition}
\begin{theorem}[Колмогорова о существании непрерывной модификации, б/д]
Пусть процесс $(X_t, t \in [a,b])$ таков, что для некоторых $C, \alpha, \eps > 0$ выполнено:

$$\forall t, s \in [a,b] \; E|X_t-X_s|^\alpha \leq C|t-s|^{1+\eps}$$

Тогда у $X_t$ существует модификация $Y_t$, все траектории которой непрерывны.
\end{theorem}
\begin{corollary}
У $W_t, t \geq 0$ существует непрерывная модификация.
\end{corollary}
\begin{proof}
Т.к. $W_t-W_s \sim N(0, |t-s|) \Rightarrow E(W_t-W_s)^4 = 3(|t-s|)^2 
\Rightarrow$ у $W_t$ существует непрерывная модификация на любом конечном отрезке.

Пусть $W_t^{(n)}$~-- непрерывная модификация $W_t$ на отрезке $[n, n+1]$, $n \in \myZ_+$.
Рассмотрим процесс 

$$X_t(\omega) = \{W_t^{(n)}(\omega, t \in [n, n+1)\}$$

Разрывы траекторий $X_t$ возможны только в целых точках времени, когда $W_{n+1}^{(n)}(\omega) \neq
W_{n+1}^{(n+1)}(\omega)$. Но $W_t^{(n)}$ и $W_t^{(n+1})$~-- модификации $W_t$ $\Rightarrow$

$$P(W_{n+1}^{(n)} = W_{n+1}) = 1 = P(W_{n+1}^{(n+1)} = W_{n+1})$$
$$\Rightarrow P(\exists n: W_{n+1}^{(n)} \neq W_{n+1}^{(n+1)}) = 0$$

Теперь рассмотрим 

$$\tilde{X_t}(\omega) = 
X_t(\omega), \text{ если } \forall n \; W_{n+1}^{(n)}(\omega) = W_{n+1}^{(n+1)}(\omega)
0, \text{ иначе }
$$

Это и будет искомая непрерывная модификация.
\end{proof}

\begin{remark}
Условие $\eps > 0$ в теореме Колмогорова существенно.
\end{remark}
\begin{proof}
Пусть $(N_t, t \geq 0)$~-- пуассоновский процесс. Тогда
$$E(N_t-N_t) = \lambda|t-s|$$

$\Rightarrow N_t$ удовлетворяет условию теоремы Колмогорова с $\eps = 0$. Но траектории
$N_t$ разрывны почти наверное на всем $\myR_+$ и разрывны с положительной вероятностью на любом конечном отрезке.
\end{proof}

\begin{remark}
Всюду далее, где это необходимо, считаем, что нам задана непрерывная модификация винеровского
процесса.
\end{remark}

\begin{theorem}[Пэли, Зигмунд, Винер, б/д]
С вероятностью $1$ траектория винеровского процесса не дифференцируема ни в одной точке $\myR_+$.
\end{theorem}
\begin{theorem}[Закон повторного логарифма]
$$P \pars{ \limsup_{t \to +\infty} \frac{W_t}{\sqrt{2t \ln \ln t}} = 1 } = 1$$

$$ \left| \limsup_{t \to +\infty} = \lim_{t\to +\infty} \sup_{s \geq t} f(s) \right| $$
\end{theorem}
\begin{corollary}
$$P \pars{ \liminf_{t \to +\infty} \frac{W_t}{\sqrt{2t \ln \ln t}} = -1 } = 1$$
\end{corollary}
\begin{proof}
Рассмотрим $Y_t = -W_t$~-- тоже винеровский процесс.
\end{proof}
\end{document}