\section{Ветвящиеся процессы Гальтона-Ватсона}
Физическая модель: --- рис.6 ---
В каждый следующий момент времени каждая частица распадается на некоторое случайное число
таких же частиц.

Мат. модель:
Пусть $\xi$~-- случайная величина со значениями в $\myZ_+$.
$\{\xi_k^{(n)}, k, n \in \myN\}$~-- независимые случайные величины с тем
же распределением, что и $\xi$. Положим

$$X_0 = 1, X_1 = \xi_1^{(1)}, X_n = \mysum_{k=1}^{X_{n-1}} \xi_k^{(n)}$$

\begin{definition}
$\{X_n, n \in \myZ_+\}$~-- \emph{ветвящийся процесс Гальтона-Ватсона}, построенный по с.в. $\xi$.
\end{definition}

\begin{itemize}
\item $X_n$~-- число частиц в $n$-м поколении
\item $\xi_k^{(n)}$~-- число потомков $k$-й частицы в $n-1$-м поколении
\end{itemize}

\underline{Вопрос}: какова вероятность вырождения процесса?

\subsection{Производящие функции}
\begin{definition}
Пусть $\xi$~-- случайная величина. Тогда ее \emph{производящей функцией} называется
$$\phi_\xi(z) = Ez^\xi, z \in \myR$$
\end{definition}

\paragraph{Свойства производящих функций}

\begin{enumerate}
\item $\phi_\xi(1) = 1$
\item $\phi'_\xi(1) = E\xi$
\item Если $\xi$ и $\eta$ независимы, то $\phi_{\xi + \eta}(z) = \phi_\xi(z) \phi_\eta(z)$
\end{enumerate}

Если $\xi$ принимает значения в $\myZ_+$, то введем $p_k = P(\xi = k), k \in \myZ_+$.

\begin{enumerate}
\setcounter{enumi}{3}
\item $\phi_\xi(z) = \mysum_{k=0}^{\infty}z^k p_k$
\item $\phi_\xi(0) = p_0$
\item $p_k = \frac{f_\xi^{(k)}(z)}{k!}$
\item Ряд для $\phi_\xi(z)$ сходится абсолютно и равномерно в области $\{|z| \leq 1\}$
\item $\phi_\xi(z)$ непрерывно дифференцируема бесконечное число раз в области $\{|z| < 1\}$
\end{enumerate}

Пусть далее $\{X_n, n \in \myZ_+\}$~-- ветвящийся процесс Г.-В., построенный по $\xi$.

\begin{lem} \label{lem:phi_xi_n}
$$\phi_{X_{n_1}}(z) = \phi_{X_n}(\phi_\xi(z))$$
\end{lem}
\begin{proof}
$\phi_{X_{n+1}}(z) = Ez^{X_{n+1}}$

$$
 E \pars {z^{X_{n+1}} | X_n = m} =
 E \pars {z^{\mysum_{k=1}^{X_n} \xi_k^{(n+1)}} \vline X_n = m} =
 E \pars {z^{\mysum_{k=1}^m \xi_k^{(n+1}} \vline X_n = m} =
 E z^{\mysum_{k=1}^m \xi_k^{(n+1)}} =
 (\phi_\xi(z))^m
$$

Значит,
$$\phi_{X_{n+1}} = Ez^{X_{n+1}} = E(E(z^{X_{n+1}}|X_n)) =
E(\phi_\xi(z))^m \vline_{m = X_n} = \phi_{X_n}(\phi_\xi(z))$$
\end{proof}

\begin{corollary} \forcenewline
\begin{enumerate}
\item $\phi_{X_n}(z) =
	\underset{n \text{ раз}}{\underbrace{\phi_\xi(\phi_\xi(\dots\phi_\xi(}} z)\dots))$
\item $\phi_{X_{n+1}}(z) = \phi_\xi(\phi_{X_n}(z))$
\begin{proof}
Применяем индуктивно лемму \ref{lem:phi_xi_n}:
$$\phi_{X_{n+1}} =
	\underset{n+1 \text{ раз}}{\underbrace{\phi_\xi(\phi_\xi(\dots\phi_\xi(}} z)\dots)) =
	\phi_\xi(\phi_{X_n}(z))$$
\end{proof}
\end{enumerate}
\end{corollary}

\subsection{Вероятность вырождения процесса} \forcenewline
Положим $q_n = P(X_n = 0)$, $q = P(\text{процесс выродился}) = P(\exists n: X_n = 0)$
\begin{lem}
$$q_n \leq q_{n+1} \text{  и  } q = \mylim_n q_n$$
\end{lem}
\begin{proof}
$\{X_n = 0\} \subset \{X_{n+1} = 0\} \Rightarrow q_n \leq q_{n+1}$

Но $P(\exists n: X_n = 0) = P\pars{\mybigcup_n \{X_n = 0\}} =$
|по непрерывности вероятностной меры| $ = \mylim_n P(X_n = 0) = \mylim_n q_n$.
\end{proof}

\begin{lem}
Вероятность вырождения $q$ является решением уравнения
$$s = \phi_\xi(s)$$
\end{lem}
\begin{proof}
$$
q \leftarrow q_n = P(X_n = 0) = \phi_{X_n}(0) = \phi_\xi(\phi_{X_{n-1}}(0)) = \phi_\xi(q_{n-1})
\to \phi_\xi(q)
$$
\end{proof}

Вопрос: что делать, если на $[0,1]$ решений несколько?

Всегда есть решение $s = 1$.

\begin{theorem}[о вероятности вырождения]

Пусть $\xi \neq 1$ п.н. Пусть $\mu = E\xi$ (м.б. $\mu = +\infty$). Тогда
\begin{enumerate}
\item Если $\mu \leq 1$, то уравнение
$$s = \phi_\xi(s)$$
имеет только одно решение $s = 1$ на $[0, 1]$. Тогда $q = 1$.
\item Если $\mu > 1$, то уравнение
$$s = \phi_\xi(s)$$
имеет единственное решение $s_0 \in [0, 1)$. В этом случае $q = s_0$.
\end{enumerate}
\end{theorem}
\begin{proof} \forcenewline
\begin{enumerate}
\item Рассмотрим производную $\phi'_\xi(s) = \mysum_{k=1}^{\infty}k s^{k-1}P(\xi = k)$
для $s \in [0, 1]$. Заметим, что эта функция строго возрастает (поскольку каждое слагаемое
строго возрастает) и положительна (поскольку есть хоть одна ненулевая вероятность).
Действительно, если $\phi_\xi(s) = 0$ для $s > 0 \Rightarrow P(\xi = k) = 0 \; \forall k \geq 1$.
Но тогда $P(\xi = 0) = 1$ и $q = 1$.

Далее считаем производную положительной.

Для $s \in (0, 1)$:
$$
1 - \phi_\xi(s) = \phi'_\xi(\theta)(1-s)
$$
где $\theta = \theta(s) \in (s, 1)$
Но $\phi'_\xi(\theta) < \phi'_\xi(1) = \mu = 1$, т.к. производная строго возрастает.
$$\Rightarrow 1 - \phi_\xi(s) < 1-s \text{ при } s \in [0, 1) \Rightarrow s < \phi_\xi(s)$$
Решений, отличных от $1$, нет.

График в этом случае выглядит так:
--- рис.7 ---

\item Рассмотрим
$$\phi''_\xi(s) = \mysum_{k=2}^{\infty} k(k-1)s^{k-2}P(\xi = k) \text{ для } s \in [0, 1)$$
Функция строго возрастает и положительна на $(0, 1)$.

Действительно, если вдруг
$\phi''_\xi(s) = 0 \Rightarrow \forall k \geq 2 \; P(\xi=k) = 0 \Rightarrow
\xi < 1 \text{ п.н. } \Rightarrow E\xi = \mu \leq 1$, что противоречит условию.

Теперь считаем, что $\phi'_\xi(s)$ строго возрастает на $[0, 1)$.
$$\Rightarrow 1 - \phi'_\xi(s) \text{ меняет знак на } [0, 1) \text { не более одного раза }$$

$$1 - \phi'_\xi(0) = 1 - P(\xi = 1) >  0$$
$$1 - \phi'_\xi(1) = 1 - \mu < 0$$
$$\Rightarrow 1 - \phi'_\xi(s) \text{ меняет знак ровно один раз}$$

График выглядит так:
--- рис.8 ---

Пусть $\phi'_\xi(s_1) = 1$. Что можно сказать про $s - \phi_\xi(s)$?
При $s < s_1$ возрастает, при $s > s_1$ возрастает.

Если $\phi_\xi(0) = P(\xi = 0) = 0$, то $s - \phi_\xi(s) \vline_{s=0} = 0$ --- рис.9 ---,
то есть ровно один корень $s = 0 \in [0, 1)$. Ясно, что в этом случае $\boxed{q = 0}$.

Если $\phi_\xi(0) > 0$, то $s - \phi_\xi(s) \vline_{s = 0} < 0$
$\Rightarrow \exists s_0$~-- единственное решение уравнения на $[0, 1)$

Заметим, что при $s < s_0 \; s < \phi_\xi(s)$, а при $s > s_0 \; s > \phi_\xi(s)$.

Но $q_n = \phi_\xi(q_{n-1}) \leq$ |т.к. $q_n \geq q_{n-1}$| $ \leq \phi_\xi(q_n)
\Rightarrow q_n \notin(s_0, 1)$.

Если $q_n = 1$, то $q_{n-1} = 1$, т.к. $\phi_\xi(s) = 1 \Leftrightarrow s = 1$.
По индукции получаем, что $q_0 = 1$. Но $q_0 = 0$, т.к. в нулевой момент времени
всегда есть одна частица.
Значит, $q \in [0, s_0]$ как предел $q_n \Rightarrow q = s_0$.

График: --- рис 11 ---
\end{enumerate}
\end{proof}

\underline{Вывод:} вероятность вырождения~-- это наименьший корень уравнения $s = \phi_\xi(s)$
из отрезка $[0, 1]$.

\underline{Интерпретация:} если среднее число потомков меньше $1$, то процесс обречен на вымирание.
Иначе есть ненулевая вероятность того, что мы будем живы до бесконечности.