\section{Конечномерные распределения случайных процесов}
Пусть $(X_t, t \in T)$~-- случайный процесс на $(\Omega, F, P)$, и $X_t$  принимает
значения в $(S_t, \myB_t)$.

\begin{definition}
Множество $S = \myprod_{t\in T} S_t $ называется \emph{пространством траекторий случайного процесса}.
\end{definition}
%% поставить нормальный крестик

$$S = \{y = (y(t), t \in t): \forall t \in T \; y(t) \in S_t\}$$

\begin{definition}
Для $\forall t \in T$ и $B_t \in \myB_t$ введем \emph{элементарный цилиндр}
с основанием $B_t$:
$$C(t, B_t) = \{y \in S: y(t) \in B_t\}$$
\underline{Смысл:} это все траектории, проходящие через $B_t$  в момент времени $t$.
--- рис.12 --- \forcenewline
\end{definition}

\begin{definition}
Минимальная $\sigma$-алгебра $\myB_T$, содержащая все эти элементарные цилиндры,
называется \emph{цилиндрической $\sigma$-алгеброй} на $S$.

$$\myB_T = \sigma\{C(t, B_t): t \in T, B_t \in \myB_t\}$$
\end{definition}

$(S, \myB_T)$~-- измеримое пространство.

\begin{lem}
$X = (X_t, t \in T)$ является случайным процессом $\Leftrightarrow X: \Omega \to S$
является измеримым относительно цилиндрической $\sigma$-алгебры $\myB_T$.
\end{lem}

\rightoffset{
\underline{Напоминание}
Критерий измеримости отображения

$X: \Omega \to E, ~ \mathcal{M} \subset \mathcal{E}
\text{ т.ч. }\sigma(\mathcal{M}) = \mathcal{E}$.
Тогда $X$~-- с.в. $\Leftrightarrow \forall B \in \mathcal{M} \; X^{-1}(B) \in \mathcal{F}$
}

\begin{proof}

$(\Rightarrow)$ Для $\forall t X_t$~-- случайный элемент со значениями в $(S_t, \myB_t)$. Рассмотрим
$\myM$~-- система элементарных цилиндров в $\myB_T$. По определению $\sigma(\myM) = \myB_T$.

$\forall C(t, B_t) \in \myM$ получаем

$$X^{-1}(C(t, B_t)) = \{\omega:  X(\omega) \in C(t, B_t)\}
= \{\omega: X_t(\omega) \in B_t \} = X_t^{-1}(B_t) \in \myF$$
т.к. $X_t$~-- случайный элемент.

$(\Leftarrow)$ По критерию измеримости отображения
$$\forall B \in \myB_T \; X^{-1}(B) \in \myF$$
\end{proof}

\begin{remark}
Лемма устанавливает эквивалентное определение случайного процесса как единого случайного
элемента со значениями в пространстве траекторий.
\end{remark}

\begin{definition}
\emph{Распределением} $P_X$ случайного процесса $X =  (X_t, t \in T)$
называется вероятностная мера на $(S, \myB_T)$
т.ч. $\forall B \in \myB_T \; P_X(B) =  P(X \in B)$.
\end{definition}

Это определение удобно только в том случае, когда <<время>> конечно.
Для счетного (или тем более континуального) <<времени>> это определение
очень трудно для понимания.

\begin{definition}
Пусть $X = (X_t, t \in T)$~--  случайный процесс, $\forall n \in \myN \;
\forall t_1, \dots, t_n \in T$
пусть $P_{t_1, \dots, t_n}$ обозначает распределение вектора $(X_{t_1}, \dots, X_{t_n})$.
Тогда набор вероятностных мер $\{P_{t_1, \dots, t_n}: n \in \myN, t_i \in T\}$
называется \emph{набором конечномерных распределений} случайного процесса $X$, а сами
$P_{t_1, \dots, t_n}$ называются конечномерными распределениями $X$.
\end{definition}

\rightoffset{
\underline{Напоминание:}
$P_{..}$~-- вер.мера на
$(S_{t_1} \times \dots \times S_{t_n},
\myB_{t_1} \otimes \dots \otimes \myB_{t_n})$,
определенная по правилу ---2---.
}

\begin{lem}
Пусть $X$ и $Y$~-- случайные процессы c одинаковым временем, имеющие одно и то же пространство
траекторий. Тогда $P_X = P_Y \Leftrightarrow$ совпадают их конечномерные распределения.
\end{lem}

\rightoffset{
\underline{Напоминание:} \forcenewline
Единственность продолжения меры

Пусть $(E, \myE)$~-- измеримое пространство, $P, Q$~-- две вероятностные меры на нем.
Пусть $\myM \subset \myE$~-- $\pi$-система и $\sigma(\myM) =  \myE$. Тогда
$$P\vline_\myM = Q\vline_\myM \Leftrightarrow P\vline_\myE = Q\vline_\myE$$
}

\begin{proof}
Рассмотрим цилиндры (не элементарные!) в $S$:
$$ \forall b \; \forall t_1 \dots t_n \in T \;\forall \seq{B}{t_1}{t_n}, B_{t_i} \in \myB_{t_i}
\text{ определим }C(t_1, \dots ,t_n, B_{t_1}, \dots, B_{t_n}) = \{y \in S: y(t_i) \in B_{t_i} \forall i = 1..n\}$$
Это пересечения каких-то элементарных цилиндров. Фактически, мы фиксируем значение
случайного процесса в нескольких моментах времени, а не только в одном, как это делалось
в случае элементарного цилиндра.

Пусть $\myM$~-- множество цилиндров.
Заметим, что $\myM$~--  $\pi$-система, причем $\sigma(\myM) = \myB_T$.

$(\Rightarrow)$ Пусть $t_1, \dots, t_n \in T, B_{t_1}, \dots, B_{t_n}, B_{t_i} \in \myB_{t_i}$.
Тогда
$$
\begin{array}{rllll}
P^X_{t_1, \dots, t_n}(B_{t_1} \times \dots \times B_{t_n}) &=&
P( (X_{t_1}, \dots, X_{t_n}) \in B_{t_1} \times \dots \times B_{t_n}) = \\
&&P(X \in C(t_1, \dots, t_n, B_{t_1}, \dots, B_{t_n})) = \\
&&P_X(C(t_1, \dots, t_n, B_{t_1}, \dots, B_{t_n})) = \\
&&P_Y(C(t_1, \dots, t_n, B_{t_1}, \dots, B_{t_n})) = \\
&&P( (Y_{t_1}, \dots, Y_{t_n}) \in B_{t_1} \times \dots \times B_{t_n}) =
P^Y_{t_1, \dots, t_n}(B_{t_1} \times \dots \times B_{t_n})
\end{array}
$$
Доказали, что распределения $X$ и $Y$ совпадают на прямоугольниках.
Но $B_{t_1} \times \dots \times B_{t_n}$~-- порождающая $\pi$-система
для $\myB_{t_1} \otimes \dots \otimes \myB_{t_n} \Rightarrow$
по единственности продолжения меры $P^x_{t_1 \dots t_n} = P^y_{t_1 \dots t_n}$.

$(\Leftarrow)$
$$
\begin{array}{l}
P_X(C(t_1, \dots, t_n, B_{t_1}, \dots, B_{t_n})) = \\
P^X_{t_1 \dots t_n}(B_{t_1} \times \dots \times B_{t_n}) = \\ % ???
P^Y_{t_1 \dots t_n}(B_{t_1} \times \dots \times B_{t_n}) = \\ % ???
P_Y(C(t_1, \dots, t_n, B_{t_1}, \dots, B_{t_n}))
\end{array}
$$

Доказали, что $P_x$ и $P_Y$ совпадают на $\myM$. Но $\myM$~-- порождающая
$\pi$-система для $\myB_T \Rightarrow$ по единствености продолжения меры
$P_X = P_Y$.
\end{proof}

Пусть $P_{t_1 \dots t_n}, n \in \myN, t_1 \dots t_n \in T$~-- конечномерное распределение $(X_t, t \in T)$.

\begin{lem}
Для $\{P_{t_1 \dots t_n}\}$ выполнены условия симметрии $(1)$ и согласованности $(2)$:
\begin{enumerate}
\item $\forall n \; \forall t_1 \dots t_n \in T \; \forall \sigma$~-- перестановки $\{1 .. n\}$ выполнено
$$P_{t_1 \dots t_n}(B_{t_1} \times \dots \times B_{t_n}) =
P_{t_{\sigma_1} \dots t_{\sigma_n}} (B_{t_{\sigma_1}} \times \dots \times B_{t_{\sigma_n}})
$$

\item $\forall n \; \forall t_1 \dots t_n \in T$
$$P_{t_1 \dots t_n}(B_{t_1} \times \dots \times B_{t_{n-1}} \times S_{t_n}) =
P_{t_1 \dots t_{n-1}}(B_{t_1} \times \dots \times B_{t_{n-1}})$$
\end{enumerate}

\begin{proof} \forcenewline
\begin{enumerate}
\item $P_{t_1 \dots t_n}(B_{t_1} \times \dots \times B_{t_n})
= P(X_{t_1} \in B_{t_1}, \dots, X_{t_n} \in B_{t_n})$~-- не зависит от перестановки.
\item $\{X_{t_n} \in S_{t_n}\} =  \Omega$, поэтому это событие ничего не добавляет в пересечение.
\end{enumerate}

Пусть теперь $X$~--  вещественный процесс, т.е. $(S_t, \myB_t) = (\myR, \myB(\myR)) \; \forall t \in T$.
\begin{theorem}[Колмогорова о существовании случайного процесса, б/д]
Пусть $T$~-- некоторое множество, $\forall n \in \myN \; \forall t_1 \dots t_n \in T$ задана
вероятностная мера
$P_{t_1 \dots t_n}$ на $(\myR^n, \myB(\myR^n))$,
причем для системы $\{P_{t_1, \dots t_n}\}$  выполнены условия симметрии и согласованности.
Тогда $\exists$ вероятностное пространство $(\Omega, \myF, P)$
и вещественный случайный процесс $(X_t, t \in T)$
т.ч. $\{P_{t_1 \dots t_n}\}$~-- это его конечномерные распределения.
\end{theorem}
\end{proof}
\end{lem}

% 26 sep 2013
\begin{theorem}[условия симметрии и согласованности для характеристических функций]
Пусть $T$~-- некоторое множество, $\forall t_1, \dots, t_n \in T$ задана вер. мера
$P_{t_1, \dots, t_n}$ на $(\myR^n, \myB(\myR^n)$ с х.ф. $\phi_{t_1, \dots, t_n}$.
Тогда $\{P_{t_1, \dots, t_n}, n \in \myN, t_i \in T\}$ обладают условиями симм. и согл. $\Leftrightarrow$
выполнены условия симметрии и согласованности для х.ф.:
\begin{enumerate}
\item $\phi_{t_1, \dots, t_n}(\lambda_1, \dots, \lambda_n) =
\phi_{t_{\sigma_1}, \dots, t_{\sigma_n}}(\lambda_{\sigma_1}, \dots, \lambda_{\sigma_n})
$
\item $\phi_{t_1, \dots, t_n}(\lambda_1, \dots, \lambda_{n-1}, 0) =
\phi_{t_1, \dots, t_{n-1}}(\lambda_1, \dots, \lambda_n)$
\end{enumerate}
\end{theorem}

\begin{corollary}
Пусть $T \subset \myR$, $\forall n \in \myN \; \forall t_1 < \dots < t_n \in T$
задана х.ф. $\phi_{t_1, \dots, t_n}$ в $\myR^n$. Тогда

$$\exists (\Omega, \myF, P) \text{ и случайный процесс } (X_t, t \in T) \text{ т.ч. }
\phi_{t_1, \dots, t_n} \text{ -- х.ф. } (X_{t_1}, \dots, X_{t_n})$$
$$\Leftrightarrow$$
$$\forall m \; \phi_{t_1, \dots, t_n}(\lambda_1, \dots, \lambda_n) \vline_{\lambda_m = 0} = %\big
\phi_{t_1, \dots, t_{m-1}, t_{m+1}, \dots, t_n}(\lambda_1, \dots, \lambda_{m-1}, \lambda_{m+1}, \dots, \lambda_n)$$
\end{corollary}

\begin{proof} \forcenewline
$(\Rightarrow)$ Очевидно из теоремы Колмогорова и теоремы об условиях симметрии и согласованности
для хар. функций

$(\Leftarrow)$ Пусть $s_1, \dots, s_n \in T, s_i \neq s_j$, рассмотрим $t_1 < \dots < t_n$ т.ч.
$t_i = s_{\sigma_i}$ для некоторой перестановки $\sigma$.

Зададим
$$\phi_{s_1, \dots, s_n}(\lambda_1, \dots, \lambda_n) :=
\phi_{t_1, \dots, t_n}(\lambda_{\sigma_1}, \dots, \lambda_{\sigma_n})$$

Проверим условия симметрии и согласованности для таких хар. функций.

Условие симметрии дано по построению. Проверим согласованность.

$$
\begin{array}{l}
\phi_{s_1, \dots, s_n}(\lambda_1, \dots, \lambda_n) \vline_{\lambda_n=0} = \\
 |s_n = t_m \Rightarrow \lambda_n = \lambda_{\sigma_m}| = \\
\phi_{t_1, \dots, t_n}(\lambda_{\sigma_1}, \dots, \lambda_{\sigma_n}) \vline_{\lambda_{\sigma_m} = 0} =\\
|\text{условие следствия}| =\\
\phi_{t_1, \dots, t_{m-1}, t_{m+1}, \dots, t_n}
(\lambda_{\sigma_1}, \dots, \lambda_{\sigma_{m-1}}, \lambda_{\sigma_{m+1}}, \dots, \lambda_{\sigma_n}) = \\
|\text{по построению}| = \\
\phi_{s_1, \dots, s_{n-1}}(\lambda_1, \dots, \lambda_{n-1})
\end{array}
$$

Условия симметрии и согласованности проверяются аналогично,
если есть совпадающие $s_i = s_j$.  По теореме Колмогорова искомый процесс существует.
\end{proof}
