\section{Фильтрации и марковские моменты}
В этой главе считаем, что $T \subset R$.

\begin{definition}
Пусть $(\Omega, \myF, P)$~-- вероятностное пространство. Множество
$\sigma$-алгебр $\myF = (\myF_t, t \in T)$ называется \emph{фильтрацией},
или \emph{потоком $\sigma$-алгебр} на $(\Omega, \myF, P)$, если $\forall s < t, s, t \in t$
$$\myF_s \subset \myF_t \subset \myF$$
\end{definition}
\begin{definition}
Процесс $(X_t, t \in T)$ называется согласованным с $\myF = (\myF_t, t \in T)$, если
$\forall t \in T \; X_t$ является $\myF_t$-измеримым, то есть
$$\sigma(X_t) = \myF_{X_t} \subset \myF_t$$
\end{definition}

\rightoffset{
$\xi$~-- случайная величина. $\myF_\xi$~-- $\sigma$-алгебра, порожденная $\xi$:
$$\myF_\xi = \{\{\xi \in B\}: B \in \myB(\myR)\}$$

}

\newcommand{\Alpha}{\mathfrak{A}}

\begin{definition}[обозначения]
$\{\xi_\alpha\}_{\alpha \in \Alpha}$~-- множество с.в., $\sigma(\xi_\alpha, \alpha \in \Alpha)$~--
$\sigma$-алгебра, порожденная всеми $\xi_alpha$ -  это
$$formule$$

минимальная сиг-алгебра, содержащая все $\myF_{\xi_alpha}$.
\end{definition}

\begin{definition}
Пусть $(X_t, t \in T)$~-- случайный процесс. Его \emph{естественной фильтрацией} называется
$\myF^x = (\myF_t^X, t \in T)$, где $\myF_t^X = \sigma(X_s, s \leq t, s \in T)$.
\end{definition}
\begin{remark}[наблюдение]
Любой процесс согласован со своей естественной фильтрацией.
\end{remark}

\begin{definition}
Отображение $\tau: \Omega \to T \cup \{+\infty\}$ называется
\emph{марковским моментом} относительно фильтрации $\myF = (\myF_t, t \in T)$
на $(\Omega, \myF, P)$, если $\forall t \in T$ выполнено
$$\{\tau \leq t\} \in \myF_t$$

$\tau$ называется \emph{моментом остановки}, если $\tau$ конечен п.н.
\end{definition}

\begin{example}
$(X_n, n \in \myN)$~-- действительный процесс. $\forall B \in \myB(\myR)$

$$\tau_B = min\{n: X_n \in B\}$$

Тогда $\tau_B$~-- марковский момент отн. $\myF^X$.
\end{example}
\begin{proof}
$$\{\tau_B \leq n\} = \bigcup_{k = 1}^{n}\{X_k \in B\} \in \myF_n^x$$
\end{proof}

\underline{Неформальный смысл:} $\tau$~-- марковский момент, момент наступления
события в процессе, если $\forall t$ можно однозначно сказать, что $\tau$ наступило
к моменту времени $t$ или еще нет, только по наблюдениям процеса $X_s$ до момента времени
$t$ включительно.

\begin{theorem}[марковское свойство $W_t$, усиленный вариант]
Пусть $(W_t, t \geq 0)$~-- винеровский процесс. Тогда
$\forall a > 0$ процесс $X_t \ W_{t+a} - W_a$ является винеровским и
не зависит от $\myF_a^W = \sigma(W_s, s \leq a)$.
\end{theorem}

\begin{definition}
Пусть $\tau$~-- марковский момент отн. $ $ на $ $. Тогда сиг-алгеброй эф-тау
называется
$$\myF_\tau = \{A \in \myF: \forall t \in T \{\tau \leq t\} \cap \in \myF_t\}$$
\end{definition}
%\begin{upr}
\begin{enumerate}
\item $\myF_\tau$~-- действительно сиг-алг
\item тау изм отн ф тау
\item если $\tau = t = const$, то $\myF_\tau = \myF_t$
\end{enumerate}
%\end{upr}

\begin{theorem}[строго марковское свойство $W_t$]
Пусть $(W_t, t \geq 0)$~-- винеровский процесс, а $\tau$~-- момент
остановки относительно $\myF^W$. Тогда процесс
$X_t = W_{t_\tau} - W_\tau$ является винеровским и не зависит от $\myF_\tau$.
\end{theorem}
\begin{lem}
\begin{enumerate}
\item Пусть $\xi, \eta$~-- случайные векторы. Тогда $\xi \overset{d}{=} \eta \Leftrightarrow
\forall f$~-- непр., огр.
$$Ef(\xi) Ef(\eta)$$
\item Пусть $xi$~-- случайный вектор. Тогда $\xi$ независимо с некоторым событием $A \Leftrightarrow
\forall f$~-- непр огр
$$E(f(\xi)I_a) = Ef(\xi) P(a)$$
\end{enumerate}
\end{lem}
\begin{proof}
\begin{enumerate}
\item $(\Rightarrow)$ очевидно.
$(\Leftarrow)$  $f(x) = cos(<t,x>)$ или $sin(<t, x>)$ - огр. непр. функции => хф $\xi$
и $\eta$ совпадают => $\xi \overset{d}{=} \eta$
\item $(\Rightarrow)$ очевидно.
$(\Leftarrow)$ $xi$~-- нез. с $A \Leftrightarrow \forall B \in \myB(\myR)
\{\xi \in B\}$ нез. с $A \Leftrightarrow$
$\{\xi \in B\}$ нез. с $A$ и $B$ замкнуто.

$\{\xi \in B\}, B$ замкнуто~-- $\pi$-система и $\sigma$ от нее~--
это $\{\{\xi \in B\}: B \in \myB(\myR)\}$

Для замкнутого $B$ рассмотрим (записи на бумажке)
\end{enumerate}
\end{proof}
\begin{proof}[СМС]
\begin{enumerate}
\item Проверим, что $W_\tau$~-- случайная величина. Рассмотрим $\forall n \in N$
\end{enumerate}
\end{proof}

\begin{theorem}[принцип отражения, б/д]
Пусть $W_t$~-- винеровский процесс, $\tau$~-- момент остановки относительно $\myF^W$. Тогда
процесс $Z_t = \{W_t, t < \tau; 2W_\tau - W_t, t \geq \tau\}$ также является винеровским.

\end{theorem}

Далее будем изучать $\tau_x = \inf\{t: W_t = x\}$~-- первый момент достижения уровня $x$.

\begin{lem}
$\tau_x$~-- момент остановки относительно $\myF^W$.
\end{lem}
\begin{proof}
Считаем, что нам задана непрерывная модификация $W_t$. Кроме того, $x > 0$ (иначе аналогично).

$\{\tau_x > t\} = |\text{ непрерывность тракторий }| = \{\forall s \leq t W_s < x\} =
\bigcup_{k=1}^{\infty} \{s \leq t: W_s \leq x - \frac{1}{k}\} =
|\text{ непр. траект. }| =
\bigcup_{k=1}^{\infty} \bigcap_{s \leq t, s \in \myQ} \{W_s \leq x - 1/k\} \in \myF_t^W$.
Значит, $\tau_x$~-- действительно марковский момент.

Из ЗПЛ известно, что
$\limsup_{t \to \infty}\frac{W_t}{\sqrt{2t \ln \ln t}} = 1$ п.н., то есть трактория п.н. растет
неограниченно вверх. Тогда в силу непрерывности $\exists t: W_t = x \Rightarrow
P(\tau_x < +\infty) = 1$.
\end{proof}

\underline{Вывод:} для $\tau_x$ выполнены строго марковское свойство и принцип отражения.

\begin{corollary}
$M_t = max_{0 \leq s \leq t} W_s$ является с.в. $\myF_t^W$-измеримой, причем
$\{\tau_x \leq t\} = \{M_t \geq x\}, x \geq 0$.
\end{corollary}

\begin{theorem}
$\forall x,y \geq 0$
$$P(W_t < y - x, M_t \geq y) = P(W_t > y + x)$$
\end{theorem}
\begin{proof}
Если $y = 0$, то утверждение тривиально. Если же $y > 0$, то рассмотрим $\tau_y$.
Это момент остановки, значит, к нему применим принцип отражения.

$Z_t = \{W_t, t \leq \tau_y; 2W_\tau - W_t, t \geq \tau_y\}$ является винеровским.

Обозначим через $\sigma_y$ первый момент достижения $y$ у $Z_t$. Тогда
$(W_t, \tau_y) \overset{d}{=} (Z_t, \sigma_y) \Rightarrow
P(W_t < y-x, M_t \geq y) = P(W_t < y-x, \tau_y \leq t) =
P(Z_t < y-x, \sigma_y \leq t) = |\sigma_y = \tau_y| =
P(Z_t < y-x,  \tau_y \leq t) = |\text{ по опр. }Z_t| =
P(W_\tau - W_t \leq y-x, \tau_y \leq t) =
P(2y - W_t \leq y-x, \tau_y \leq t) =
P(W_t \geq y+x, \tau_y \leq t) = P(W_t > y+x)$.
\end{proof}

\begin{corollary}[теорема Башелье]
$$M_t \overset{d}{=} |W_t|$$
\end{corollary}
\begin{proof}
$P(M_t \geq y) = P(M_t \geq y, W_t < y) + P(M_t \geq y, W_t \geq y) =
P(W_t > y) + P(M_t \geq y, W_t \geq y) =
P(W_t > y) + P(W_t \geq y) = 2P(W_t \geq y) = P(|W_t \geq y|)$.
\end{proof}